\chapter{ចរន្តអគ្គិសនី រេស៊ីស្ទង់ និងកម្លាំងអគ្គិសនីចលករ}
\section{មេរៀនសង្ខេប}
\subsection{អនុភាព និងថាមពល}
\begin{enumerate}[m,2]
	\item អនុភាពអគ្គិសនី $P_{e}=VI$
	\item ថាមពលអគ្គិសនី $W_{e}=P_{e}t=VIt$
\end{enumerate}
\subsection{អង្គធាតុចម្លង់អូម}
\begin{enumerate}[m]
	\item តង់ស្យុងរវាងគោលអង្គធាតុចម្លងអូម $V=RI$
	\item អានុភាពអគ្គិសនីអង្គធាតុចម្លងអូម $P_{e}=VI=RI^{2}$
	\item ថាមពលអគ្គិសនីអង្គធាតុចម្លងអូម $W_{e}=P_{e}t=VIt=RI^{2}t$
	\item ទិន្នផលអង្គធាតុចម្លងៈ អង្គធាតុចម្លងផ្តល់ទិន្នផល $100\%$
\end{enumerate}
\subsection{គ្រឿងទទួលគំរូ(ម៉ូទ័រ និងផើងវិភាគ):} លក្ខខណ្ឌដើម្បីឲ្យគ្រឿងទទួល ដំណើរការៈ $V>E'$
\begin{enumerate}
	\item តង់ស្យុងរវាងគោលគ្រឿងទទួលគំរូៈ $V=E'+Ir'$
	\item អានុភាពអគ្គិសនីគ្រឿងទទួលគំរូៈ $P_{e}=P_{U}+P_{J}=E'I+r'I^{2}$
	\item ទិន្នផលគ្រឿងទទួលគំរូៈ $R_{d}=\frac{W_{U}}{W_{e}}=\frac{P_{U}}{P_{e}}=\frac{E'}{V}$
\end{enumerate}
\subsection{ករណីម៉ូទ័រគាំង ឬផើងវិភាគមានបាតុភូតអាណូតរលាយៈ  $V=r'I$}
\subsection{ជនិតាចរន្តជាប់ៈ(អាគុយ និងថ្មពិល) ភ្លើង $DC$}
\begin{enumerate}
	\item តង់ស្យុងគោលជនិតាចរន្តជាប់ៈ $V=E-Ir$
	\begin{itemize}
		\item បើ $r=0$ នោះ $V=E=$ថេរ ជនិតាជាប្រភពអុីដេអាល់នៃតង់ស្យុង។
		\item បើ $I=0$ នោះ $V=E$ ជាសៀគ្វីចំហរគ្មានចរន្តឆ្លងកាត់។
		\item បើ $V=0$ នោះ $I_{CC}=\frac{E}{r}$(បាតុភូតឆ្លងភ្លើង)
	\end{itemize}
	\item អានភាពអគ្គិសនី នៃជនិតា $P_{g}=P_{e}+P_{J}$
	\begin{flalign*}
		\text{ដែល}\quad: &\quad P_{g}=EI\quad\text{ជាអនុភាពគីមីនៃជនិតាគិតជា $W$}\\
		\quad:&\quad P_{e}=VI\quad\text{ជាអនុភាពអគ្គិសនីបញ្ចេញដោយជនិតាគិតជា $W$} \\
		\quad:&\quad P_{J}=rI^{2}\quad\text{ជាអានុភាពកម្តៅបញ្ចេញដោយជនិតាគិតជា $W$}
	\end{flalign*}
	\item ថាមពលអគ្គិសនី នៃជនិតាៈ $W_{g}=W_{e}+W_{J}$
	\item ទិន្នផលជនិតាចរន្តជាប់ៈ $R_{d}=\frac{W_{e}}{W_{g}}=\frac{P_{e}}{P_{g}}=\frac{V}{E}$
\end{enumerate}
\subsection{ចំណុះអាគុយ-វដ្តអាគុយៈ}
\begin{enumerate}
	\item អាគុយពេលសាកភ្លើង(បញ្ចូលភ្លើង): $V=E+Ir$
	\item អាគុយពេលប្រើៈ $V=E-Ir$
\end{enumerate}
\subsection{ចរន្តអគ្គិសនីៈ}
\begin{enumerate}
	\item ចរន្តឆ្លងកាត់ខ្សែចម្លងៈ $I=\frac{q}{t}$
	\item ទំនាក់ទំនង់រវាងចរន្ត និងល្បឿនអេឡិចត្រុងៈ $I=nAve$
\end{enumerate}
\begin{flalign*}
	\text{ដែល}\quad: &\quad n=\frac{V}{n}\quad\text{ជាចំនួនអេឡិចត្រុងក្នុងមួយខ្នាតមាឌ $m^{-3}$}\\
	\quad:&\quad N\quad\text{ជាចំនួនអេឡិចត្រុងសរុបឆ្លងកាត់ផ្ទៃ $A$} \\
	\quad:&\quad V=AL\quad\text{ជាមាឌខ្សែ $m^{2}$}\\
	\quad:&\quad v\quad\text{ជាល្បឿនអេឡិចត្រុង $m/s$}
\end{flalign*}
\begin{center}
	{\Large \kml\color{magenta} ចប់ដោយសង្ខេប!!!}
\end{center}