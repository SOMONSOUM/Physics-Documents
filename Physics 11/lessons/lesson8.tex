\chapter{ចលនាក្នុងប្លង់}
\section{លំហាត់អនុវត្តន៍}
\begin{enumerate}[m]
		\item ចំណុចរូបធាតុមួយផ្លាស់ទីពីទីតាំងទី១ ដែល $\vec{r_1} = \left(-3.0m\right)\vec{i}+\left(2.0m\right)\vec{j}$ ទៅទីតាំងទី២ ដែល $\vec{r_2}=\left(9.0m\right)\vec{i}+\left(3.0m\right)\vec{j}$ ។\\
		រកបម្លាស់ទីរបស់ចំណុចរូបធាតុដែលផ្លាស់ទីពីទីតាំងទី១ ទៅទីតាំងទី២ ព្រមទាំងគូសក្រាបបញ្ជាក់ពីបម្លាស់ទី។
		\item តាមលំហាត់ទី១ បើចំណុចរូបធាតុនោះផ្លាស់ទីក្នុងរយៈពេល $\Delta t=2.0s$។ គណនាតម្លៃនៃវុិចទ័រល្បឿនមធ្យមនៃបម្លាស់ទីនោះ។
		\item ចំណុចរូបធាតុមួយផ្លាស់ទីពីចំណុច $A$ ដែល $\vec{r_A}\left[\left(0.0m\right)\vec{i}+\left(2.0m\right)\vec{j}\right]$ ទៅចំណុច $B$ ដែល $\vec{r_B}\left[\left(3.0m\right)\vec{i}+\left(6.0m\right)\vec{j}\right]$ ក្នុងរយៈពេល $2.0s$។
		\begin{enumerate}[k]
			\item គូសទីតាំង $A$ និងទីតាំង $B$ នៃចំណុចរូបធាតុ។
			\item គណនាបម្លាស់ទីពី $A$ ទៅ $B$។
			\item គណនាវិុទ័រល្បឿនមធ្យមរបស់ចំណុចរូបធាតុ។
		\end{enumerate}
		\item ចល័តមួយផ្លាស់ទីពីទីតាំងទី១ $x_1=\left(2+5t\right)m$ និង $y_1=\left(-4+2t\right)m$ ទៅទីតាំងទី២ $x_2=\left(4+5t\right)m$ និង $y_2=\left(-4-2t\right)m$។ គណនាបម្លាស់ទីនៃចល័តនោះនៅខណៈ $t=2.0s$ ។
		\item នៅខណៈ $t$ វិុទ័រល្បឿន $\vec{v}=\left(5.0m/s\right)\vec{i}+\left(2.0m/s\right)\vec{j}$ ។ ចូររកតម្លៃនៃវិុទ័រល្បឿននៅខណៈនោះ។
		\item គេចោលគ្រាប់ក្រូសមួយដោយល្បឿនដើម $v_0=2m/s$ ដែលមានទិសបង្កើតជាមួយទិសដេកបានមុំ $30^\circ$ ។
		\begin{enumerate}[k]
			\item សរសេរសមីការគន្លង
			\item គណនា $y$ បើ $x=2m$ ។
		\end{enumerate}
		\item នៅខណៈ $t=0$ គេទាត់បាល់មួយចេញពីចំណុច $0$ ដោយវុិចទ័រល្បឿនដែលមានទិសបង្កើតបានមុំ $45^\circ$ ធៀបនឹងអ័ក្សដេក $\vv{Ox}$ និងមានតម្លៃ $v=8.0m/s$ ។ គណនា៖
		\begin{enumerate}[k]
			\item ចម្ងាយធ្លាក់ ។
			\item កម្ពស់ឡើង​ ។
			\item ខណៈដែលបាល់ទៅដល់កំពូល $S$ នៃប៉ារ៉ាបូល និងកន្លែងបាល់ធ្លាក់ ។
		\end{enumerate}
			\item យន្តហោះជួយសង្រ្គោះមួយ ហោះតាមទិសដេកដោយល្បឿនថេរ $180km/h$ នៅរយៈកម្ពស់ $490m$ ពីផ្ទៃទឹក។ អ្នកជួយសង្រ្គោះចង់ចាកចេញពីយន្តហោះទៅជួយស្រង់អ្នករងគ្រោះម្នាក់ដោយគាត់លិចទូក ដែលកំពុងព្យាយាមហែលទឹក។ គេចាត់ទុកកម្លាំងទប់នៃខ្យល់លើអ្នកជួយសង្រ្គោះអាចចោលបាន។
			\begin{enumerate}[k]
				\item តើមុំ $\alpha$ មានតម្លៃស្មើនឹងប៉ុន្មាន?
				\item នៅខណៈដែលអ្នកជួយសង្រ្គោះមកដល់ផ្ទៃទឹក\\ តើវិុចទ័រល្បឿនមានតម្លៃស្មើនឹងប៉ុន្មាន?
				តើវុិចទ័រល្បឿនមានទិសបង្កើតជាមួយខ្សែដេកបានមុំ $\theta$ មានតម្លៃស្មើនឹងប៉ុន្មាន? គេឲ្យ៖ $g=9.8m/s^2$
			\end{enumerate}
			\begin{figure}[H]
				\centering
				\begin{tikzpicture}[x=1.0cm,y=1.0cm, scale=.80]
				\draw [shift={(0,4)},color=xfqqff,fill=xfqqff,fill opacity=0.1] (0,0) -- (-90:0.53) arc (-90:-45:0.53) -- cycle;
				\draw [->,color=qqqqff] (0,4) -- (5,4);
				\draw [->,color=qqqqff] (0,4) -- (0,-1);
				\draw [->,color=qqqqff] (0,4) -- (5,-1);
				\draw [->,color=qqqqff] (-0.48,2.54) -- (-0.48,4.02);
				\draw [->,color=qqqqff] (-0.5,1.7) -- (-0.5,-0.04);
				\draw [->,color=qqqqff] (1.52,-0.34) -- (0.1,-0.36);
				\draw [->,color=qqqqff] (2.26,-0.34) -- (4.02,-0.34);
				\draw [->,color=qqqqff] (-0.02,4.24) -- (0.98,4.22);
				\draw (0.3,5.08) node[anchor=north west] {$\vv{v_0}$};
				\draw (-0.75,2.49) node[anchor=north west] {$h$};
				\draw (1.60,-0.01) node[anchor=north west] {$x$};
				\draw (4.0,-0.47) node[anchor=north west] {$ \vv{v} $};
				\draw (-0.60,4.40) node[anchor=north west] {$O$};
				\draw (-0.55,-0.55) node[anchor=north west] {$y$};
				\draw (4.60,3.92) node[anchor=north west] {$x$};
				\draw (0.17,3.44) node[anchor=north west] {$ \alpha $};
				\draw [color=qqqqff] (5,0)-- (0,0);
				\draw (4.50,0.009) node[anchor=north west] {$ \beta $};
				\draw [shift={(4.3,-0.13)}] plot[domain=-1.91:1.24,variable=\t]({1*0.14*cos(\t r)+0*0.14*sin(\t r)},{0*0.14*cos(\t r)+1*0.14*sin(\t r)});
				\end{tikzpicture}
				\caption{គន្លងចលនាអ្នកជួយសង្រ្គោះ}
			\end{figure}
		\item អង្គធាតុមួយមានចលនាវង់ស្មើដោយល្បឿនថេរ $10m/s$។ គន្លងវង់នោះមានកាំ $15m$។\\ រកសំទុះចូរផ្ចិតនៃចលនារបស់អង្គធាតុនោះ។
		\item ចល័តមួយផ្លាស់ទីលើរង្វង់មួយដែលមានកាំ $5m$ ដោយចលនាស្មើ។ វាវិលបាន $2$ ជុំក្នុងរយៈពេល $4s$។
		\begin{enumerate}[k]
			\item រករយៈពេលដែលចល័តនោះវិលបានមួយជុំ។
			\item គណនាល្បឿនរង្វិលរបស់ចល័ត។
			\item គណនាសំទុះចូរផ្ចិត។
		\end{enumerate}
		\item ចល័តមួយផ្លាស់ទីតាមទិសដែលបង្កើតបានមុំ $30^\circ$ ជាមួយទិសដេក។ ដោយវិុចទ័រល្បឿន $v=35m/s$។ ចូររកវិុចទ័រល្បឿន $v_x$ តាមទិសដេក និងតាមទិសឈរ $v_y$។
		\item រថភ្លើងមួយផ្លាស់ទីក្នុងពេលមានភ្លៀងនិងខ្យល់សំដៅទិសខាងត្បូងដោយល្បឿនថេរ $27.0m/s$ ធៀបនឹងដី។ អ្នកសង្កេតម្នាក់ដែលឈរនៅលើដីឃើញតំណក់ទឹកភ្លៀងធ្លាក់មានទិសបង្កើតជាមួយទិសឈរបានមុំ $60^\circ C$ ។ អ្នកសង្កេតម្នាក់ទៀតនៅអង្គុយក្នុងរថភ្លើងឃើញតំណក់ទឹកភ្លៀងធ្លាក់តាមទិសឈរ។ ចូរកំណត់ល្បឿនតំណក់ទឹកភ្លៀងធ្លាក់ធៀបនឹងដី។
	\end{enumerate}