\chapter{មូលដ្ឋានគ្រឹះខ្លះៗនៃគណិតវិទ្យា}
\section{ស្វ័យគុណ}
\quad ស្វ័យគុណត្រូវបានប្រើជាញឹកញាប់នៅក្នុងរូបវិទ្យា ពេលយើងសរសេរ $3^{4}$ ដែល $4$ ហៅថាស្វ័យគុណ ហើយ $3$ ជាគោល។ យើងអាចអានបានថា $3$ ស្វ័យគុណ $4$។
\begin{formula}
	\begin{enumerate}[m,2]
		\item $a^{0}=1\quad \left(a\ne 0\right)$
		\item $a^{n}=a\times a\times a \times \cdots \times a\quad\left(a\ne 0\right)$
%		\emph{ឧទាហរណ៍ទី១ៈ} $10^{4}=10\times10\times10\times10=10000$\\
%		\emph{ឧទាហរណ៍ទី២ៈ} $10^{2}=10\times10=100$
		\item $a^{-n}=\frac{1}{a^{n}}\quad \left(a\ne 0\right)$
		\item $a^{m}\cdot a^{n}=a^{m+n}\quad \left(a\ne 0, n\ne 0, m\ne 0\right)$
		\item $\frac{a^{m}}{a^{n}}=a^{m-n}\quad \left(a\ne 0, n\ne 0, m\ne 0\right)$
		\item $\left(a\cdot b\right)^{n}=a^{n}\cdot b^{n}\quad \left(n\ne 0\right)$
		\item $\left(a^{m}\right)^{n}=\left(a^{n}\right)^{m}=a^{m\cdot n}\quad \left(a\ne 0, n\ne 0, m\ne 0\right)$
		\item $\left(\frac{a}{b}\right)^{n}=\frac{a^{n}}{b^{n}}\quad \left(b\ne 0, n\ne 0\right)$
	\end{enumerate}
\end{formula}
\section{ឯកលក្ខណៈភាពសំខាន់ៗ}
\begin{formula}
	\begin{enumerate}[m,2]
		\item $\left(a+b\right)^{2}=a^{2}+2ab+b^{2}$
		\item $\left(a-b\right)^{2}=a^{2}-2ab+b^{2}$
		\item $\left(a+b\right)^{3}=a^{3}+3a^{2}b+3ab^{2}+b^{3}$
		\item $\left(a+b\right)^{3}=a^{3}-3a^{2}b+3ab^{2}-b^{3}$
		\item $a^{2}-b^{2}=\left(a-b\right)\left(a-b\right)$
		\item $a^{2}+b^{2}=\left(a+b\right)^{2}-2ab$
		\item $a^{3}-b^{3}=\left(a-b\right)\left(a^{2}+ab+b^{2}\right)$
		\item $a^{3}+b^{3}=\left(a+b\right)\left(a^{2}-ab+b^{2}\right)$
	\end{enumerate}
\end{formula}
\section{លក្ខណៈនៃប្រភាគពីរស្មើគ្នា}
\begin{generality}
	ឧបមាថាយើងមានប្រភាគពីរស្មើគ្នា $\frac{a}{b}=\frac{c}{d}$។ យើងអាចសរសេរបានដូចខាងក្រោមៈ
	\begin{enumerate}[m,2]
		\item $\frac{d}{b}=\frac{c}{a}$ (ប្តូរតួចុង)
		\item $\frac{a}{c}=\frac{b}{d}$ (ប្តូរតួមធ្យម)
		\item $a\cdot d=b\cdot c$ (ផលគុណតួចុងស្មើនឹងផលគុណតួមធ្យម)
		\item $\frac{a}{b}=\frac{c}{d}=\frac{a\pm c}{b\pm d}$ (លក្ខណៈផលធៀបស្មើតគ្នា)
	\end{enumerate}
\end{generality}
\section{សមីការបន្ទាត់}
\begin{formula}
	សមីការបន្ទាត់មានរាង $y=ax+b$ ដែល $a$ ជាមេគុណប្រាប់ទិស និង $b$ ជាចំនួនថេរ។ បើ $b=0$ នោះសមីការបន្ទាត់មានរាង $y=ax$ គេថាបន្ទាត់កាត់តាមគល់ $0$។
	\begin{align*}
		\text{មេគុណប្រាប់ទិសនៃបន្ទាត់គឺ}\quad :&\quad a=\frac{\Delta y}{\Delta x}=\frac{y_{2}-y_{1}}{x_{2}-x_{1}}
	\end{align*}
\end{formula}
\section{ទម្រង់ស្តង់ដានៃស្វ័យគុណ}
 ទម្រង់ស្តង់ដានៃស្វ័យគុណរបស់ចំនួនមួយគឺជាផលគុណនៃចំនួន $A$ ដែល $1\le A<10$ នឹងស្វ័យគុណ $10$។ ដូចនេះទម្រង់ស្តង់ដាមានរាង $A\times10^{n}$ ដែល $1\le A<10$ ហើយ $n$ ជាចំនួនគត់រុឺឡាទីប។
 \begin{example}
 	សរសេរចំនួនខាងក្រោមជាទម្រង់ស្តង់ដាៈ
 	\begin{enumerate}[k,2]
 		\item $550~000~000=55\times10^{7}$
 		\item $0.000~000~343=343\times10^{-9}$
 		\item $0.000~000~000~004mm=4\times10^{-12}mm$
 		\item $300~000km/s=3\times10^{5}km/s$
 	\end{enumerate}
 \end{example}
\section{ទ្រឹស្តីបទកូសុីនុស និងសុីនុស}
\begin{theorem}
	\begin{multicols}{2}
		\emph{\kml $\bullet$ ទ្រឹស្តីបទកូសុីនុស}
		\begin{align*}
		a^{2}=b^{2}+b^{2}-2bc\cos\alpha\\
		b^{2}=a^{2}+c^{2}-2ac\cos\beta\\
		c^{2}=a^{2}+b^{2}-2ab\cos\gamma\\
		\end{align*}
		\emph{\kml $\bullet$ ទ្រឹស្តីបទសុីនុស}
		\begin{align*}
		\frac{a}{\sin \alpha}=\frac{b}{\sin \beta}=\frac{c}{\sin \gamma}=2R\\ \text{$R$ ជាកាំរង្វង់ចរឹកក្រៅត្រីកោណ}
		\end{align*}
		\emph{\kml $\bullet$ ផលបូកមុំក្នុងនៃត្រីកោណៈ} $\alpha + \beta + \gamma=180^\circ$
		\begin{figure}[H]
			\centering
			\begin{tikzpicture}[scale=1]
			\RectTri{(0,4)}{(1,0)}{5cm}
			\end{tikzpicture}
			\caption{ត្រីកោណនៃទ្រឹស្តីបទកូសុីនុស និងសុីនុស}
		\end{figure}
	\end{multicols}
\end{theorem}
\section{ផលគុណស្កាលែនៃពីរវុិចទ័រ}
	\begin{multicols}{2}
	\emph{\kml ផលគុណស្កាលែនៃពីរវុិចទ័រៈ} បើគេមានវុិចទ័រពីរ $\overrightarrow{A}$ និង $\overrightarrow{B}$ ដែលផ្គុំគ្នាបានមុំ $\theta$ ដូចរូបខាងស្តាំ។ \newline
	នោះគេអាចសរសេរ
		\begin{align*}
			\text{គេសរសេរ}\quad :&\quad\overrightarrow{A}\cdot\overrightarrow{B}=\abs{\overrightarrow{A}}\abs{\overrightarrow{B}}\cos\theta\\
			\text{ម្យ៉ាងទៀត}\quad :&\quad \quad\overrightarrow{A}\cdot\overrightarrow{B}=AB\cos\theta
		\end{align*}
		\begin{figure}[H]
			\centering
			\begin{tikzpicture}
			\begin{scope}
			\coordinate (O) at (0,0);
			\coordinate (A) at (2,2);
			\coordinate (B) at (3,0);
			\coordinate (C) at (4,2);
			\draw [->] (O) -- (A);
			\draw [->] (O) -- (B);
			\draw [dashed] (A) -- (2,0);
			\coordinate[label=above:$\overrightarrow{A}$] (A) at (A);
			\coordinate[label=below:$\overrightarrow{B}$] (B) at (B);
			\pic [draw, -, "$\theta$", angle eccentricity=1.5] {angle = B--O--A};
			\end{scope}
			\end{tikzpicture}
			\caption{\koc ផលគុណស្កាលែនៃពីរវុិចទ័រ}
		\end{figure}
	\end{multicols}
\section{ធរណីមាត្រក្នុងប្លង់ និងអនុគមន៍ត្រីកោណមាត្រ}
\subsection{ការេ}