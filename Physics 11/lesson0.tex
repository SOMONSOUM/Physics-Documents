\chapter{មូលដ្ឋានគ្រឹះខ្លះៗនៃគណិតវិទ្យា}
\section{ស្វ័យគុណ}
\quad ស្វ័យគុណត្រូវបានប្រើជាញឹកញាប់នៅក្នុងរូបវិទ្យា ពេលយើងសរសេរ $3^{4}$ ដែល $4$ ហៅថាស្វ័យគុណ ហើយ $3$ ជាគោល។
\begin{formula}
	\begin{enumerate}[m,2]
		\item $a^{0}=1\quad \left(a\ne 0\right)$
		\item $a^{n}=a\times a\times a \times \cdots \times a\quad\left(a\ne 0\right)$
%		\emph{ឧទាហរណ៍ទី១ៈ} $10^{4}=10\times10\times10\times10=10000$\\
%		\emph{ឧទាហរណ៍ទី២ៈ} $10^{2}=10\times10=100$
		\item $a^{-n}=\frac{1}{a^{n}}\quad \left(a\ne 0\right)$
		\item $a^{m}\cdot a^{n}=a^{m+n}\quad \left(a\ne 0, n\ne 0, m\ne 0\right)$
		\item $\frac{a^{m}}{a^{n}}=a^{m-n}\quad \left(a\ne 0, n\ne 0, m\ne 0\right)$
		\item $\left(a\cdot b\right)^{n}=a^{n}\cdot b^{n}\quad \left(n\ne 0\right)$
		\item $\left(a^{m}\right)^{n}=\left(a^{n}\right)^{m}=a^{m\cdot n}\quad \left(a\ne 0, n\ne 0, m\ne 0\right)$
		\item $\left(\frac{a}{b}\right)^{n}=\frac{a^{n}}{b^{n}}\quad \left(b\ne 0, n\ne 0\right)$
		\item $\sqrt[n]{a^{m}}=a^{\frac{m}{n}}$ និង $\sqrt[n]{a}\times\sqrt[n]{b}=\sqrt[n]{a\times b}$
		\item $\frac{\sqrt[n]{a}}{\sqrt[n]{b}}=\sqrt[n]{\frac{a}{b}}$
	\end{enumerate}
\end{formula}
\section{ឯកលក្ខណៈភាពសំខាន់ៗ}
\begin{formula}
	\begin{enumerate}[m,2]
		\item $\left(a+b\right)^{2}=a^{2}+2ab+b^{2}$
		\item $\left(a-b\right)^{2}=a^{2}-2ab+b^{2}$
		\item $\left(a+b\right)^{3}=a^{3}+3a^{2}b+3ab^{2}+b^{3}$
		\item $\left(a+b\right)^{3}=a^{3}-3a^{2}b+3ab^{2}-b^{3}$
		\item $a^{2}-b^{2}=\left(a-b\right)\left(a-b\right)$
		\item $a^{2}+b^{2}=\left(a+b\right)^{2}-2ab$
		\item $a^{3}-b^{3}=\left(a-b\right)\left(a^{2}+ab+b^{2}\right)$
		\item $a^{3}+b^{3}=\left(a+b\right)\left(a^{2}-ab+b^{2}\right)$
	\end{enumerate}
\end{formula}
\section{លក្ខណៈនៃប្រភាគពីរស្មើគ្នា}
\begin{generality}
	ឧបមាថាយើងមានប្រភាគពីរស្មើគ្នា $\frac{a}{b}=\frac{c}{d}$។ យើងអាចសរសេរបានដូចខាងក្រោមៈ
	\begin{enumerate}[m,2]
		\item $\frac{d}{b}=\frac{c}{a}$ (ប្តូរតួចុង)
		\item $\frac{a}{c}=\frac{b}{d}$ (ប្តូរតួមធ្យម)
		\item $a\cdot d=b\cdot c$ (ផលគុណតួចុងស្មើនឹងផលគុណតួមធ្យម)
		\item $\frac{a}{b}=\frac{c}{d}=\frac{a\pm c}{b\pm d}$ (លក្ខណៈផលធៀបស្មើតគ្នា)
	\end{enumerate}
\end{generality}
\section{សមីការបន្ទាត់}
\begin{formula}
	សមីការបន្ទាត់មានរាង $y=ax+b$ ដែល $a$ ជាមេគុណប្រាប់ទិស និង $b$ ជាចំនួនថេរ។ បើ $b=0$ នោះសមីការបន្ទាត់មានរាង $y=ax$ គេថាបន្ទាត់កាត់តាមគល់ $0$។
	\begin{align*}
		\text{មេគុណប្រាប់ទិសនៃបន្ទាត់គឺ}\quad :&\quad a=\frac{\Delta y}{\Delta x}=\frac{y_{2}-y_{1}}{x_{2}-x_{1}}
	\end{align*}
\end{formula}
\section{ទម្រង់ស្តង់ដានៃស្វ័យគុណ}
 ទម្រង់ស្តង់ដានៃស្វ័យគុណរបស់ចំនួនមួយគឺជាផលគុណនៃចំនួន $A$ ដែល $1\le A<10$ នឹងស្វ័យគុណ $10$។ ដូចនេះទម្រង់ស្តង់ដាមានរាង $A\times10^{n}$ ដែល $1\le A<10$ ហើយ $n$ ជាចំនួនគត់រុឺឡាទីប។
 \begin{example}
 	សរសេរចំនួនខាងក្រោមជាទម្រង់ស្តង់ដាៈ
 	\begin{enumerate}[k,2]
 		\item $550~000~000=55\times10^{7}$
 		\item $0.000~000~343=343\times10^{-9}$
 		\item $0.000~000~000~004mm=4\times10^{-12}mm$
 		\item $300~000km/s=3\times10^{5}km/s$
 	\end{enumerate}
 \end{example}
\section{ទ្រឹស្តីបទកូសុីនុស និងសុីនុស}
\begin{theorem}
	\begin{multicols}{2}
		\emph{\kml $\bullet$ ទ្រឹស្តីបទកូសុីនុស}
		\begin{align*}
		a^{2}=b^{2}+b^{2}-2bc\cos\alpha\\
		b^{2}=a^{2}+c^{2}-2ac\cos\beta\\
		c^{2}=a^{2}+b^{2}-2ab\cos\gamma\\
		\end{align*}
		\emph{\kml $\bullet$ ទ្រឹស្តីបទសុីនុស}
		\begin{align*}
		\frac{a}{\sin \alpha}=\frac{b}{\sin \beta}=\frac{c}{\sin \gamma}=2R\\ \text{$R$ ជាកាំរង្វង់ចរឹកក្រៅត្រីកោណ}
		\end{align*}
		\emph{\kml $\bullet$ ផលបូកមុំក្នុងនៃត្រីកោណៈ} $\alpha + \beta + \gamma=180^\circ$
		\begin{figure}[H]
			\centering
			\begin{tikzpicture}[scale=1]
			\RectTri{(0,4)}{(1,0)}{5cm}
			\end{tikzpicture}
			\caption{ត្រីកោណនៃទ្រឹស្តីបទកូសុីនុស និងសុីនុស}
		\end{figure}
	\end{multicols}
\end{theorem}
\section{ផលគុណស្កាលែនៃពីរវុិចទ័រ}
	\begin{multicols}{2}
	\emph{\kml ផលគុណស្កាលែនៃពីរវុិចទ័រៈ} បើគេមានវុិចទ័រពីរ $\overrightarrow{A}$ និង $\overrightarrow{B}$ ដែលផ្គុំគ្នាបានមុំ $\theta$ ដូចរូបខាងស្តាំ។ \newline
	នោះគេអាចសរសេរ
		\begin{align*}
			\text{គេសរសេរ}\quad :&\quad\overrightarrow{A}\cdot\overrightarrow{B}=\abs{\overrightarrow{A}}\abs{\overrightarrow{B}}\cos\theta\\
			\text{ម្យ៉ាងទៀត}\quad :&\quad \quad\overrightarrow{A}\cdot\overrightarrow{B}=AB\cos\theta\\
			\text{បើ}\quad :&\quad\overrightarrow{A}\cdot\overrightarrow{B}=0 \quad \text{នោះ}\quad \overrightarrow{A}\perp\overrightarrow{B}\\
			\text{ដែល}\quad :&\quad  \abs{\overrightarrow{A}}=A\quad \text{និង}\quad \abs{\overrightarrow{B}}=B\quad \text{ហៅថាណម ឬប្រវែងនៃវុិចទ័រ}
		\end{align*}
		\begin{figure}[H]
			\centering
			\begin{tikzpicture}[scale=1.3]
			\begin{scope}
			\coordinate (O) at (0,0);
			\coordinate (A) at (2,2);
			\coordinate (B) at (3,0);
			\coordinate (C) at (4,2);
			\draw [->] (O) -- (A);
			\draw [->] (O) -- (B);
			\draw [dashed] (A) -- (2,0);
			\coordinate[label=above:$\overrightarrow{A}$] (A) at (A);
			\coordinate[label=below:$\overrightarrow{B}$] (B) at (B);
			\pic [draw, -, "$\theta$", angle eccentricity=1.5] {angle = B--O--A};
			\end{scope}
			\end{tikzpicture}
			\caption{ផលគុណស្កាលែនៃពីរវុិចទ័រ}
		\end{figure}
	\end{multicols}
\section{ធរណីមាត្រក្នុងប្លង់ និងអនុគមន៍ត្រីកោណមាត្រ}
\subsection{ការេ}
គេមានការេ $ABCD$ ដែលមានជ្រុង $a$ ដូចរូប។ គេបាន
\begin{multicols}{2}
	\begin{align*}
		\text{ជ្រុង}\quad :&\quad \abs{AB}=\abs{BC}=\abs{CD}=\abs{DA}=a\\
		\text{អង្កត់ទ្រូង}\quad :&\quad \abs{AC}=\abs{BD}=a\sqrt{2}\\
		\text{ពីកំពូលទៅផ្ចិត}\quad :&\quad \abs{AO}=\abs{BO}=\abs{CO}=\abs{DO}=\frac{a\sqrt{2}}{2}\\
		\text{បរិមាត្រ}\quad :&\quad P=4a\\
		\text{ផ្ទៃក្រឡា}\quad :&\quad S=a\cdot a=a^{2}
	\end{align*}
	\begin{figure}[H]
		\centering
		\begin{tikzpicture}
			\begin{scope}
				\coordinate[label=below left:$A$] (A) at (0,0);
				\coordinate[label=below right:$B$] (B) at (3,0);
				\coordinate[label=above right:$C$] (C) at (3,3);
				\coordinate[label=above left:$D$] (D) at (0,3);
				\coordinate[label=below:$O$] (O) at (1.5,1.5);
				\coordinate[label=above:$a$] (a) at (1.5,3);
				\draw (A)--(B)--(C)--(D)--cycle;
				\draw[dashed] (A)--(C);
				\draw[dashed] (B)--(D);
			\end{scope}
		\end{tikzpicture}
		\caption{ការេ}
	\end{figure}
\end{multicols}
\subsection{ចតុកោណកែង}
គេមានចតុកោណកែង $ABCD$ ដែលមានទទឹង​ $a$ និងបណ្តោយ $b$ ដូចរូប។ គេបាន
\begin{multicols}{2}
	\begin{align*}
	\text{ជ្រុង}\quad :&\quad \abs{AD}=\abs{BC}=a,~\abs{AB}=\abs{DC}=b\\
	\text{អង្កត់ទ្រូង}\quad :&\quad \abs{AC}=\abs{BD}=\sqrt{a^{2}+b^{2}}\\
	\text{បរិមាត្រ}\quad :&\quad P=2a+2b\\
	\text{ផ្ទៃក្រឡា}\quad :&\quad S=a\cdot b
	\end{align*}
	\begin{figure}[H]
		\centering
		\begin{tikzpicture}
		\begin{scope}
		\coordinate[label=below left:$A$] (A) at (0,0);
		\coordinate[label=below right:$B$] (B) at (4,0);
		\coordinate[label=above right:$C$] (C) at (4,3);
		\coordinate[label=above left:$D$] (D) at (0,3);
		\coordinate[label=below:$O$] (O) at (2,1.5);
		\coordinate[label=above:$b$] (b) at (2,3);
		\coordinate[label=right:$a$] (a) at (4,1.5);
		\draw (A)--(B)--(C)--(D)--cycle;
		\draw[dashed] (A)--(C);
		\draw[dashed] (B)--(D);
		\end{scope}
		\end{tikzpicture}
		\caption{ចតុកោណកែង}
	\end{figure}
\end{multicols}
\subsection{ប្រភេទនៃត្រីកោណ}
\begin{enumerate}[m]
	\item \emph{\kml ត្រីកោណសាមញ្ញ}
	\begin{multicols}{2}
		គេមានត្រីកោណ $ABC$ ដែលមានកម្ពស់ $h$ ដូចរូប។
		\begin{align*}
		\text{យើងបាន}\quad :&\quad \text{ផ្ទៃក្រឡា}=\frac{\text{បាត}\times \text{កម្ពស់}}{2}\\
		\text{គេអាចសរសេរ}\quad :&\quad S=\frac{AC\times h}{2}\\
		\text{មុំ}\quad :&\quad \alpha + \beta + \theta =180^\circ
		\end{align*}
			\begin{figure}[H]
			\centering
			\begin{tikzpicture}
			\begin{scope}
				\coordinate[label=below left:$A$] (A) at (-1,0);
				\coordinate[label=above right:$B$] (B) at (4,3);
				\coordinate[label=below right:$C$] (C) at (6,0);
				\coordinate[label=below:$H$] (H) at (4,0);
				\coordinate[label=right:$h$] (h) at (4,1.5);
				\draw (A)--(B)--(C)--cycle;
				\draw (B)--(H);
			\end{scope}
			\end{tikzpicture}
			\caption{ត្រីកោណសាមញ្ញ}
		\end{figure}
	\end{multicols}
	\item \emph{\kml ត្រីកោណកែង} គេមានត្រីកោណកែង $ABC$ ដែលមានកម្ពស់ $h$ ដូចរូប។
	\begin{multicols}{2}
		\begin{align*}
			\text{យើងបានក្រឡាផ្ទៃ}\quad :&\quad  S=\frac{AC\times h}{2}\\
			\text{មុំ}\quad :&\quad \alpha + \beta + \theta =180^\circ\\
			\text{ដែល}\quad :&\quad \theta = 90^\circ
		\end{align*}
		\begin{figure}[H]
			\centering
			\begin{tikzpicture}
				\begin{scope}
					\coordinate[label=below left:$A$] (A) at (-1,0);
					\coordinate[label=above right:$B$] (B) at (4,3);
					\coordinate[label=below right:$C$] (C) at (4,0);
%					\coordinate[label=below:$H$] (H) at (4,0);
					\coordinate[label=right:$h$] (h) at (4,1.5);
					\draw (A)--(B)--(C)--cycle;
%					\draw (B)--(H);
				\end{scope}
			\end{tikzpicture}
			\caption{ត្រីកោណកែង}
		\end{figure}
	\end{multicols}
	\item \emph{\kml ត្រីកោណសមបាត} គេមានត្រីកោណសមបាត $ABC$ ដូចរូប។ យើងបាន
	\begin{multicols}{2}
		\begin{align*}
			\text{ជ្រុង}\quad :&\quad \abs{AB}=\abs{BC}=\abs{AC}\times\frac{\sqrt{2}}{2}\\
			\text{កម្ពស់}\quad :&\quad \abs{BH}=\abs{AH}=\abs{HC}=\frac{AC}{2}\\
			\text{មុំ}\quad :&\quad \alpha + \beta + \theta =180^\circ\\
			\text{ដែល}\quad :&\quad \theta = \beta = 45^\circ
		\end{align*}
		\begin{figure}[H]
			\centering
			\begin{tikzpicture}
				\begin{scope}
				\coordinate[label=below left:$A$] (A) at (0,0);
				\coordinate[label=above right:$B$] (B) at (3,3);
				\coordinate[label=below right:$C$] (C) at (6,0);
				\coordinate[label=below:$H$] (H) at (3,0);
				\coordinate[label=right:$h$] (h) at (3,1.5);
				\draw (A)--(B)--(C)--cycle;
				\draw (B) -- (H);
				\end{scope}
			\end{tikzpicture}
			\caption{ត្រីកោណសមបាត}
		\end{figure}
	\end{multicols}
	\item \emph{\kml ត្រីកោណសម័ង្ស} គេមានត្រីកោណសម័ង្ស $ABC$ ដូចរូប។ យើងបានៈ
		\begin{multicols}{2}
			\begin{align*}
			\text{ជ្រុង}\quad :&\quad \abs{AB}=\abs{BC}=\abs{AC}=a\\
			\text{កម្ពស់}\quad :&\quad \abs{BH}=\frac{a\sqrt{3}}{2}\\
			\text{មុំ}\quad :&\quad \alpha + \beta + \theta =180^\circ\\
			\text{ដែល}\quad :&\quad \theta = \beta = \alpha=60^\circ
			\end{align*}
			\begin{figure}[H]
				\centering
				\begin{tikzpicture}
				\begin{scope}
				\coordinate[label=below left:$A$] (A) at (0,0);
				\coordinate[label=above right:$B$] (B) at (2,4);
				\coordinate[label=below right:$C$] (C) at (4,0);
				\coordinate[label=below:$H$] (H) at (2,0);
				\coordinate[label=right:$h$] (h) at (2,1.5);
				\coordinate[label=right:$a$] (a) at (3.3,1.5);
				\draw (A)--(B)--(C)--cycle;
				\draw (B) -- (H);
				\end{scope}
				\end{tikzpicture}
				\caption{ត្រីកោណសមបាត}
			\end{figure}
		\end{multicols}
\end{enumerate}