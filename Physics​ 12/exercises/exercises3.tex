\section{សំណួរ និងលំហាត់អនុវត្ត}
\begin{enumerate}
	\item ចូរអនុវត្តច្បាប់ទី១ ទែម៉ូឌីណាមិចក្នុងលំនាំអាដ្យាបាទិច។
	\item ចូររៀបរាប់វគ្គទាំងបួននៃដំណើរការម៉ាសុីនម៉ាស៊ូត។
	\item ចូររៀបរាប់ដំណើរប្រព្រឹត្តទៅនៃសុិចកាកណូ។
	\item ដូចម្តេចដែលហៅថា ម៉ូទ័រចំហេះក្រៅ? ម៉ូទ័រចំហេះក្នុង?
	\item ធ្វើដូចម្តេចដើម្បីតម្លើងទិន្នផលម៉ាសុីនកម្តៅ?
	\item ម៉ាសុីនកាកណូ ៣ $\left(a,b,c\right)$ ដំណើរការចន្លោះសីតុណ្ហភាពៈ $(a) 400K$ និង $500K$ $\left(b\right) 500K$ និង $600K$ $\left(c\right) 400K$ និង $600K$។ ម៉ាសុីននីមួយៗស្រូបបរិមាណកម្តៅដូចៗគ្នាពីធុងក្តៅរាល់សុិច។ ចូររៀបតម្លៃកម្មន្តដែលធ្វើដោយម៉ាសុីនទាំងបីតាមលំដាប់ពីធំទៅតូច។
	\item តើប្រភពក្តៅ និងប្រភពត្រជាក់របស់ម៉ាសុីនសាំងបន្ទុះបួនវគ្គស្ថិតនៅត្រង់តំបន់ណា? ចូរពន្យល់?
	\item កម្មន្តដែលធ្វើលើឧស្ម័នក្នុងរយៈពេលនៃលំនាំអាដ្យបាទិចគឺ $140J$។ គណនាកំណើនថាមពលក្នុងនៃប្រព័ន្ធជាកាឡូរី។
	\item ម៉ាសុីនអុីដេអាល់មួយបានបំពេញកម្មន្ត $300J$។ យើងដឹងថាម៉ាសុីនបានបញ្ចេញកម្តៅទៅមជ្ឈដ្ឋានក្រៅ $600J$។ តើម៉ាសុីននោះមានទិន្នផលប៉ុន្មាន?
	\item ម៉ាសុីនកាកណូស្រូបកម្តៅ $1200cal$ ក្នុងរយៈពេលមួយសុិចនិងដំណើរការនៅចន្លោះសីតុណ្ហភាព $500K$ និង $300K$។
	\begin{enumerate}
		\item គណនាទិន្នផលនៃម៉ាសុីន។
		\item គណនាកម្តៅដែលម៉ាសុីនបានបញ្ចេញចោល។
		\item គណនាកម្មន្តដែលបានធ្វើក្នុងរយៈពេលមួយសិុចជាស៊ូល។
	\end{enumerate}
	\item ម៉ាសុីនកាកណូមានដំណើរការនៅចន្លោះសីតុណ្ហភាព $T_{h}=850K$ និង $T_{c}=300K$។ ក្នុងសុិចនីមួយៗម៉ាសុីនបានបំពេញកម្មន្ត $1200J$ ក្នុងរយៈពេល $0.25s$។
	\begin{enumerate}
		\item គណនាទិន្នផលនៃម៉ាសុិន។
		\item គណនាតម្លៃមធ្យមនៃអានុភាពរបស់ម៉ាសុីន។
		\item គណនាបរិមាណកម្តៅដែលផ្តល់ដោយធុងដែលមានសីតុណ្ហភាពខ្ពស់។
		\item គណនាបរិមាណកម្តៅដែលផ្តល់ដោយធុងដែលមានសីតុណ្ហភាពទាប។
	\end{enumerate}
	\item ម៉ូទ័រសាំងនៃរថយន្តរេណូល{\en(Renault)} បានទទួលកម្តៅ $2\times10^{5}J/s$ ដើម្បីឲ្យមានបន្ទុះក្នុងកាប៊ុយរ៉ង់។ វាបានបញ្ចេញកម្តៅ $1.3\times10^{5}J/s$ ទៅមជ្ឈដ្ឋានក្រៅ។
	\begin{enumerate}
		\item គណនាកម្មន្តដែលធ្វើដោយពិស្តុងក្នុងរយៈពេល $1$ វិនាទី។
		\item គណនាទិន្នផលកម្តៅនៃម៉ាសុីន។
		\item គេដឹងថាទិន្នផលមេកានិចគឺ $0.85$។ គណនាកម្មន្តដែលភ្លៅម៉័ទ័របានទទួលក្នុងរយៈពេល $1$ វិនាទី។
	\end{enumerate}  
	\item គណនាកម្មន្តអតិបរមាដែលម៉ាសុីនកាកណូមួយអាចបង្កើតឡើងពេលវាទទួលកម្តៅ $1kcal$ បើវាស្រូបកម្តៅនៅសីតុណ្ហភាព $427^\circ C$ និងបញ្ខេញនៅ $177^\circ C$។
	\item ម៉ាសុីនមួយបញ្ចេញកម្តៅ $8200J$ ខណៈពេលដែលម៉ាសុីនធ្វើកម្មន្តបាន $2600J$។ គណនាទិន្នផលនៃម៉ាសុីននេះ។
	\item ម៉ាសុីនកម្តៅមួយទទួលថាមពល $360J$ ពីធុងក្តៅ និងផ្តល់កម្មន្ត $25J$ ក្នុងវគ្គនីមួយៗ។
	\begin{enumerate}
		\item គណនាទិន្នផលនៃម៉ាសុីន។
		\item គណនាកម្តៅស្រូបដោយធុងត្រជាក់ក្នុងវដ្តនីមួយៗ
	\end{enumerate}
	\item ម៉ាសុីនមួយមានទិន្នផលកម្តៅ $30\%$។ គណនា៖
	\begin{enumerate}
		\item កម្មន្តដែលបានធ្វើ ប្រសិនបើវាស្រូបកម្តៅ $150J$ ពីធុងក្តៅ។
		\item កម្តៅភាយចេញទៅធុងត្រជាក់ក្នុងវដ្តនីមួយៗ។
	\end{enumerate}
	\item ម៉ាសុីនកាកណូធ្វើការរវាងធុងក្តៅពីរនៅសីតុណ្ហភាព $500K$ និង $300K$។ 
	\begin{enumerate}
		\item រកទិន្នផលកម្តៅនៃម៉ាសុីនកាកណូ។
		\item បើវាស្រូបកម្តៅ $200kJ$ ពីធុងក្តៅ។ គណនាកម្មន្តដែលបានធ្វើ។
	\end{enumerate}
	\item ម៉ាសុីនកម្តៅមយមានអានុភាព $580MW$។ គណនាកម្តៅដែលម៉ាសុីនបាត់បង់រាល់វិនាទី បើគេដឹងថាម់ាសុីនមានទិន្នផល $32\%$។
	\item ម៉ូទ័រម៉ាសុីនម៉ាស៊ូតនៃរថយន្តមួយដែលមានទិន្នផលកម្តៅ $0.43$ ហើយស្រូបកម្តៅ $4.0MJ$ ពីប្រភពក្តៅ។\\គណនាៈ
	\begin{enumerate}
		\item កម្មន្តមេកានិចដែលបានពីពិស្តុង។
		\item បរិមាណកម្តៅដែលបញ្ចេញទៅក្នុងបរិយាកាស។
		\item កម្មន្តបានការ បើគេដឹងថាទិន្នផលគ្រឿងបញ្ចូន $0.82$។
	\end{enumerate} 
	\item ម៉ាសុីនកាកណូដែលមានអានុភាព $500W$ ដំណើរការចន្លោះសីតុណ្ហភាព $100^\circ C$ និង $60^\circ C$។ 
	\begin{enumerate}
		\item គណនាថាមពលកម្តៅស្រូបដោយម៉ាសុីនរាល់វិនាទី។
		\item គណនាកម្តៅបញ្ចេញដោយម៉ាសុីនរាល់វិនាទី។
	\end{enumerate}
	\item ម៉ាសុីនកាកណូដំណើរការនៅចន្លោះធុងកម្តៅពីរដែលមានសីតុណ្ហភាព $235^\circ C$ និង $115^\circ C$ ដោយស្រូបកម្តៅ $6.30\times10^{4}J$ រាល់វដ្តពីធុងក្តៅ។
	\begin{enumerate}
		\item គណនាទិន្នផលនៃម៉ាសុីន។
		\item គណនាកម្មន្តដែលម៉ាសុីនបានបំពេញ។
	\end{enumerate}
	\item ម៉ាសុីនម៉ាស៊ូតនៃរថយន្តមួយមានទិន្នផលកម្តៅ $0.40$ ហើយវាស្រូបបរិមាណកម្តៅ $6.0\times10^{6}J$។ គណនាៈ
	\begin{enumerate}
		\item កម្មន្តមេកានិចដែលបានពីពិស្តុង។
		\item បរិមាណកម្តៅដែលបញ្ចេញទៅក្នុងបរិយាកាស។
		\item កម្មន្តបានការ បើទិន្នផលគ្រឿងបញ្ជូនស្មើនឹង $0.8$។
	\end{enumerate}
	\item ម៉ាសុីនអុីដេអាល់មួយដំណើរការនៅចន្លោះធុងកម្តៅពីរដែលមានសីតុណ្ហភាព $500K$ និង $400K$ វាស្រូបកម្តៅ $10.0\times10^{2}J$ ពីធុងដែលមានសីតុណ្ហភាពខ្ពស់ក្នុងរយៈពេលសិុចនីមួយៗ។
	\begin{enumerate}
		\item គណនាទិន្នផលរបស់ម៉ាសុីននោះ។
		\item តើកម្តៅដែលម៉ាសុីនបញ្ចេញទៅមជ្ឈដ្ឋានក្រៅមានតម្លៃប៉ុន្មាន?
	\end{enumerate}
	\item កម្មន្តដែលបំពេញដោយម៉ាសុីនមួយស្មើនឹង $1/4$ នៃកម្តៅស្រូបពីធុងក្តៅ។
	\begin{enumerate}
		\item គណនាទិន្នផលអតិបរមានៃម៉ាសុីន។
		\item តើម៉ាសុីនខាតបង់កម្តៅប៉ុន្មានភាគរយ។
	\end{enumerate}
	\item ម៉ូទ័រម៉ាស៊ូតមួយទទួើលកម្តៅ $3.83MJ$។ វាមានទិន្នផលកម្តៅ $0.45$។
	\begin{enumerate}
		\item គណនាកម្មន្តមេកានិចដែលផ្តល់ដោយពិស្តុង។
		\item តើកម្តៅដែលបញ្ចេញទៅក្នុងបរិយាកាសមានតម្លៃប៉ុន្មាន?
		\item ទិន្នផលគ្រឿងបញ្ចូនគឺ $0.85$។
			  គណនាកម្មន្តដែលបញ្ចូនដោយភ្លៅម៉ូទ័រ។
	\end{enumerate}
	\item ម៉ាសុីនកម្តៅមួយមានអានុភាពចេញ $5.00kW$ និងមានទិន្នផល $25\%$។ ម៉ាសុីនបានបំភាយកម្តៅ $8.00\times10^{3}J$ រាល់វដ្តនីមួយៗ។
	\begin{enumerate}
		\item គណនាកម្តៅស្រូបដោយម៉ាសុីនរាល់វដ្តនីមួយៗ។
		\item គណនារយៈពេលក្នុងមួយវដ្តនៃដំណើរការ។
	\end{enumerate}
	\item ម៉ាសុីនកម្តៅមួយស្រូបកម្តៅ $360J$ ពីធុងក្តៅ និងបំពេញកម្មន្ត $25.0J$ ក្នុងវដ្តនីមួយៗ។
	\begin{enumerate}
		\item គណនាទិន្នផលនៃម៉ាសុីន។
		\item គណនាកម្តៅភាយទៅធុងត្រជាក់។
	\end{enumerate}
	\item សីតុណ្ហភាពនៅក្នុងធុងត្រជាក់នៃម៉ាសុីនកាណូគឺ $230^\circ C$។ គណនាសីតុណ្ហភាពនៅក្នុងធុងក្តៅ បើម៉ាសុីនមានទិន្នផល $34\%$។
	\item ម៉ាសុីនកាណូមួយដំណើរការចន្លោះសីតុណ្ហភាព $210^\circ C$ និង $45^\circ C$។ អានុភាពចេញរបស់វាគឺ $910W$។\\ គណនាកម្តៅភាយចេញពីម៉ាសុីនរាល់វិនាទី។
	\item ម៉ាសុីនកាណូមួយមានទិន្នផល $22\%$។ វាដំណើរការចន្លោះធុងកម្តៅពីរដែលមានបម្រែបម្រួលសីតុណ្ហភាព $75.0^\circ C$។
	\begin{enumerate}
		\item គណនាសីតុណ្ហភាពក្នុងធុងក្តៅ។
		\item គណនាសីតុណ្ហភាពក្នុងធុងត្រជាក់។
	\end{enumerate}
	\item ម៉ាសុីនកាកណូមួយមានអានុភាព $520kW$ ដោយស្រូបកម្តៅ $950kcal$ រាល់វិនាទី។ ប្រសិនបើសីតុណ្ហភាពប្រភពក្តៅ $520^\circ C$ ចូរគណនាសីតុណ្ហភាពប្រភពត្រជាក់។
	\item ម៉ាសុីនសាំងមួយដែលមានសុីឡាំងចំនួនបួនមានទិន្នផល $0.22$ និងបំពេញកម្មន្តបាន $180J$ រាល់ជុំក្នុងសុីឡាំងនីមួយៗ។ ប្រសិនបើម៉ាសុីនដំណើរការបាន $25\mathbf{rps}$។
	\begin{enumerate}
		\item គណនាកម្មន្តបំពេញក្នុងមួយវិនាទី។
		\item គណនាកម្តៅសរុបដែលផ្តលឲ្យម៉ាសុីនក្នុងមួយវិនាទី។
		\item ប្រសិនបើចំហេះសាំង $1\ell$ ផ្តល់ថាមពលបាន $32.21MJ$។\\ តើក្នុងសាំងមួយលីត្រអាចប្រើបានក្នុងរយៈពេលប៉ុន្មាន។
	\end{enumerate}
	\item ម៉ាសុីនកម្តៅទី $1$ ទទួលកម្តៅធំជាងម៉ាសុីនទី $2$ បួនដង បានបំពេញកម្មន្តពីរដង និងបញ្ចេញកម្តៅប្រាំពីរដងនៃម៉ាសុីនទី $2$ ទៅធុងត្រជាក់វិញ។ គណនាទិន្នផលនៃម៉ាសុីនទាំងពីរ។
	\item ម៉ាសុីនកាណូមួយដំណើរការចន្លោះសីតុណ្ហភាព $293K$ និង $67K$។ តើវិធីសាស្រ្តណាមួយដែលធ្វើឲ្យទិន្នផលនៃម៉ាសុីនកើនឡើងខ្ពស់ជាង "បង្កើនសីតុណ្ហភាព $10^\circ C$ នៅក្នុងធុងក្តៅ" ឬក៏ "បន្ថយសីតុណ្ហភាព $10^\circ C$ នៅក្នុងធុងត្រជាក់"? ចូរបង្ហាញហេតុផលសាមញ្ញមួយ។
	\item ម៉ាសុីនកម្តៅបញ្ចេញកម្តៅទៅកាន់ធុងដែលមានសីតុណ្ហភាព $340^\circ C$ និងមានទិន្នផលទ្រឹស្តី $36\%$។ តើធុងត្រជាក់មានសីតុណ្ហភាពប៉ុន្មានអង្សា ប្រសិនបើម៉ាសីុនកើនទិន្នផលដល់ $42\%$ និងរក្សាសីតុណ្ហភាពក្នុងប្រភពក្តៅដដែល។
	\item ម៉ាសុីនកាណូមួយដំណើរការចន្លោះប្រភពកម្តៅដែលមានសីតុណ្ហភាព $580^\circ C$ និងមានទិន្នផលអតិបរមា $22\%$។ ដើម្បីបង្កើនទិន្នផលម៉ាសុីនដល់ $42\%$ តើគេត្រូវតម្លើងសីតុណ្ហភាពប្រភពក្តៅដល់ប៉ុន្មានអង្សា បើសីតុណ្ហភាពប្រភពត្រជាក់ត្រូវរក្សា?
	\item ម៉ាសុីនកាណូមួយមានធុងត្រជាក់ដែលមានសីតុណ្ហភាព $17^\circ C$ មានទិន្នផល $40\%$។ តើគេត្រូវតម្លើងសីតុណ្ហភាព ប្រភពក្តៅប៉ុន្មានដើម្បឲ្យទិន្នផលរបស់ម៉ាសុីនកើនដល់ $50\%$។\\ ដោយដឹងថាសីតុណ្ហភាពប្រភពត្រជាក់ត្រូវបានរក្សា។
	\item ម៉ាសុីនកាណូមួយប្រើចំហាយទឹកក្តៅ $100^\circ C$ ដែលជាធុងក្តៅ។ រីឯធុងត្រជាក់គឺជាមជ្ឈដ្ឋានខាងក្រៅ ដែលមានសីតុណ្ហភាព $20^\circ C$។ អត្រាថាមពលដែលត្រូវបានបញ្ចូនទៅកាន់ធុងត្រជាក់មាន $15.4W$។
	\begin{enumerate}
		\item គណនាអានុភាពបានការនៃម៉ាសុីនកម្តៅ។
		\item គណនាចំហាយកំណជាទឹកនៅធុងក្តៅក្នុងរយៈពេល $1.00h$ ហើយកម្តៅឡាតង់ដើម្បីឲ្យចំហាយកំណជាទឺក $L=2.26\times10^{6}J/kg$។
	\end{enumerate}
	\item ម៉ាសុីនកម្តៅមួយ ត្រូវបានតភ្ជាប់ទៅធុងកម្តៅពីរដែលមួយជាអាលុយមីញ៉ូមរលាយនៅសីតុណ្ហភាព $660^\circ C$ និងធុងមួយទៀត គឺដុំបារតនៅសីតុណ្ហភាព $-38.9^\circ C$។ ម៉ាសុីនដំណើរការដោយបង្កកអាលុយមីញ៉ូម $1.0g$ និងរំលាយបារត $15.0g$។ គេដឹងថា បន្សាយកម្តៅម៉ាសអាលុយមីញ៉ូម $3.97\times10^{5}J/kg$ និងបន្សោយកម្តៅម៉ាសបារត $1.18\times10^{4}J/kg$។ គណនាទិន្នផលអតិបរមានៃម៉ាសុីន។
	\item ម៉ាសុីនកាណូប្រើចំហាយទឹកមួយចាប់ផ្តើមដំណើរការរវាងសីតុណ្ហភាពចំហាយ $220^\circ C$ និងសីតុណ្ហភាព $35^\circ C$ ដោយផ្តល់នូវអានុភាព $8hp$(សេះ)។ គណនាកម្តៅស្រូបក្នុងមួយវិនាទីដោយម៉ាសុីនចំហាយ និងកម្តៅបញ្ចេញក្នុងមួយវិនាទីគិតជាកាឡូរី។ បើម៉ាសុីនកាណូដំណើរការរវាងសីតុណ្ហភាពកម្រិតទាំងនេះមានទិន្នផល $30\%$ នៃទិន្នផលកម្តៅ។
	\newpage
	\item ម៉ាសុីន $X$ បានទទួលថាមពលកម្តៅពីធុងដែលមានសីតុណ្ហភាពខ្ពស់ $4$ដងធំ ជាងម៉ាសុីន $Y$។ ម៉ាសុីន $X$ បានធ្វើកម្មន្ត $2$ដង ហើយបានបញ្ចេញថាមពលកម្តៅ $7$ដងដោយធុងដែលមានសីតុណ្ហភាពទាបធំជាងម៉ាសុីន $Y$។
	\begin{enumerate}
		\item គណនាទិន្នផលម៉ាសុីនកម្តៅ $Y$។
		\item គណនាទិន្នផលម៉ាសុីនកម្តៅ $X$។
	\end{enumerate}
	\item សុីឡាំងច្រើនរបស់ម៉ាសុីនសាំងយន្នហោះមួយដំណើរការដោយល្បឿន $2500\mathbf{tr/mn}$ ដោយទទួលថាមពលកម្តៅ $7.89\times10^{3}J$ និងបញ្ចេញថាមពលកម្តៅ $4.58\times10^{3}J$ ក្នុងជុំនីមួយៗនៃម៉ាសុីនយន្នហោះវីឡឺ​​​ប្រឺ​​កាំង។
	\begin{enumerate}
		\item គណនាប្រេងសាំងគិតជាលីត្រក្នុងរយៈពេល $1$ម៉ោងនៃដំណើរការ។\\ ប្រសិនបើកម្តៅចំហេះនៃសាំង $4.03\times10^{7}J/L$។
		\item គណនាអានុភាពមេកានិចដែលម៉ាសុីនផលិតបាន។
		\item គណនាម៉ូម៉ង់ដែលម៉ាសុីនយន្តហោះប្រើលើបន្ទុក។
		\item គណនាអានុភាពមិនបានការដែលបានបញ្ចេញដោយធុងសីតុណ្ហភាពទាប។
	\end{enumerate}
	\item ម៉ាសុីនកាណូមួយផលិតអានុភាពបានការ $150kW$។ ម៉ាសុីននេះបានដំណើរការរវាងសីតុណ្ហភាពពីរគឺ $20^\circ C$ និង $500^\circ C$។
	\begin{enumerate}
		\item គណនាថាមពលជាកម្តៅដែលវាទទួលបានក្នុងរយៈពេល $1$ម៉ោង។
		\item គណនាថាមពលជាកម្តៅដែលវាបាត់បង់ក្នុងរយៈពេល $1$ម៉ោង។
	\end{enumerate}
	\item ម៉ាសុីនកាណូមួយមានអានុភាព $P$។ ម៉ាសុីននេះបានដំណើរការរវាងដំណើរការរវាងសីតុណ្ហភាពពីរគឺ $T_{h}$ និង $T_{c}$។
	\begin{enumerate}
		\item គណនាថាមពលជាកម្តៅដែលចូលក្នុងម៉ាសុីននៅចន្លោះពេល $\Delta t$។
		\item គណនាថាមពលជាកម្តៅដែលបាត់ប៉ង់ក្នុងចន្លោះពេល $\Delta t$។
	\end{enumerate}
	\item ម៉ាសុីនមួយបានបញ្ចូនថាមពលជាកម្តៅ $2\times10^{3}J$ ពីប្រភពធុងសីតុណ្ហភាពខ្ពស់ក្នុងអំឡុងខួបនីមួយៗហើយបានបញ្ចូន $1.5\times10^{3}J$ ទៅប្រភពសីតុណ្ហភាពទាប។
	\begin{enumerate}
		\item គណនាទិន្នផលរបស់ម៉ាសុីន។
		\item គណនាកម្មន្តដែលធ្វើដោយម៉ាសុីនក្នុង $1$ខួប។
		\item គេដឹងថា ម៉ាសុីននេះដំណើរការដោយល្បឿន $2000\mathbf{tr/mn}$។\\
		គណនាអានុភាពមេកានិចដែលម៉ាសុីននោះផលិតបានក្នុង $1$ជុំ។
	\end{enumerate}
	\item កម្មន្តដែលធ្វើដោយម៉ាសុីនស្មើ $1/4$ នៃថាមពលកម្តៅដែលស្រូបចេញពីធុងសីតុណ្ហភាពខ្ពស់។
	\begin{enumerate}
		\item គណនាទិន្នផលកម្តៅរបស់ម៉ាសុីន។
		\item គណនាផលធៀបថាមពលដែលស្រូបនិងថាមពលដែលបញ្ចេញទៅធុងសីតុណ្ហភាពទាប។
	\end{enumerate}
	\item កាំភ្លើងមយយត្រូវបានចាត់ទុកជាម៉ាសុីនកម្តៅ។ គេដឹងថាកាំភ្លើងធ្វើពីដែកដែលមានម៉ាសស្មើ $1.8kg$។ គ្រាប់កាំភ្លើងនេះមានម៉ាស $2.40g$ ហើយពេលបាញ់ចេញមានល្បឿន $320m/s$ និងមានទិន្នផលថាមពលស្មើ $1.10\%$។ សន្មតថាតួ(ដង)កំាភ្លើងស្រូបថាមពលទាំងអស់ដែលបញ្ចេញនិងកើនឡើងសីតុណ្ហភាពស្មើសាច់ក្នុងរយៈពេលខ្លឺមុនពេលបាត់បង់ថាមពលកម្តៅខ្លះទៅក្នុងមជ្ឈដ្ឋានបរិយាកាស។\\ គណនាកំណើនសីតុណ្ហភាពនៅក្នុងគ្រាប់កាំភ្លើង។ គេឲ្យកម្តៅម៉ាសដែក $C_{\text{ដែក}=448J/kg\cdot C^{\circ}}$។
	\item ម៉ាសុីនមួយកើតឡើងពីចំហេះធ្យូងថ្មបង្វិលតួប៊ីននៅជ្រលងស្ទឹង(អូហាយអូ) នៅសហរដ្ឋអាមេរិចដែលដំណើរការរវាងសីតុណ្ហភាពពីរ $1870^\circ C$ និង $430^\circ C$។
	\begin{enumerate}
		\item តើទិន្នផលម៉ាសុីនទ្រឹស្តីអតិបរមាស្មើប៉ុន្មាន?
		\item ទិន្នផលម៉ាសុីនពិតស្មើ $42\%$។ គណនាអនុភាពមេកានិចដែលម៉ាសុីនបានបញ្ចូន ប្រសិនវាស្រូបថាមពលកម្តៅ $1.40\times10^{5}J$ រៀងរាល់វិនាទីពីធុងសីតុណ្ហភាពខ្ពស់។
	\end{enumerate}
	\item ម៉ាសុីនកម្តៅមយយបង្កើតឡើងមានទិន្នផលស្មើម៉ាសុីនកាណូ $65\%$ នៅពេលវាដំណើរការរវាងធុងសីតុណ្ហភាពពីរ។
	\begin{enumerate}
		\item ប្រសិនបើសីតុណ្ហភាពធុងត្រជាក់ស្មើនឹង $20^\circ C$ តើសីតុណ្ហភាពដែលនៅធុងដែលមានសីតុណ្ហភាពខ្ពស់ស្មើប៉ុន្មាន?
		\item តើទិន្នផលម៉ាសុីនពិតអាចស្មើ $65\%$ ដែរឬទេ? ចូរពន្យល់?
	\end{enumerate}
	\item ក្នុងភាពទី១នៃភាពពីររបស់ម៉ាសុីនកាណូមួយថាមពលដែលស្រូប $Q_{1}$ ក្រោមសីតុណ្ហភាព $T_{1}$ ហើយធ្វើកម្មន្ត $W_{1}$ និងបញ្ចេញថាមពលកម្តៅ $Q_{2}$ ក្រោមសីតុណ្ហភាពទាប $T_{2}$។ ភាពទី២ ស្រូបថាមពលកម្តៅ $Q_{2}$ ធ្វើកម្មន្ត $W_{2}$ ហើយបញ្ចេញថាមពលកម្តៅ $Q_{3}$ ក្រោមសីតុណ្ហភាព $T_{3}$។\\
	ចូរបង្ហាញថា ទិន្នផលរបស់ម៉ាសុីនកម្តៅនេះគឺ $e=\frac{T_{1}-T_{3}}{T_{1}}=1-\frac{T_{3}}{T_{1}}$។ 
	\item ទិន្នផលនៃម៉ាសុីនពិត $20\%$ គឺប្រើដើម្បីបង្កើនល្បឿនរបស់រថភ្លើងមួយចេញពីស្ងៀមទៅល្បឿនប្រហែល $5m/s$។ យើងដឹងដា ម៉ាសុីនកាណូមួយបានប្រើប្រភពធុងត្រជាក់និងធុងក្តៅដូចគ្នាដើម្បីពន្លឿនរថភ្លើងដូចគ្នាពីនៅស្ងៀមទៅល្បឿន $6.50m/s$ ដោយប្រើបរិមាណបេ្រងឥន្ធនៈស្មើគ្នា។ ម៉ាសុីននេះប្រើខ្យល់ដែលមានសីតុណ្ហភាព $300K$ ធ្វើជាប្រភពធុងត្រជាក់។\\
	គណនាសីតុណ្ហភាពនៃចំហាយដែលប្រើនៅប្រភពធុងក្តៅ។
	\item ក្នុងមួយខួប ម៉ាសុីនកម្តៅមួយបានស្រូប $500J$ ពីធុងដែលមានសីតុណ្ហភាពខ្ពស់ហើយបញ្ចេញ $300J$ ទៅប្រភពធុងសីតុណ្ហភាពទាប។ គេដឹងថាទិន្នផលម៉ាសុីនស្មើនឹង $60\%$ នៃម៉ាសុីនកាណូ។ គណនាផលធៀបសីតុណ្ហភាពធុងត្រជាក់ និងសីតុណ្ហភាពធុងក្តៅនៃម៉ាសុីនកាណូ។
	\item ម៉ាសុីនកាណូមួយបានដំណើរការរវាងសីតុណ្ហភាព $T_{h}=100^\circ C$ និង $T_{c}=20^\circ C$។ តើផលធៀបទិន្នផលនៃម៉ាសុីនកាណូនេះកើនឡើងស្មើនឹងប៉ុន្មាន? បើសីតុណ្ហភាពធុងក្តៅកើនឡើងដល់ $550^\circ C$។
	\item $1500kW$ នៃម៉ាសុីនកម្តៅមួយបានដំណើរការដោយទិន្នផល $20\%$។ ថាមពលកម្តៅបានបញ្ចេញទៅធុងត្រជាក់ដោយការស្រូបចំហាយទឹកចូលក្នុងរបុំមានសិតុណ្ហភាព $20\%$។ ប្រសិនបើទឹក $60$លីត បានហូរឆ្លងកាត់របុំក្នុង១វិនាទី។ តើកំណើនសីតុណ្ហភាពទឹកស្មើនឹងប៉ុន្មាន? \\គេឲ្យកម្តៅម៉ាសទឹក $C=4186J/kg^\circ C$។ ម៉ាសមាឌទឹក $\rho=10^{3}kg/m^{3}$។
	\item ម៉ាសុីនកាណូមួយបានដំណើរការរវាងសីតុណ្ហភាពពីរគឺ $100^\circ C$ និង $20^\circ C$។
	គណនាម់ាសទឹកកកដែលអាចឲ្យម៉ាសុីនរំលាយបន្ទាប់ពីវាធ្វើកម្មន្ត $5\times10^{4}J$។ គេឲ្យកម្តៅឡាតង់របស់ទឹកកក $L_{f}=3.33\times10^{5}J/kg$
	\item ម៉ាសុីនកាណូមួយមានសីតុណ្ហភាពធុងត្រជាក់ $17^\circ C$ និងមានទិន្នផល $40^\circ C$។ តើសីតុណ្ហភាពធុងក្តៅកើនប៉ុន្មានដើម្បីឲ្យបានទិន្នផលរបស់ម៉ាសុីនកើនបាន $50\%$?
	\item ម៉ាសុីនកាណូមួយបានស្រូបថាមពលកម្តៅ $52kJ$ ហើយបានបញ្ចេញថាមពលកម្តៅ $36kJ$ ក្នុងខួបនីមួយៗ។
	\begin{enumerate}
		\item គណនាទិន្នផលរបស់ម៉ាសុីនកាណូ។
		\item គណនាកម្មន្តដែលបំពេញបានក្នុងខួបនីមួយៗ។
	\end{enumerate}
\end{enumerate}