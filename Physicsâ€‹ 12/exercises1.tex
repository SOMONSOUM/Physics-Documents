\section{លំហាត់}
\begin{enumerate}[m]
	\item ចូរពោលទ្រឹស្តីសុីនេទិចនៃឧស្ម័ន។
	\item ចូរសរសេរសមីការភាពនៃឧស្ម័នបរិសុទ្ធ។
	\item ចូរសរសេររូបមន្តថាមពលសុីនេទិចមធ្យមនៃម៉ូលេគុលឧស្ម័ននីមួយៗ។
	\item ចូរសរសេររូបមន្តថាមពលសុីនេទិចសរុបនៃម៉ូលេគុលឧស្ម័ន។
	\item ចូរសរសេររូបមន្តល្បឿនប្ញសការេនៃការេល្បឿនមធ្យមម៉ូលេគុលឧស្ម័ន។
	\item ក្នុងធុងបិទជិតមួយមានផ្ទុកឧស្ម័នអុកសុីសែន $\left(\ce{O2}\right)~2mol$។\\
	គណនាចំនួនម៉ូលេគុលរបស់ឧស្ម័នអុកសុីសែននេះ បើចំនួនអាវ៉ូកាដ្រូ $N_{A}=6.022\times10^{23}$ ម៉ូលេគុល$/mol$។
	\item ក្នុងធុងបិទជិតមួយមានឧស្ម័នអុីដ្រូសែន $\left(\ce{H2}\right)~0.2mol$ និងមានម៉ាសម៉ូល $2.0g/mol$។\\
	បើគេដឹងថា ចំនួនអាវ៉ូកាដ្រូ $N_{A}=6.022\times10^{23}$ម៉ូលេគុល$/mol$។
	\begin{enumerate}[k]
		\item គណនាចំនួនម៉ូលេគុលអុីដ្រូសែនក្នុងធុងនេះ។
		\item គណនាម៉ាសសរុបរបស់ឧស្ម័នអុីដ្រូសែន។
	\end{enumerate}
	\item ក្នុងធុងបិទជិតមួយមានឧស្ម័ន $0.25mol$ និងមានម៉ាសសរុប $7.0g$។\\
	បើគេដឹងថា ចំនួនអាវ៉ូកាដ្រូ $N_{A}=6.022\times10^{23}$ម៉ូលេគុល$/mol$។
	\begin{enumerate}[k]
		\item គណនាចំនួនម៉ូលេគុលសរុបរបស់ឧស្ម័នក្នុងធុងនេះ។
		\item តើឧស្ម័ននេះជាឧស្ម័នអ្វី?
	\end{enumerate}
	\item ក្នុងធុងបិទជិតមួយមានឧស្ម័នពេញ មានម៉ាសសរុប $64.0g$ និងមានចំនួនម៉ូលេគុលសរុបគឺ $12.044\times10^{23}$ម៉ូលេគុល។\\
	បើគេដឹងថា ចំនួនអាវ៉ូកាដ្រូ $N_{A}=6.022\times10^{23}$ម៉ូលេគុល$/mol$។
	\begin{enumerate}[k]
		\item គណនាចំនួនម៉ូលរបស់ឧស្ម័នក្នុងធុងនេះ។
		\item តើឧស្ម័ននេះជាឧស្ម័នអ្វី?
	\end{enumerate}
	\item ក្នុងធុងបិទជិតមួយមានផ្ទុក ឧស្ម័ន $\ce{H2}$ ពេញមានម៉ាសសរុប $1.0g$។ ដោយឧស្ម័ននេះមានម៉ាសម៉ូល $2.0g/mol$ និងចំនួនអាវ៉ូកាដ្រូ $N_{A}=6.022\times10^{23}$ម៉ូលេគុល$/mol$។
	\begin{enumerate}[k]
		\item គណនាចំនួនម៉ូលេគុលសរុបរបស់ឧស្ម័នក្នុងធុងនេះ។
		\item គណនាចំនួនម៉ូលរបស់ឧស្ម័ន $\ce{H2}$។
	\end{enumerate}
	\item ផង់នីមួយៗមានម៉ាស $m_{0}$ និងផ្លាស់ទីដោយល្បឿន $v$ តាមបណ្តោយអ័ក្ស $\overrightarrow{ox}$។ គេដឹងថាក្នុងផ្ទៃ $1mm^{2}$ និងក្នុង $1s$ មានផង់ចំនួន $10^{15}$ ទៅទង្គិចនឹងផ្ទៃនោះ។
	ចូររកសម្ពាធរបស់ផង់លើផ្ទៃប៉ះ។\\
	គេឲ្យ $m_{0}=9.1\times10^{-31}kg$ និង $v=8\times10^{7}m/s$។ គេសន្មត ទង្គិចរវាងផង់ និងផ្ទៃប៉ះជាទង្គិចស្ទក់។
	\item គេបាញ់ផង់ឲ្យផ្លាស់ទីតាមបណ្តោយអ័ក្ស $\overrightarrow{ox}$ ដែលកែងនឹងផ្ទៃរបស់អេក្រង់មួយ។ គេដឹងថា ផង់នីមួយៗមានម៉ាស $m_{0}$ និងល្បឿន $v_{0}$។ គេដឹងថាក្នុង $1.25mm^{2}$ ផ្ទៃរបស់អេក្រង់មានផង់ចំនួន $4\times10^{14}$ ទៅទង្គិចរៀងរាល់វិនាទី។ \\គេសន្មតថា ទង្គិចនោះជាទង្គិចស្ទក់។ គណនាល្បឿនរបស់ផង់ដែលផ្លាស់ទីតាមអ័ក្ស $\overrightarrow{ox}$។\\ បើគេដឹងថា សម្ពាធដែលកើតឡើងដោយសារការទង្គិចរបស់ផង់លើផ្ទៃអេក្រង់គឺ $P=3.64\times10^{-3}N/m^{2}\\~m_{0}=9.1\times10^{-31}kg$។
	\item ផង់នីមួយមានម៉ាស $m_{0}$ នឹងផ្លាស់ទីដោយល្បឿន $v$ តាមបណ្តោយអ័ក្ស $\overrightarrow{ox}$។ គេដឹងថាក្នុងផ្ទៃ $2mm^{2}$ និងក្នុងមួយវិនាទីមានផង់ចំនួន $2\times10^{15}$ ទៅទង្គិចនឹងផ្ទៃនោះ។ គេឲ្យៈ $m_{0}=9.1\times10^{-31}kg$ និង $v=5\times10^{7}m/s$។ គេសន្មតថា ទង្គិចរវាងផង់ និងផ្ទៃប៉ះជាទង្គិចស្ទក់។
	\begin{enumerate}[k,2]
		\item គណនាកម្លាំងសរុបដែលផង់មានអំពើលើផ្ទៃប៉ះ។
		\item គណនាសម្ពាធសរុបរបស់ផង់លើផ្ទៃប៉ះ។
	\end{enumerate}
	\item ប្រូតុងមួយមានម៉ាស $m_{p}=1.67\times10^{-27}kg$ ផ្លាស់ទីដោយល្បឿន $v$ តាមបណ្តោយអ័ក្ស $\overrightarrow{ox}$ ក្នុងមាឌមួយមានរាងជាគូបដែលទ្រនុងនីមួយៗមានរង្វាស់ $3mm$ ប្រូតុងផ្លាស់ពីផ្ទៃម្ខាងទៀតក្នុង $2ns$។ គេសន្មត់ថា ទង្គិចរវាងប្រូតុង និងផ្ទៃខាងនៃគូបជាទង្គិចស្ទក់។
	\begin{enumerate}[k]
		\item រកល្បឿនដើមប្រូតុង នៅខណៈវាចាប់ផ្តើមចេញពីផ្ទៃខាងនៃគូប។
		\item រកសម្ពាធរបស់ប្រូតុងលើផ្ទៃខាងនៃគូប។
		\item គេដឹងថាក្នុងរយៈពេល $2ns$ មានចំនួនប្រូតុង $2\times10^{6}$ ទៅទង្គិចនឹងផ្ទៃខាងនៃគូប។ រកសម្ពាធសរុបរបស់ប្រូតុងលើផ្ទៃខាងនៃគូប។
	\end{enumerate}
	\item អេឡិចត្រុងមួយមានម៉ាស $m_{e}=9.1\times10^{-31}kg$ ផ្លាស់ទីដោយល្បឿន $v$ តាមបណ្តោយអ័ក្ស $\overrightarrow{ox}$ ក្នុងមាឌមួយមានរាងជាគូបដែលទ្រនុងនីមួយៗមានរង្វាស់ $5mm$ ប្រូតុងផ្លាស់ពីផ្ទៃម្ខាងទៀតក្នុង $25ns$។ \\គេសន្មត់ថា ទង្គិចរវាងប្រូតុង និងផ្ទៃខាងនៃគូបជាទង្គិចស្ទក់។
	\begin{enumerate}[k]
		\item រកល្បឿនដើមអេឡិចត្រុង នៅខណៈវាចាប់ផ្តើមចេញពីផ្ទៃខាងនៃគូប។
		\item រកសម្ពាធរបស់អេឡិចត្រុងលើផ្ទៃខាងនៃគូប។
		\item គេដឹងថាក្នុងរយៈពេល $25ns$ មានចំនួនអេឡិចត្រុង $2\times10^{10}$ ទៅទង្គិចនឹងផ្ទៃខាងនៃគូប។\\ រកសម្ពាធសរុបរបស់អេឡិចត្រុងមានលើផ្ទៃខាងនៃគូប។
	\end{enumerate}
	\item អេឡិចត្រុងមួយមានម៉ាស $m_{e}=9.1\times10^{-31}kg$ ផ្លាស់ទីដោយល្បឿន $v$ តាមបណ្តោយអ័ក្ស $\overrightarrow{ox}$ ក្នុងមាឌមួយមានរាងជាគូបដែលទ្រនុងនីមួយៗមានរង្វាស់ $2mm$ ប្រូតុងផ្លាស់ពីផ្ទៃម្ខាងទៀតក្នុង $25ns$។ គេសន្មត់ថា ទង្គិចរវាងប្រូតុង និងផ្ទៃខាងនៃគូបជាទង្គិចខ្ទាត។
	\begin{enumerate}[k]
		\item រកល្បឿនដើមអេឡិចត្រុង នៅខណៈវាចាប់ផ្តើមចេញពីផ្ទៃខាងនៃគូប។
		\item រកសម្ពាធរបស់អេឡិចត្រុងលើផ្ទៃខាងនៃគូប។
		\item គេដឹងថាក្នុងរយៈពេល $25ns$ មានចំនួនអេឡិចត្រុង $25\times10^{6}$ ទៅទង្គិចនឹងផ្ទៃខាងនៃគូប។\\ រកសម្ពាធសរុបរបស់អេឡិចត្រុងមានលើផ្ទៃខាងនៃគូប។
	\end{enumerate}
	\item អាតូមអុីដ្រូសែនមួយមានម៉ាស $m$ ផ្លាស់ទីដោយល្បឿន $v=1500km/s$ តាមបណ្តោយអ័ក្ស $\overrightarrow{ox}$ ក្នុងមាឌមួយមានរាងគូបដែលទ្រនុងនីមួយមានរង្វាស់ $3mm$។ អុីដ្រូសែន ផ្លាស់ទីពីផ្ទៃម្ខាងទៅម្ខាងទៀត។ គេសន្មតថាសន្មត់ថា ទង្គិចរវាងអុីដ្រូសែន និងផ្ទៃខាងនៃគូបជាទង្គិចខ្នាត។
	\begin{enumerate}[k]
		\item រករយៈពេលដែលអាតូមអុីដ្រូសែនទៅប៉ះនឹងផ្ទៃម្ខាងទៀតនៃគូប។
		\item គេដឹងថាក្នុងរយៈពេល $2ns$ មានចំនួនអាតូមអុីដ្រូសែន $2\times10^{6}$ ទៅទង្គិចនឹងផ្ទៃខាងនៃគូបហើយផ្ទៃខាងរងនៅសម្ពាធសរុប $27.83\times10^{-2}N/m^{2}$។ រកម៉ាសអាតូមអុីដ្រូសែនមួយ។
	\end{enumerate}
	\item ឧស្ម័នបរិសុទ្ធមួយមានមាឌ $V=100cm^{3}$ ស្ថិតក្រោមសម្ពាធ $2.00\times10^{5}Pa$ នៅសីតុណ្ហភាព $20^\circ C$។\\ តើឧស្ម័ននោះមានប៉ុន្មានម៉ូល? $\left(R=8.31J/mol\cdot K\right)$
	\item ឧស្ម័នបរិសុទ្ធមួយមាន $n=0.08\times10^{-1}mol$ មានសម្ពាធ $P=5.00\times10^{5}Pa$ នៅសីតុណ្ហភាព $60^\circ C$។\\ តើឧស្ម័ននោះមានមាឌប៉ុន្មាន?
	\item នៅសីតុណ្ហភាព $293K$ និងសម្ពាធ $5atm$ មេតាន $1kmol$ មានម៉ាស $16.0kg$។ \\គណនាម៉ាសមាឌនៃមេតានក្នុងលក្ខខណ្ឌខាងលើ។
	\item នៅក្នុងបំពង់បិទជិតដែលមានមាឌ $20mL$ នៅសីតុណ្ហភាពកំណត់មួយយ៉ាងទាបមានតំណក់នីត្រូសែនរាវមានម៉ាស $50mg$។ គណនាសម្ពាធនីត្រូសែននៅក្នុងបំពង់នោះ កាលណាបំពង់នោះមានសីតុណ្ហភាព $300K$ ដោយសន្មតថានីត្រូសែននេះជាឧស្ម័នបរិសុទ្ធ។ គេឲ្យៈ $R=8.31J/mol\cdot K$។
	\item ធុងមួយមានផ្ទុកអេល្យូម $2.00mol$ នៅសីតុណ្ហភាព $27^\circ C$។ គេសន្មតថាអេល្យូមជាឧស្ម័នបរិសុទ្ធ។
	\begin{enumerate}[k]
		\item គណនាតម្លៃមធ្យមនៃថាមពលសុីនេទិចរបស់ម៉ូលេគុលនីមួយៗ
		\item គណនាថាមពលសុីនេទិចសរុបរបស់ម៉ូលេគុលទាំងអស់។\\
		គេឲ្យៈ $k_{B}=1.38\times10^{-23}J/K,~R=8.31J/mol\cdot K$។
	\end{enumerate}
	\item នៅក្នុងធុងមួយដែលមានមាឌ $2.00mL$ មានឧស្ម័នដែលមានម៉ាស $50mg$ និងសម្ពាធ $100kPa$។\\ ម៉ាសរបស់មូលេគុលនៃឧស្ម័ននីមួយៗគឺ $8.0\times10^{-26}kg$។
	\begin{enumerate}[k]
		\item រកចំនួនម៉ូលេគុលនៃឧស្ម័ននោះ។
		\item រកតម្លៃមធ្យមនៃថាមពលសុីនេទិចរបស់ម៉ូលេគុលនីមួយៗ។ គេឲ្យៈ $k=1.38\times10^{-23}J/K$
	\end{enumerate}
	\item ចូរគណនាប្ញសការេនៃការេល្បឿនមធ្យមរបស់អាតូមអេល្យូមនៅសីតុណ្ហភាព $20.0^\circ C$។ \\ម៉ាសម៉ូលអេល្យូមគឺ $4.00\times10^{-3}kg/mol$។ គេឲ្យៈ $R=8.31J/mol\cdot K$។
	\item រកប្ញសការេនៃការេល្បឿនមធ្យមរបស់ម៉ូលេគុលអុកសុីសែននៅសីតុណ្ហភាព $200^\circ C$។ \\ម៉ាសម៉ូលអុកសុីសែន $32\times10^{-3}kg/mol$ និង $R=8.31J/mol\cdot K$។
	\item \begin{enumerate}[k]
		\item គណនាម៉ាសម៉ូលេគុលនៃអុីដ្រូសែន។ គេឲ្យម៉ាសម៉ូលគឺ $M=2.00\times10^{-3}kg/mol$ \\និងចំនួនអាវ៉ូកាដ្រូ $N_{A}=6.02\times10^{23}/mol$។
		\item គណនាតម្លៃប្ញសការេនៃការេល្បឿនមធ្យមរបស់ឧស្ម័នអុីដ្រូសែននៅសីតុណ្ហភាព $100^\circ C$។
		\item គណនាតម្លៃមធ្យមនៃថាមពលសុីនេទិចរបស់ម៉ូលេគុលនៃឧស្ម័នអុីដ្រូសែននីមួយៗនៅសីតុណ្ហភាព $100^\circ C$។\\ គេឲ្យៈ $k=1.38\times10^{-23}$។
	\end{enumerate}
	\item ដោយប្រើតម្លៃលេខ $1,3,7$ និង $8$ ចូរបង្ហាញថា ប្ញសការេនៃការេល្បឿនមធ្យម $v_{rms}$ \\ខុសគ្នាពីតម្លៃមធ្យម $v_{av}$ របស់វា។
	\item ចូរកំណត់រកល្បឿន $v_{rms}$ របស់ម៉ូលេគុលឧស្ម័នអុកសុីសែន $\left(O_{2}\right)$ និងអាសូត $\left(N_2\right)$ ក្នុងបន្ទប់មួយដែលមានសីតុណ្ហភាព $20^\circ C$។
	\item \begin{enumerate}[k]
		\item បង្ហាញថាល្បឿន $v_{rms}$ នៃឧស្ម័នបរិសុទ្ធ អាចសរសេរជាទម្រង់មួយទៀតគឺ $v_{rms}=\sqrt{\frac{3P}{\rho}}$ ដែល $\rho$ ជាដង់សុីតេ ឬហៅថាម៉ាសមាឌ ហើយ $P$ ជាសម្ពាធ។
		\item ល្បឿន $v_{rms}$ របស់ម៉ូលេគុលឧស្ម័នមួយប្រភេទស្មើ $450m/s$។\\ ប្រសិនបើវាស្ថិតនៅសម្ពាធបរិយាកាស តើដងសុីតេរបស់ឧស្ម័ននោះស្មើប៉ុន្មាន?
	\end{enumerate}
	\item កែវបាឡុងមួយចំណុះ $1L$ មានអុកសុីសែនជាឧស្ម័នបរិសុទ្ធដែលមានសីតុណ្ហភាព $27^\circ C$ ក្រោមសម្ពាធ $2atm$។\\
	គណនាម៉ាសអុកសុីសែន។ គេឲ្យៈ $O=16$
	\item គេមានខ្យល់មានមាឌ $1m^3$ នៅសីតុណ្ហភាព $18^\circ C$ ក្នុងសម្ពាធបរិយាកាស $P_{1}=1atm$ ទៅបណ្ណែននៅសីតុណ្ហភាពដដែល តែក្នុងសម្ពាធបរិយាកាស $P_{2}=3.5atm$។ គណនាមាឌស្រេចនៃខ្យល់។
	\item ដបមួយផ្ទុកឧស្ម័នមានសម្ពាធ $P_{0}=1.0atm$ នៅសីតុណ្ហភាព $17^\circ C$។\\
	តើគេត្រូវកម្តៅឪ្យឧស្ម័ននេះដល់សីតុណ្ហភាពប៉ុន្មាន ដើម្បីសម្ពាធកើនឡើងដល់ $1.5atm$?
	\item គេយកបំពង់អុកសុីសែនមានចំណុះ $20L$ ក្រោមសម្ពាធ $P_{1}=200atm$ នៅសីតុណ្ហភាព $20^\circ C$ ទៅដាក់ក្នុងបាឡុង កៅស៊ូស្តើងមួយ។\\ គណនាមាឌបាឡុង បើឧស្ម័នក្នុងបាឡុងមានសម្ពាធ $P_{2}=1atm$ និងសីតុណ្ហភាព $9^\circ C$។
	\item \begin{enumerate}[k]
		\item ចូរគណនាល្បឿនប្រសិទ្ធ $\left(v_{rms}\right)$ នៃម៉ួលេគុលឧស្ម័ននីត្រូសែននៅសីតុណ្ហភាព $20^\circ C$។
		\item គណនាសីតុណ្ហភាព ប្រសិនបើល្បឿនប្រសិទ្ធ $\left(v_{rms}\right)$ ថយចុះពាក់កណ្តាល។
		\item គណនាសីតុណ្ហភាព ប្រសិនបើល្បឿនប្រសិទ្ធ $\left(v_{rms}\right)$ កើនឡើងពីរដងវិញ។
	\end{enumerate}
	\item មួយ ម៉ូលេគុលឧស្ម័ននីដ្រូសែនផ្សំឡើងពីអាតូមនីដ្រូសែនពីរ។គណនាម៉ាសម៉ូលេគុលនីត្រូសែន។\\ ម៉ាសម៉ូលនីដ្រូសែនគឺ $M=28kg/kmol$ គេឲ្យ $N_{A}=6.02\times10^{23}$ ម៉ូលេគុល$/mol$
	\item គណនាមាឌឧស្ម័នអុកសុីសែន $3.2g$ ដែលផ្ទុកក្នុងធុងនៅសម្ពាធ $76cmHg$ និងសីតុណ្ហភាព $27^\circ C$។
	\item រកល្បឿនប្រសិទ្ធ $v_{rms}$ នៃម៉ូលេគុលអាសូតដោយម៉ាសម៉ូល $M=28g/mol$ នៅ $300K$។ គេឲ្យៈ $R=8.31J/mol\cdot K$
	\item គណនាសីតុណ្ហភាពដែលធ្វើឲ្យល្បឿនប្រសិទ្ធនៃម៉ូលេគុលអុីដ្រូសែនស្មើ $331m/s$។ គេឲ្យៈ $M_{H_{2}}=2.0g/mol$។ 
	\item គណនាតម្លៃមធ្យមនៃថាមពលសុីនេទិចនៃម៉ូលេគុលឧស្ម័ននៅសីតុណ្ហភាព $727^\circ C$។ គេឲ្យៈ $R=8.31J/mol\cdot K$ និង $N_{A}=6.02\times10^{23}$ម៉ូលេគុល$/mol$។
	\item រកតម្លៃមធ្យមនៃថាមពលសុីនេទិចរបស់ម៉ូលេគុលឧស្ម័នអុកសុីសែននីមួយៗក្នុងខ្យល់នៅក្នុងបន្ទប់មានសីតុណ្ហភាព $300K$ គិតជាអេឡិចត្រុង-វ៉ុល។ គេឲ្យ $1eV=1.6\times10^{-19}J$ និង $k_{B}=1.38\times10^{-23}J/K$
	\item មួយម៉ូលេគុលនីដ្រូសែននៅពេលស្ថិតនៅលើផ្ទៃដីវាកើតមានល្បឿនប្រសិទ្ធ នៅសីតុណ្ហភាព $0^\circ C$។ ប្រសិនបើវាផ្លាស់ទីឡើងត្រង់ទៅលើដោយគ្មានទង្គិចនឹងម៉ូលេគុលផ្សេងទៀត។ ចូរគណនាកម្ពស់ដែលវាឡើងដល់។ \\គេឲ្យម៉ាសមួយម៉ូលេគុលរបស់នីដ្រូសែន $m=4.65\times10^{-26}kg$ និង $g=10m/s^{2}$។
	\item សុីទែនមួយស្ថិតក្រោមលក្ខខណ្ឌស្តង់ដា {\en (STP)} ផ្ទុកឧស្ម័ននីដ្រូសែន $28.5kg$។
	\begin{enumerate}[k]
		\item ចូរគណនាមាឌរបស់សុីទែន។
		\item ប្រសិនបើគេបន្ថែមនីដ្រូសែន $32.2kg$ ទៀតចូលក្នុងសុីទែនដោយរក្សាសីតុណ្ហភាពនៅដដែល។ \\ចូរគណនាសម្ពាធឧស្ម័ននីដ្រូសែនក្នុងសុីទែន។
	\end{enumerate}
	\item បាច់ម៉ូលេគុលអុីដ្រូសែនត្រូវបានបាញ់លើជញ្ជាំងដោយទិសបង្កើតបានមុំ $55^\circ$ ជាមួយនឹងវុិចទ័រឯកតាផ្ទៃ $\left(\overrightarrow{n}\right)$ របស់ជញ្ជាំង។ ម៉ូលេគុលនីមួយៗនៃឧស្ម័នអុីដ្រូសែនមានល្បឿន $1km/s$ និងម៉ាស $3.3\times10^{-24}kg$។ បាច់អុីដ្រូសែនបានទៅទង្គិចនឹងជញ្ជាំងដែលមានផ្ទៃ $2cm^{2}$ ដោយអត្រា $10^{23}$ ម៉ូលេគុលក្នុងមួយវិនាទី។\\ ដោយសន្មតថាទង្គិចនេះ ជាទង្គិចខ្ទាត ចូរគណនាសម្ពាធដែលមានលើជញ្ជាំង។
	\item គេបាញ់ផង់ឲ្យផ្លាសើទីតាមបណ្តោយអ័ក្ស $\overrightarrow{ox}$ ដែលកែងនឹងផ្ទៃរបស់អេក្រង់មួយ។ គេដឹងថាផង់នីមួយៗមានម៉ាស $m_{0}$ និងមានល្បឿន $v$។ គេដឹងថាក្នុង $1.25mm^{2}$ ផ្ទៃរបស់អេក្រង់មានផង់ $4\times10^{14}$ ទៅទង្គិចរៀងរាល់វិនាទី។\\ គេសន្មត់ថា ទង្គិចនោះជាទង្គិចស្ទក់។
	គណនាល្បឿនរបស់ផង់ដែលផ្លាស់ទីតាមតាមអ័ក្ស $\overrightarrow{ox}$។ បើគេដឹងថា សម្ពាធដែលកើតឡើងដោយសារការទង្គិចរបស់ផង់លើផ្ទៃរបស់អេក្រង់គឺ $3.64\times10^{-3}N\cdot m^{-2}$ និង $m_{0}=9.1\times10^{-31}kg$។
	\item ផង់នីមួយៗមានម៉ាស $m_{0}$ និងផ្លាស់ទីដោយល្បឿន $v$ តាមបណ្តោយអ័ក្ស $\overrightarrow{ox}$។ គេដឹងថាក្នុងផ្ទៃ $2mm^{2}$ និងក្នុងមួយវិនាទីមានផង់ចំនួន $2\times10^{15}$ ទៅទង្គិចនឹងផ្ទៃនោះ។ គេឲ្យៈ $m_{0}=9.1\times10^{-31}kg$ និង $v=5.0\times10^{15}m/s$។\\
	គេសន្មតថា ទង្គិចរវាងផង់និងផ្ទៃប៉ះជាទង្គិចស្ទក់។
	\begin{enumerate}[k]
		\item គណនាកម្លាំងសរុបដែលផង់មានអំពើលើផ្ទៃប៉ះ។
		\item គណនាសម្ពាធសរុបរបស់ផង់លើផ្ទៃប៉ះ។
	\end{enumerate}
	\item  ប្រូតុងមួយមានម៉ាស $m_{P}=1.67\times10^{-27}kg$ និងផ្លាស់ទីដោយល្បឿនដើម $\overrightarrow{v}_{0}$ តាមបណ្តោយអ័ក្ស $\overrightarrow{ox}$ ក្នុងធុងមួយមានរាងជាគូប។ គេដឹងថាក្នុងផ្ទៃ $4mm^{2}$ និងក្នុងមួយវិនាទីមានប្រូតុងចំនួន $5\times10^{13}$ ទៅទង្គិចនឹងផ្ទៃនោះហើយសម្ពាធរបស់ប្រូតុងលើផ្ទៃប៉ះគឺ $8.35\times10^{-2}Pa$។ គេសន្មតថាទង្គិចរវាងផង់នឹងផ្ទៃប៉ះជាទង្គិចស្ទក់។
	\begin{enumerate}[k]
		\item គណនាកម្លាំងដែលប្រូតុងនីមួយៗមានអំពើលើផ្ទៃប៉ះ។
		\item គណនាល្បឿនប្រូតុងនៅខណៈវាទៅប៉ះនឹងផ្ទៃម្ខាងទៀតនៃគូប។
	\end{enumerate}
	\item អេឡិចត្រុងមួយមានម៉ាស $m_{e}=9.1\times10^{31}kg$ ផ្លាស់ទីដោយល្បឿន $v$ តាមបណ្តោយអ័ក្ស $\overrightarrow{ox}$។ ក្នុងធុងមួយមានរាងជាគូបដែលទ្រនុងនីមួយៗមានរង្វាស់ $l=5mm$។ អេឡិចត្រុងផ្លាស់ទីពីផ្ទៃម្ខាងទៅផ្ទៃម្ខាងទោក្នុង $25ns$។\\ គេសន្មតថាទង្គិចរវាងអេឡិចត្រុង នឹងផ្ទៃខាងនៃគូបជាទង្គិចស្ទក់។
	\begin{enumerate}[k]
		\item គណនាល្បឿនស្រេចអេឡិចត្រុង នៅខណៈវាទៅប៉ះនឹងផ្ទៃម្ខាងទៀតនៃគូប។
		\item គណនាសម្ពាធរបស់អេឡិចត្រុងមានលើផ្ទៃខាងនៃគូប។
		\item គេដឹងថាក្នុងរយៈពេល $25ns$ មានចំនួនអេឡិចត្រុង $2\times10^{10}$ ទៅទង្គិចនិងផ្ទៃខាងនៃគូប។\\ គណនាសម្ពាធសរុបរបស់អេឡិចត្រុងមានលើផ្ទៃខាងនៃគូប។
	\end{enumerate}
	\item សម្ពាធនៃឧស្ម័ននៅក្នុងធុងមួយមានមាឌ $250mL$ ស្ថិតនៅក្រោមសម្ពាធ $125kPa$ និងថាមពលសុីនេទិចមធ្យមនៃភាគល្អិតនីមួយៗគឺ $1.875\times10^{-21}J$។
	\begin{enumerate}[k]
		\item គណនាចំនួនភាគល្អិតនៃឧស្ម័ននៅក្នុងធុង។
		\item គណនាចំនួនម៉ូលនៃ ឧស្ម័ននៅក្នុងធុង។ គេឲ្យៈ $N_{A}=6.022\times10^{23}$ម៉ូលេគុល$/mol$
	\end{enumerate}
\item ក្នុងធុងមួយមានមាឌ $200mL$ មានម៉ូលេគុលសរុប $5\times10^{21}$ ហើយស្ថិតនៅក្រោមសម្ពាធ $250kPa$។\\ ថេរបុលស្មាន់ $k_{B}=1.38\times10^{-23}J/K$ និង ចំនួនអាវ៉ូកាដ្រូ $N_{A}=6.022\times10^{23}$ម៉ូលេគុល$/mol$
\begin{enumerate}[k]
	\item គណនាថាមពលសុីនេទិចមធ្យមនៃភាគល្អិតនីមួយៗ។
	\item គណនាចំនួនម៉ូលនៃ ឧស្ម័ននៅក្នុងធុង។
	\item គណនាសីតុណ្ហភាពនៃឧស្ម័ននៅក្នុងធុង។
\end{enumerate}
\item ឧស្ម័នបរិសុទ្ធមួយមានមាឌ $V=500cm^{3}$ ស្ថិតក្រោមសម្ជាធ $600kPa$ នៅសីតុណ្ហភាព $27^\circ C$។ \\គណនាចំនួនម៉ូលនៃ ឧស្ម័ននោះ។ គេឲ្យថេរសាកលនៃឧស្ម័ន $R=8.31J/mol\cdot K$
\item ឧស្ម័នបរិសុទ្ធមួយមាន $n=0.25mol$ មានសម្ពាធ $P=250kPa$ នៅសីតុណ្ហភាព $57^\circ C$។ \\តើឧស្ម័ននោះមានមាឌប៉ុន្មាន? គេឲ្យថេរសាកលនៃឧស្ម័ន $R=8.31J/mol\cdot K$
\item ធុងមួយមានផ្ទុកឧស្ម័នអេល្យូម $0.5mol$ នៅសីតុណ្ហភាព $27^\circ C$។ គេសន្មតថាអេល្យូមជាឧស្ម័នបរិសុទ្ធ។\\ គេឲ្យៈ $k_{B}=1.38\times10^{-23}J/K$ និង $R=8.31J/mol\cdot K$។
\begin{enumerate}[k]
	\item គណនាតម្លៃមធ្យមនៃថាមពលសុីនេតិចរបស់ម៉ូលេគុលឧស្ម័ននីមួយៗ។
	\item គណនាថាមពលសុីនេទិចសរុបរបស់ម៉ូលេគុលទាំងអស់។
	\item គណនាសម្ពាធឧស្ម័នអេល្យូមក្នុងធុង​ បើធុងមានមាឌ $4.53\times10^{-3}m^{3}$។
\end{enumerate}
\item \begin{enumerate}[k]
	\item គណនាល្បឿនប្រសិទ្ធនៃម៉ូលេគុលអុកសុីសែននៅសុីតុណ្ហភាព $127^\circ C$។ \\ម៉ាសម៉ូលអុកសុីសែនគឺ $32g/mol$ និង $R=8.31J/mol\cdot K$។
	\item គណនាតម្លៃថាមពលសុីនេទិចមធ្យមនៃម៉ូលេគុលឧស្ម័នអុកសុីសែននីមួយៗ នៅសីតុណ្ហភាព $127^\circ C$។ \\គេឲ្យៈ $k_{B}=1.38\times10^{-23}J/K$
\end{enumerate}
\item \begin{enumerate}[k]
	\item គណនាសីតុណ្ហភាពនៃម៉ូលេគុលអុីដ្រូសែនគិតជា $^\circ C$។\\ បើដឹងថា ល្ពៀនប្រសិទ្ធនៃម៉ូលេគុលអុីដ្រូសែន $v_{rms}=1933.78m\cdot s^{-1}$ ម៉ាសម៉ូលអុីដ្រូសែនស្មើនឹង $2.0g/mol$ និងគេឲ្យៈ $R=8.31J/mol\cdot K;~k_{B}=1.38\times10^{-23}J/K$។
	\item គណនាតម្លៃថាមពលសុីនេទិចមធ្យមនៃម៉ូលេគុលអុីដ្រូសែននីមួយៗ នៅសីតុណ្ហភាពនោះ។​
\end{enumerate}
\item ធុងមួយមានមាឌ $V=2.5mL$ មានផ្ទុកឧស្ម័នដែលមានម៉ាស $50mg$ ស្ថិតក្រោមសម្ពាធ $1035kPa$។ \\ម៉ាសរបស់ម៉ូលេគុលនៃឧស្ម័ននីមួយៗគឺ $8\times10^{-26}kg$។
\begin{enumerate}[k]
	\item គណនាចំនួនម៉ូលេគុលសរុបនៃឧស្ម័ននោះ។ គេឲ្យៈ $k_{B}=1.38\times10^{-23}J/K$។
	\item គណនាតម្លៃថាមពលសុីនេទិចមធ្យមនៃម៉ូលេគុលឧស្ម័ននីមួយៗ
	\item គណនាតម្លៃថាមពលសុីនេទិចសរុបរបស់ម៉ូលេគុលក្នុងធុង។
	\item គណនាសីតុណ្ហភាពនៃឧស្ម័នក្នុងធុង។
\end{enumerate}
\item ឧស្ម័នបរិសុទ្ធមួយមានមាឌ $V=125cm^3$ ស្ថិតក្រោមសម្ពាធ $2\times10^{5}Pa$។\\ គណនាសីតុណ្ហភាពនៃឧស្ម័នបរិសុទ្ធនោះ។ បើគេដឹងថាឧស្ម័ននោះមាន $n=9.4\times10^{-3}mol;~R=8.31J/mol\cdot K$។
\item ធុងមួយមានមាឌ $0.025m^{3}$ ផ្ទុកម៉ាស $0.084kg$ នៃឧស្ម័ននីដ្រូសែន $\ce{N2}$ ស្ថិតនៅក្រោមសម្ពាធ $3.17atm$។\\ គណនាសីតុណ្ហភាពនៃឧស្ម័នគិតជាអង្សារសេ$\left(^\circ C\right)$។ គេឲ្យៈ $1atm=1.013\times10^{5}Pa$ ម៉ាសម៉ូល $M=28g/mol$ និង $R=8.31J/mol\cdot K$។
\item ផង់នីមួយៗមានម៉ាស $m_{0}$ និងផ្លាស់ទីដោយល្បឿន $\overrightarrow{v}$ តាមបណ្តោយអ័ក្ស $\overrightarrow{ox}$។ គេដឹងថាក្នុងផ្ទៃ $5mm^2$ និងក្នុងមួយវិនាទីមានផង់ចំនួន $1\times10^{15}$ ទៅទង្គិចនឹងផ្ទៃនោះ។ គណនាសម្ពាធសរុបរបស់ផង់មានលើផ្ទៃប៉ះ។ គេសន្មតថា ទង្គិចរវាងផង់នឹងផ្ទៃប៉ះជាទង្គិចស្ទក់ ហើយម៉ាសផង់នីមួយៗគឺ $m_{0}=9.1\times10^{-31}kg$ និង $v=8\cdot10^{7}m/s$។
\item គណនាចំនួនម៉ូលេគុលសរុបដែលមាននៅក្នុង $500g$ នៃខ្យល់។\\ បើគេដឹងថាក្នុងខ្យល់មានអុកសុីសែន​ $22\%$ និងមានអាសូត $78\%$ ជាម៉ាស។
\item ក្នុងធុងបិទជិតមួយមានមាឌសរុប $16.62dm^3$ មានផ្ទុកឧស្ម័នបរិសុទ្ធពេញស្ថិតក្រោមសម្ពាធ $3\times10^{5}Pa$ និងមានសីតុណ្ហភាព $47^\circ C$។ គេឲ្យថេរឧស្ម័នបរិសុទ្ធ $R=8.31J/mol\cdot K$។ គណនាចំនួនម៉ូលនៃឧស្ម័នបរិសុទ្ធក្នុងធុងនោះ។
\item ឧស្ម័នបរិសុទ្ធមួយមានម៉ាសម៉ូលេគុលនីមួយៗគឺ $8\times10^{-26}kg$ នៅសីតុណ្ហភាព $57^\circ C$។​\\ គេឲ្យៈ $k_{B}=1.38\times10^{-23}J/K$។
\begin{enumerate}[k]
	\item គណនាប្ញសការេនៃការេល្បឿនមធ្យម $v_{rms}$។
	\item គណនាតម្លៃថាមពលសុីនេទិចមធ្យមនៃម៉ូលេគុលឧស្ម័នបរិសុទ្ធនីមួយៗ។
\end{enumerate}
\item \begin{enumerate}
	\item គណនាម៉ាសម៉ូលេគុលនីមួយៗរបស់ឧស្ម័នអុកសុីសែន។\\ បើគេដឹងថាម៉ាសម៉ូលរបស់វាគឺ $32g/mol$ និង $N_{A}=6.022\times10^{23}$ម៉ូលេគុល$/mol$
	\item គណនាល្បឿនប្រសិទ្ធនៃឧស្ម័នអុកសុីសែនស្ថិតនៅសីតុណ្ហភាព $0^\circ C$។
	\item គណនាតម្លៃថាមពលសុីនេទិចមធ្យមនៃម៉ូលេគុលនីមួយៗ របស់ឧស្ម័នអុកសុីសែននៅសីតុណ្ហភាព $0^\circ C$។\\ គេឲ្យៈ $k_{B}=1.38\times10^{-23}J/K$
\end{enumerate}
\item បាឡុងពីរត្រូវបានតភ្ជាប់គ្នាដោយបំពង់មួយមានរ៉ូពីនេបិទជិត។ ដោយបាឡុងទី១ មានផ្ទុកឧស្ម័នដែលមានសម្ពាធ $5atm$ និងមានមាឌ $6L$ ចំណែកបាឡុងទី២នៅទទេមានមាឌ $4L$។\\ គេចាប់ផ្តើមបើករ៉ូពីនេ(បើគេដឹងថាបាឡុងនីមួយៗមានសីតុណ្ហភាពថេរ)។\\ គណនាសម្ពាធរបស់បាឡុងនីមួយៗ ក្រោយពេលគេបើករ៉ូពីនេ។
\item បាឡុងពីរត្រូវបានតភ្ជាប់គ្នាដោយបំពង់មួយមានរ៉ូពីនេបិទជិត។ ដោយបាឡុងទី១ មានផ្ទុកឧស្ម័នដែលមានសម្ពាធ $6atm$ និងមានមាឌ $5L$ ចំណែកបាឡុងទី២ មានផ្ទុកឧស្ម័នដូចគ្នាដែលមានសម្ពាធ $4atm$ និងមានមាឌ $3L$។\\ គេចាប់ផ្តើមបើករ៉ូពីនេ(បើគេដឹងថាបាឡុងនីមួយៗមានសីតុណ្ហភាពថេរ)។\\ គណនាសម្ពាធរបស់បាឡុងនីមួយៗ ក្រោយពេលគេបើករ៉ូពីនេ។
\item កំណត់សីតុណ្ហភាពដើម្បីឲ្យល្បឿនប្រសិទ្ធនៃម៉ូលេគុលឧស្ម័នអាសូតដែលមានម៉ាសម៉ូល $M_{\left(\ce{N2}\right)}=28g/mol$ ស្មើនឹងល្បឿនប្រសិទ្ធនៃម៉ូលេគុលឧស្ម័នអុកសុីសែន ដែលមានម៉ាសម៉ូល $M_{\left(\ce{O2}\right)}=32g/mol$ នៅសីតុណ្ហភាព $47^\circ C$។
\item គូបមួយមានជ្រុង $10.0cm$ ផ្ទុកខ្យល់ដែលមានម៉ាសម៉ូល $28.9g/mol$ នៅសម្ពាធបរិយាកាស និងសីតុណ្ហភាព $300K$។
\begin{enumerate}[k]
	\item គណនាម៉ាស និងទម្ងន់នៃឧស្ម័នក្នុងរូប។
	\item គណនាកម្លាំងដែលមានអំពើលើផ្ទៃខាងនីមួយៗនៃគូប។
	\item តើហេតុអ្វីបានជាសំណាកដ៏តូចល្អិតមួយអាចបង្កើតកម្លាំងដ៏មហិមានេះបាន?
\end{enumerate}
\item \begin{enumerate}[k]
	\item គណនាចំនួនម៉ូលនៃឧស្ម័នបរិសុទ្ធដែលមានមាឌ $1m^{3}$ នៅសីតុណ្ហភាព $20.0^\circ C$ និងសម្ពាធបរិយាកាស។
	\item ក្នុងមួយម៉ូលនៃម៉ូលេគុលខ្យល់មានម៉ាស $28.9g$។ គណនាម៉ាសខ្យល់ក្នុង $1m^{3}$។
\end{enumerate}
\item ឧស្ម័នអុកសុីសែនមួយម៉ូលមានសម្ពាធ $P_{1}$ នៅសីតុណ្ហភាព $27.0^\circ C$។
\begin{enumerate}[k]
	\item បើឧស្ម័នត្រូវបានកម្តៅដោយរក្សាមាឌថេររហូតដល់សម្ពាធកើនឡើងបីដង ចូរគណនាសីតុណ្ហភាពនៃឧស្ម័ន។
	\item បើឧស្ម័នមានសម្ពាធ និងមាឌកើនឡើងពីរដង ចូរគណនាសីតុណ្ហភាពរបស់ឧស្ម័ន។
\end{enumerate}
\item នៅក្រោមផ្ទៃទឹកសមុទ្រជម្រៅ $25.0m$ មានម៉ាសមាឌ $\rho=1025kg/m^{3}$ មានសីតុណ្ហភាព $5^\circ C$។ ពពុះខ្យល់មួយមានមាឌ $1cm^{3}$ ផុសចេញមកលើផ្ទៃទឹកដែលមានសីតុណ្ហភាព $20^\circ C$។\\ គណនាមាឌរបស់ពពុះខ្យល់ពេលរៀបបែកចូលក្នុងខ្យល់។
\item គេដាក់ទឹក $9.0g$ ទៅក្នុងធុងដែលមានចំណុះ $2.0L$ រួចដុតកម្តៅដល់សីតុណ្ហភាព $500^\circ C$។ \\គណនាសម្ពាធក្នុងធុង។
\item សវនដ្ខានមួយមានវិមាត្រ $10.0m\times20.0m\times30.0m$។\\ គណនាចំនួនម៉ូលេគុលខ្យល់នៅក្នុងសវនដ្ខាននោះនៅកម្រិតសីតុណ្ហភាព $20.0^\circ C$ និងសម្ពាធ $101kPa$។
\item \begin{enumerate}[k]
	\item បង្ហាញឲ្យឃើញថា ម៉ាសមាឌឧស្ម័នបរិសុទ្ធដែលមានមាឌ $V$ មានទំនាក់ទំនង់ $\rho =\frac{PM}{RT}$ ដែល $P$ ជាសម្ពាធឧស្ម័ន $M$ ជាម៉ាសម៉ូលឧស្ម័ន $T$ ជាសីតុណ្ហភាពឧស្ម័ន និង $R$ ជាថេរសកលនៃឧស្ម័ន។
	\item គណនាម៉ាសមាឌនៃឧស្ម័នអុកសុីសែននៅសម្ពាធធម្មតា និងសីតុណ្ហភាព $20.0^\circ C$។
\end{enumerate}
\item មាសមានម៉ាសម៉ូល $197g/mol$។
\begin{enumerate}[k]
	\item គណនាចំនួនម៉ូលនៃអាតូមមាសក្នុងគម្រូមាសសុទ្ធ $2.50g$។
	\item គណនាចំនួនអាតូមដែលមានក្នុងគម្រូខាងលើ។
\end{enumerate}
\item គណនាៈ ចំនួនម៉ូល និងចំនួនម៉ូលេគុលក្នុង $1.00cm^{3}$ នៃឧស្ម័នបរិសុទ្ធនៅសម្ពាធ $100Pa$ និងសីតុណ្ហភាព $220K$។
\item គណនាតម្លៃមធ្យមនៃថាមពលសុីនេទិចរបស់ម៉ូលេគុលឧស្ម័នបរិសុទ្ធក្នុងករណីៈ
\begin{enumerate}[1]
	\item \begin{enumerate}[k]
		\item សីតុណ្ហភាព $0.00^\circ C$។
		\item សីតុណ្ហភាព $100^\circ C$។
	\end{enumerate}
	\item គណនាថាមពលសុីនេទិចសរុបក្នុងមួយម៉ូលនៃឧស្ម័នបរិសុទ្ធក្នុងករណីៈ
	\begin{enumerate}[k]
		\item សីតុណ្ហភាព $0.00^\circ C$។
		\item សីតុណ្ហភាព $100^\circ C$។
	\end{enumerate}
\end{enumerate}
\item គណនា​ប្ញសការេនៃការេល្បឿនមធ្យម $v_{rms}$ នៃអាតូមអេល្យូមនៅសីតុណ្ហភាព $1000K$។ \\គេឲ្យម៉ាសម៉ូលអេល្យូម $M=4.00g/mol$
\item គណនាថាមពលសុីនេទិចមធ្យមនៃម៉ូលេគុលនីត្រូសែននៅសីតុណ្ហភាព $1600K$។
\item ឧស្ម័នអុកសុីសែនមួយមានមាឌ $1000cm^{3}$ នៅសីតុណ្ហភាព $40^\circ C$ និងមានសម្ពាធ $1.01\times10^{5}Pa$ បានរីករហូតដល់មាឌរបស់វា $1500cm^{3}$ និងសម្ពាធរបស់វាគឺ $1.06\times10^{5}Pa$។
\begin{enumerate}[k]
	\item គណនាចំនួនម៉ូលនៃឧស្ម័នអុកសុីសែនខាងលើ។
	\item គណនាសីតុណ្ហភាពស្រេចនៃឧស្ម័នគម្រូខាងលើ។
\end{enumerate}
\item ក្នុងប្រព័ន្ធសុញ្ញាកាសខ្ពស់មួយ សម្ពាធដែលអាចវាស់បានស្មើនឹង $1.00\times10^{-10}torr$(ដែល $1torr=133Pa$)។ ឧបមាថា សីតុណ្ហភាពស្មើនឹង $300K$។ គេឲ្យ ថេប៊ុលស្មាន់ $k_{B}=1.38\times10^{-23}J/K$\\
ចូរគណនាចំនួនម៉ូលេគុលក្នុងមាឌមួយស្មើនឹង $1.00cm^{3}$។
\item បរិមាណនៃឧស្ម័នបរិសុទ្ធនៅសីតុណ្ហភាព $10.0^\circ C$ និងសម្ពាធ $100kPa$ ត្រូវបានគេបំពេញទៅក្នុងមាឌ $2.50m^{3}$។ គេឲ្យ ថេរសកលនៃឧស្ម័ន $R=8.31J/mol\cdot K$
\begin{enumerate}[k]
	\item គណនាចំនួនម៉ូលនៃឧស្ម័នដែលបានរៀបរាប់ខាងលើ។
	\item ប្រសិនបើសម្ពាធឡើងដល់ $300kPa$ និងសីតុណ្ហភាពឡើងដល់ $30.0^\circ C$។\\ គណនាមាឌដែលត្រូវយកឧស្ម័នទៅបំពេញ សន្មតថាគ្មានលិចឧស្ម័ន។
\end{enumerate}
\item ប្រសិនបើ ម៉ាស $m=2.1212g$ នៃឧស្ម័នបរិសុទ្ធមួយមានមាឌ $V=1.49L$ ស្ថិតក្នុងលក្ខខណ្ឌដែលមានសីតុណ្ហភាព $t=0^\circ C$ និងសម្ពាធ $P=810.6kPa$ តើវាជាឧស្ម័នអ្វី? គេឲ្យៈ $R=8.31J/mol\cdot K$។
\item បាឡុងរាងស៊វែរមួយមានមាឌ $4000cm^{3}$ ផ្ទុកដោយអេល្យូមនៅសម្ពាធ(ខាងក្នុង) $1.20\times10^{5}Pa$។ គណនាចំនួនម៉ូលនៃអេល្យូមក្នុងបាឡុង។ ប្រសិនបើថាមពលសុីនេទិចមធ្យមនៃអាតូមអេល្យូមនីមួយៗស្មើនឹង $3.60\times10^{-23}J$។ គេឲ្យៈ $R=8.31J/mol\cdot K$ និង $k_{B}=1.38\times10^{-23}J/K$។
\item \begin{enumerate}[k]
	\item តើអាតូមនៃឧស្ម័នអេល្យូមប៉ុន្មាន ដែលបំពេញក្នុងបាឡុងមួយដែលមានអង្គត់ផ្ចិត $30.0cm$ នៅសីតុណ្ហភាព $20.0^\circ C$ និងសម្ពាធ $1.00atm$។
	\item តើថាមពលសុីនេទិចមធ្យមនៃអាតូមអេល្យូមមួយស្មើប៉ុន្មាន?
	\item គណនាប្ញសការេនៃការេល្បឿនមធ្យមនៃអាតូមអេល្យូម។
	គេឲ្យៈ $R=8.31J/K\cdot mol,\\~k_{B}=1.38\times10^{-23}J/K,~1atm=10^{5}Pa$ និងម៉ាសម៉ូលអេល្យូម $M=4\times10^{-3}kg/mol$
\end{enumerate}
\item ធុងមួយមានមាឌ $20.0L$ ផ្ទុកឧស្ម័នអេល្យូម $0.225kg$ នៅសីតុណ្ហភាព $18.0^\circ C$។ ម៉ាសម៉ូលអេល្យូមគឺ $4.00g/mol$។ យក $R=8.31J/mol\cdot K$ និង $1atm=1.013\times10^{5}Pa$
\begin{enumerate}[k]
	\item គណនាចំនួនម៉ូលនៃឧស្ម័នអេល្យូមនៅក្នុងធុង។
	\item គណនាសម្ពាធនៅក្នុងធុងគិតជា $Pa$ និង $atm$។
\end{enumerate}
\item វិមាត្រនៃបន្ទប់មួយគឺ $4.20m\times3.00m\times2.50m$។
\begin{enumerate}[k]
	\item គណនាចំនួនម៉ូលេគុលខ្យល់ក្នុងបន្ទប់នៅសម្ពាធបរិយាកាស​ ($1atm$) ដែលមានសីតុណ្ហភាព $20^\circ C$។
	\item គណនាម៉ាសខ្យល់នេះ។ សន្មត់ថាខ្យល់ចាត់ចូលម៉ូលេគុលឌីអាតូម ដែលមានម៉ាសម៉ូល $28.9g/mol$។
	\item គណនាតម្លៃមធ្យមនៃថាមពលសុីនេទិចនៃម៉ូលេគុលនីមួយៗ។
	\item គណនាប្ញសការេនៃការេល្បឿនមធ្យមរបស់ម៉ូលេគុល។ 
\end{enumerate}
\item អ្នកផ្លុំបំប៉ោងបាឡុងរាងស៊្វែមួយដល់អង្កត់ផ្ចិត $50.0cm$ រហូតដល់សម្ពាធខាងក្នុងគឺ $1.25atm$ និងសីតុណ្ហភាពគឺ $22.0^\circ C$។ សន្មតថាឧស្ម័នទាំងអស់ជា $\ce{N2}$ មានម៉ាសម៉ូល $28.0g/mol$។
\begin{enumerate}[k]
	\item រកម៉ាសនៃម៉ូលេគុល $\ce{N2}$ មួយ។
	\item រកតម្លៃមធ្យមនៃថាមពលសុីនេទិចនៃម៉ូលេគុល $\ce{N2}$ មាន។
	\item តើមានចំនួនម៉ុលេគុល $\ce{N2}$ ក្នុងបាឡុងនេះប៉ុន្មាន?
	\item រកថាមពលសុីនេទិចសរុបនៃម៉ូលេគុល $\ce{N2}$ ទាំងអស់ក្នុងបាឡុង។
	គេឲ្យៈ $R=8.31J/mol\cdot K;\\~k_{B}=1.38\times10^{-23}J/K;~N_{A}=6.022\times10^{23}\text{ម៉ូលេគុល}/mol$ និង $1atm=10^{5}Pa$
\end{enumerate}              
\item \begin{enumerate}[k]
	\item គណនាមាឌ ដើម្បីរក្សា $4.0g$ នៃឧស្ម័នអុកសុីសែន$\left(M=32g/mol\right)$ នៅលក្ខខណ្ឌ{\en (S.T.P)}។
	\item គណនាថាមពលសុីនេទិចសរុបនៃម៉ូលេគុលឧស្ម័នអុកសុីសែនោះ។
	\item គណនាល្បឿ {\en rms} នៃអុកសុីសែននោះ។
\end{enumerate}
\item ឧស្ម័នមួយត្រូវបានផ្ទុកក្នុងធុង $8.00L$ បិទជិតមួយនៅសីតុណ្ហភាព $20.0^\circ C$ និងមានសម្ពាធ $9.00atm$។
\begin{enumerate}[k]
	\item គណនាចំនួនម៉ូលនៃម៉ូលេគុលឧស្ម័នក្នុងធុង។
	\item គណនាចំនួនម៉ូលេគុលដែលមានក្នុងធុង។
\end{enumerate}
\end{enumerate}