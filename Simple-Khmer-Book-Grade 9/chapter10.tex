\chapter{សមីការបន្ទាត់}


\section{ការសង់បន្ទាត់}
\subsection{ការសង់បន្ទាត់តាមតារាងតម្លៃលេខ}
\subsection{ការសង់បន្ទាត់កាត់តាមពីរចំណុច}

\section{សមីការនៃបន្ទាត់}
\subsection{សមីការនៃបន្ទាត់កាត់តាមពីរចំណុច}
\begin{general}
សមីការបន្ទាត់កាត់តាមពីរចំណុច $A(x_A,y_A)$ និង $B(x_B,y_B)$  មានរាង $\dfrac{y_B-y_A}{x_B-x_A}=\dfrac{y-y_A}{x-x_A}$។
\end{general}
\subsection{លក្ខខណ្ឌនៃបន្ទាត់ស្រប}

\begin{general}
គេមានសមីការបន្ទាត់ពីរគឺ $(D_1):y_1=a_1x+b_1$ និង  $(D_2):y_2=a_2x+b_2$ សមីការបន្ទាត់ទាំងពីរស្របគ្នាលុះត្រាតែ $a_1=a_2$ ។
\end{general}
\subsection{លក្ខខណ្ឌនៃបន្ទាត់កែង}

\begin{general}
គេមានសមីការបន្ទាត់ពីរគឺ $(D_1):y_1=a_1x+b_1$ និង  $(D_2):y_2=a_2x+b_2$ សមីការបន្ទាត់ទាំងពីរកែកគ្នាលុះត្រាតែ $a_1\times a_2=-1$ ។
\end{general}

%%%%%%%%%%%%%%%%%%%%%%%%%%%%%%%%%%%%%%%%
\newpage
\pros
\begin{enumerate}
%N1
\item ក្នុងចំណោមចំណុចខាងក្រោមចំណុចមួយណាស្ថិតនៅលើបន្ទាត់ដែលមានសមីការ $y=2x+1$៖
\begin{enumerate}[label=\alph*.]
\begin{multicols}{4}
\item $A(0,2)$
\item $A(-1,1)$
\item $A(0,1)$
\item $A(2,-1)$
\end{multicols}
\end{enumerate}

%N2
\item ក្នុងចំណោមចំណុចខាងក្រោមចំណុចមួយណាស្ថិតនៅលើបន្ទាត់ដែលមានសមីការ $y=x+2$៖
\begin{enumerate}[label=\alph*.]
\begin{multicols}{4}
\item $A(0,2)$
\item $A(-1,1)$
\item $A(0,1)$
\item $A(2,-1)$
\end{multicols}
\end{enumerate}

%N3
\item ក្នុងចំណោមចំណុចខាងក្រោមចំណុចមួយណាស្ថិតនៅលើបន្ទាត់ដែលមានសមីការ $y=3x$៖
\begin{enumerate}[label=\alph*.]
\begin{multicols}{4}
\item $A(0,2)$
\item $A(-1,1)$
\item $A(0,0)$
\item $A(2,-1)$
\end{multicols}
\end{enumerate}

%N4
\item ក្នុងចំណោមចំណុចខាងក្រោមចំណុចមួយណាស្ថិតនៅលើបន្ទាត់ដែលមានសមីការ $y=-3x+2$៖
\begin{enumerate}[label=\alph*.]
\begin{multicols}{4}
\item $A(0,2)$
\item $A(-1,1)$
\item $A(1,-1)$
\item $A(2,-1)$
\end{multicols}
\end{enumerate}

%N5
\item ក្នុងចំណោមចំណុចខាងក្រោមចំណុចមួយណាស្ថិតនៅលើបន្ទាត់ដែលមានសមីការ $y=x-2$៖
\begin{enumerate}[label=\alph*.]
\begin{multicols}{4}
\item $A(0,2)$
\item $A(-1,1)$
\item $A(1,-1)$
\item $A(2,-1)$
\end{multicols}
\end{enumerate}

%N6
\item គេមានសមីការបន្ទាត់ $y=x-2$ ជាសមីការបន្ទាត់ដែលកាត់តាមចំណុច៖
\begin{enumerate}[label=\alph*.]
\begin{multicols}{4}
\item $A(0,2)$
\item $A(-1,1)$
\item $A(1,-1)$
\item $A(2,-1)$
\end{multicols}
\end{enumerate}

%N7
\item គេមានសមីការបន្ទាត់ $y=x+2$ ជាសមីការបន្ទាត់ដែលកាត់តាមចំណុច៖
\begin{enumerate}[label=\alph*.]
\begin{multicols}{4}
\item $A(0,2)$
\item $A(-1,1)$
\item $A(1,3)$
\item $A(2,-1)$
\end{multicols}
\end{enumerate}

%N8
\item គេមានសមីការបន្ទាត់ $y=3x+7$ ជាសមីការបន្ទាត់ដែលកាត់តាមចំណុច៖
\begin{enumerate}[label=\alph*.]
\begin{multicols}{4}
\item $A(0,2)$
\item $A(-1,1)$
\item $A(1,3)$
\item $A(0,7)$
\end{multicols}
\end{enumerate}

%N9
\item គេមានសមីការបន្ទាត់ $y=\dfrac{x}{2}+1$ ជាសមីការបន្ទាត់ដែលកាត់តាមចំណុច៖
\begin{enumerate}[label=\alph*.]
\begin{multicols}{4}
\item $A(0,2)$
\item $A(-1,1)$
\item $A(-2,0)$
\item $A(0,7)$
\end{multicols}
\end{enumerate}

%N10
\item គេមានសមីការបន្ទាត់ $y=-\dfrac{x}{2}+3$ ជាសមីការបន្ទាត់ដែលកាត់តាមចំណុច៖
\begin{enumerate}[label=\alph*.]
\begin{multicols}{4}
\item $A(0,2)$
\item $A(-1,1)$
\item $A(-2,2)$
\item $A(0,7)$
\end{multicols}
\end{enumerate}

%N11
\item សមីការបន្ទាត់ដែលកាត់តាមចំណុច $A(0,1)$ ហើយស្របទៅនឹងបន្ទាត់ដែលមានសមីការ $y=2x+5$ គឺ៖
\begin{enumerate}[label=\alph*.]
\begin{multicols}{4}
\item $y=2x+3$
\item $y=2x-3$
\item $y=2x+1$
\item $y=-2x+1$
\end{multicols}
\end{enumerate}

%N12
\item សមីការបន្ទាត់ដែលកាត់តាមចំណុច $A(0,3)$ ហើយស្របទៅនឹងបន្ទាត់ដែលមានសមីការ $y=2x+5$ គឺ៖
\begin{enumerate}[label=\alph*.]
\begin{multicols}{4}
\item $y=2x+3$
\item $y=2x-3$
\item $y=2x+1$
\item $y=-2x+1$
\end{multicols}
\end{enumerate}

%N13
\item សមីការបន្ទាត់ដែលកាត់តាមចំណុច $A(0,-3)$ ហើយស្របទៅនឹងបន្ទាត់ដែលមានសមីការ $y=2x+5$ គឺ៖
\begin{enumerate}[label=\alph*.]
\begin{multicols}{4}
\item $y=2x+3$
\item $y=2x-3$
\item $y=2x+1$
\item $y=-2x+1$
\end{multicols}
\end{enumerate}

%N14
\item សមីការបន្ទាត់ដែលកាត់តាមចំណុច $A(0,0)$ ហើយស្របទៅនឹងបន្ទាត់ដែលមានសមីការ $y=2x+5$ គឺ៖
\begin{enumerate}[label=\alph*.]
\begin{multicols}{4}
\item $y=2x+3$
\item $y=2x-3$
\item $y=2x$
\item $y=-2x+1$
\end{multicols}
\end{enumerate}

%N15
\item សមីការបន្ទាត់ដែលកាត់តាមចំណុច $A(1,5)$ ហើយស្របទៅនឹងបន្ទាត់ដែលមានសមីការ $y=2x+5$ គឺ៖
\begin{enumerate}[label=\alph*.]
\begin{multicols}{4}
\item $y=2x+3$
\item $y=2x-3$
\item $y=2x+3$
\item $y=-2x+1$
\end{multicols}
\end{enumerate}

%N16
\item សមីការបន្ទាត់កែងទៅនឹងបន្ទាត់ដែលមានសមីការ $y=2x+5$ គឺ៖
\begin{enumerate}[label=\alph*.]
\begin{multicols}{4}
\item $y=\dfrac{x}{2}+3$
\item  $y=-\dfrac{x}{2}+3$
\item  $y=\dfrac{x}{2}-3$
\item $y=\dfrac{x}{2}+7$
\end{multicols}
\end{enumerate}

%N17
\item សមីការបន្ទាត់កែងទៅនឹងបន្ទាត់ដែលមានសមីការ $y=-2x+5$ គឺ៖
\begin{enumerate}[label=\alph*.]
\begin{multicols}{4}
\item $y=\dfrac{x}{2}+3$
\item  $y=-\dfrac{x}{2}+3$
\item  $y=\dfrac{x}{2}-3$
\item $y=\dfrac{x}{2}+7$
\end{multicols}
\end{enumerate}

%N18
\item សមីការបន្ទាត់កែងទៅនឹងបន្ទាត់ដែលមានសមីការ $y=-\dfrac{x}{2}+5$ គឺ៖
\begin{enumerate}[label=\alph*.]
\begin{multicols}{4}
\item $y=-2x+3$
\item  $y=2x+3$
\item  $y=\dfrac{x}{2}-3$
\item $y=\dfrac{x}{2}+7$
\end{multicols}
\end{enumerate}

%N19
\item សមីការបន្ទាត់កែងទៅនឹងបន្ទាត់ដែលមានសមីការ $y=-\dfrac{x}{3}+5$ គឺ៖
\begin{enumerate}[label=\alph*.]
\begin{multicols}{4}
\item $y=-3x+3$
\item  $y=3x+3$
\item  $y=\dfrac{x}{3}-3$
\item $y=\dfrac{x}{3}+7$
\end{multicols}
\end{enumerate}

%N20
\item សមីការបន្ទាត់កែងទៅនឹងបន្ទាត់ដែលមានសមីការ $2y-3x+10=0$ គឺ៖
\begin{enumerate}[label=\alph*.]
\begin{multicols}{4}
\item $y=-3x+3$
\item  $y=3x+3$
\item  $y=-\dfrac{2x}{3}-3$
\item $y=-\dfrac{3x}{2}+7$
\end{multicols}
\end{enumerate}

%N21
\item សមីការបន្ទាត់ដែលកាត់តាមចំណុច $A(0,0)$ ហើយកែងទៅនឹងបន្ទាត់ដែលមានសមីការ $y=2x+5$ គឺ៖
\begin{enumerate}[label=\alph*.]
\begin{multicols}{4}
\item $y=\dfrac{x}{2}$
\item $y=-\dfrac{x}{2}+3$
\item $y=-\dfrac{x}{2}$
\item $y=\dfrac{x}{2}+3$
\end{multicols}
\end{enumerate}

%N22
\item សមីការបន្ទាត់ដែលកាត់តាមចំណុច $A(3,2)$ ហើយកែងទៅនឹងបន្ទាត់ដែលមានសមីការ $y=3x+5$ គឺ៖
\begin{enumerate}[label=\alph*.]
\begin{multicols}{4}
\item $y=\dfrac{x}{2}$
\item $y=-\dfrac{x}{3}+3$
\item $y=-\dfrac{x}{3}$
\item $y=\dfrac{x}{3}+3$
\end{multicols}
\end{enumerate}

%N23
\item សមីការបន្ទាត់ដែលកាត់តាមចំណុច $A(2,0)$ ហើយកែងទៅនឹងបន្ទាត់ដែលមានសមីការ $y=\dfrac{2x}{3}+5$ គឺ៖
\begin{enumerate}[label=\alph*.]
\begin{multicols}{4}
\item $y=-\dfrac{3x}{2}$
\item $y=-\dfrac{3x}{2}+3$
\item $y=-\dfrac{x}{3}$
\item $y=\dfrac{x}{3}+3$
\end{multicols}
\end{enumerate}

%N24
\item សមីការបន្ទាត់ដែលកាត់តាមចំណុច $A(0,1)$ ហើយកែងទៅនឹងបន្ទាត់ដែលមានសមីការ $3y+x+6=0$ គឺ៖
\begin{enumerate}[label=\alph*.]
\begin{multicols}{4}
\item $y=-3x+1$
\item $y=3x+5$
\item $y=-\dfrac{x}{3}$
\item $y=\dfrac{x}{3}+3$
\end{multicols}
\end{enumerate}

%N25
\item សមីការបន្ទាត់ដែលកាត់តាមចំណុច $A(1,8)$ ហើយកែងទៅនឹងបន្ទាត់ដែលមានសមីការ $3y-x+6=0$ គឺ៖
\begin{enumerate}[label=\alph*.]
\begin{multicols}{4}
\item $y=-3x+1$
\item $y=3x+5$
\item $y=-\dfrac{x}{3}$
\item $y=\dfrac{x}{3}+3$
\end{multicols}
\end{enumerate}

%N26
\item សមីការបន្ទាត់ដែលកាត់តាមចំណុច $A(0,1)$ នឹង មានមេគុណប្រាប់ទិស $a=2$គឺ៖
\begin{enumerate}[label=\alph*.]
\begin{multicols}{4}
\item $y=2x-1$
\item $y=2x$
\item $y=2x+1$
\item $y=2x-5$
\end{multicols}
\end{enumerate}

%N27
\item សមីការបន្ទាត់ដែលកាត់តាមចំណុច $A(1,-1)$នឹង មានមេគុណប្រាប់ទិស $a=-2$គឺ៖
\begin{enumerate}[label=\alph*.]
\begin{multicols}{4}
\item $y=-2x-1$
\item $y=-2x$
\item $y=-2x+1$
\item $y=-2x-5$
\end{multicols}
\end{enumerate}

%N28
\item សមីការបន្ទាត់ដែលកាត់តាមចំណុច $A(-1,-3)$ នឹង មានមេគុណប្រាប់ទិស $a=-2$គឺ៖
\begin{enumerate}[label=\alph*.]
\begin{multicols}{4}
\item $y=-2x-1$
\item $y=-2x$
\item $y=-2x+1$
\item $y=-2x-5$
\end{multicols}
\end{enumerate}

%N29
\item សមីការបន្ទាត់ដែលកាត់តាមចំណុច $A(1,-3) $នឹង មានមេគុណប្រាប់ទិស $a=-2$គឺ៖
\begin{enumerate}[label=\alph*.]
\begin{multicols}{4}
\item $y=-2x-1$
\item $y=-2x$
\item $y=-2x+1$
\item $y=-2x-5$
\end{multicols}
\end{enumerate}
%N30
\item សមីការបន្ទាត់ដែលកាត់តាមចំណុច $A(0,0)$ នឹង មានមេគុណប្រាប់ទិស $a=-2$គឺ៖
\begin{enumerate}[label=\alph*.]
\begin{multicols}{4}
\item $y=-2x-1$
\item $y=-2x$
\item $y=-2x+1$
\item $y=-2x-5$
\end{multicols}
\end{enumerate}

\end{enumerate}

%%%%%%%%%%%%%%%%%%%%%%%%%%%%%%%%%%%%%%%
\newpage
\problem
\begin{enumerate}
%N1
\item ចូរសង់បន្ទាត់ដែលកាត់តាមមួយចំណុច និង មានមេគុណប្រាប់ទិស៖
\begin{enumerate}[label=\alph*.]
\begin{multicols}{2}
\item $a=2, \quad (-4,1)$
\item $a=-\dfrac{2}{3},\quad (1,5)$
\end{multicols}
\end{enumerate}
%N2
\item ចូររកមេគុណប្រាប់ទិសនៃបន្ទាត់ដែលកាត់តាមពីរចំណុច៖
\begin{enumerate}[label=\alph*.]
\begin{multicols}{2}
\item $A(3,1)\quad ,B(6,-2)$
\item $A(3,-4)\quad ,B(-9,2)$
\item $A(-2,-1)\quad ,B(-4,-4)$
\item $A(1,4)\quad ,B(4,-2)$
\end{multicols}
\end{enumerate}

%N3
\item  ចូររកសមីការបន្ទាត់ដែលកាត់តាមពីរចំណុច៖
\begin{enumerate}[label=\alph*.]
\begin{multicols}{2}
\item $A(-6,1)\quad ,B(6,-2)$
\item $A(3,-4)\quad ,B(2,6)$
\item $A(2,0)\quad ,B(1,-8)$
\item $A(-4,-4)\quad ,B(-2,6)$
\item $A(2,-1)\quad ,B(-2,1)$
\item $A(0,5)\quad ,B(3,-2)$
\end{multicols}
\end{enumerate}

%N4
\item  ក្នុងចំណោមបន្ទាត់ទាំងបីខាងក្រោម តើបន្ទាត់មួយណាស្របគ្នា៖
\begin{enumerate}[label=\alph*.]
\begin{multicols}{3}
\item $2x+3y=-11$
\item $4x+8y-1=0$
\item $y=-\dfrac{2x}{3}+1$
\end{multicols}
\end{enumerate}

%N5
\item  ចូររកមេគុណប្រាប់ទិសនៃបន្ទាត់តាងសមីការខាងក្រោម រួចប្រាប់ថាតើគូសមីការណាខ្លះដែលមានបន្ទាត់ស្របគ្នា៖
\begin{enumerate}[label=\alph*.]
\begin{multicols}{2}
\item $y=x+4\quad ,x-y+5=0$
\item $y=2x-3\quad ,x+2y+1=0$
\item $3x-y+4=0\quad ,y-2=3(x+1)$
\item $2y=5x+6\quad ,5x+2y-1=0$
\end{multicols}
\end{enumerate}
%N6
\item តើគូសមីការណាខ្លះដែលមានបន្ទាត់កែងគ្នា?
\begin{enumerate}[label=\alph*.]
\item $y=3x+1,\quad y=-\dfrac{x}{3}$
\item $y=2x+5,\quad x-2y+6=0$
\item $y=5x-4,\quad x+5y-1=0$
\end{enumerate}

%N7
\item ចូរកំណត់សមីការបន្ទាត់ដែលស្របនឹង
\begin{enumerate}[label=\alph*.]
\item $y=3x-4$ ហើយកាត់តាមចំណុច $A(5,1)$
\item $3x-2y+5=0$ ហើយកាត់តាមចំណុច $A(-2,4)$
\item $2x+2y+9=0$ ហើយកាត់តាមចំណុច $A(2,5)$
\item $x-5y+6=0$ ហើយកាត់តាមចំណុច $A(0,0)$
\end{enumerate}

%N8
\item ចូរកំណត់សមីការបន្ទាត់ដែលកែងនឹង
\begin{enumerate}[label=\alph*.]
\item $y=\dfrac{x}{2}+4$ ហើយកាត់តាមចំណុច $A(5,0)$
\item $x-y+5=0$ ហើយកាត់តាមចំណុច $A(0,0)$
\item $8x+3y+1=0$ ហើយកាត់តាមចំណុច $A(-1,4)$
\item $y=-x+6$ ហើយកាត់តាមចំណុច $A(-4,-\dfrac{2}{3})$
\end{enumerate}

%N9
\item  
\begin{enumerate}[label=\alph*.]
\item ចូរកំណត់សមីការនៃបន្ទាត់ $(d_1)$ កាត់តាមចំណុច $A(-3,5)$ ដែលមានមេគុណប្រាប់ទិសស្មើនឹង $-2$ រួចសង់បន្ទាត់នេះ។
\item គេមានសមីការបន្ទាត់ $(d_2): y=(m-1)x+2$។ ចូរកំណត់តម្លៃ $m$ ដើម្បីអោយបន្ទាត់ $(d_2)$ ស្រប $(d_1)$ ។ តើមានតម្លៃ $m$ ដែលឲ្យ $(d_2)$ កែងនឹង $(d_1)$ ឬទេ? ចូរសរសេរសមីការនៃបន្ទាត់ត្រូវនឹងតម្លៃ $m$ នីមួយៗ រួចសង់បន្ទាត់ទាំងនេះ។
\item  $B$ ជាចំណុចដែលមានកូអរដោនេ $(6,0)$។ ចូរសរសេរសមីការនៃបន្ទាត់កាត់តាម $A$ និង $B$ រួចបញ្ជាក់ចំណុចជួបនឹងអ័ក្ស $y'y$។
\end{enumerate}

%N10
\item  ក្នុងតម្រុយអរតូណរមេ គេឲ្យបីចំណុច $A(0,6), B(-3,0)$ និង $C(6,0)$។
\begin{enumerate}[label=\alph*.]
\item  ចូររកសមីការបន្ទាត់ $AB$ និង $AC$។
\item ចូររកសមីការនៃបន្ទាត់ $(d)$ កាត់តាមចំណុច $C$ ហើយកែងនឹងបន្ទាត់ $AB$។
\item  ចូររកសមីការនៃបន្ទាត់ $(d')$ កាត់តាមចំណុច $B$ ហើយស្របនឹងបន្ទាត់ $AC$។
\end{enumerate}

%N11
\item  
\begin{enumerate}[label=\alph*.]
\item  ក្នុងតម្រុយអរតូណរមេ រកសមីការនៃបន្ទាត់កាត់តាមចំណុច $A(6,2)$ និង $B(2,-2)$។ ចូររកសមីការនៃខ្សែមេដ្យាទ័ររបស់ $AB$។
\item គេឲ្យចំណុច $C(-2,6)$ ។ចូរបង្ហាញថា $CAB$ ជាត្រីកោណសមបាត។
\item  ចូររកសមីការបន្ទាត់$(d)$ កាត់តាម $C$ ហើយស្របនឹង $AB$។
\item  ចូរបង្ហាញថាបន្ទាត់ $(d)$ កាត់តាម  $M(-5,3)$។
\end{enumerate}

%N12
\item  ក្នុងតម្រុយអរតូណរមេ គេឲ្យចំណុច $A(-6,0), B(6,0)$ និង  $C(3,0)$។
\begin{enumerate}[label=\alph*.]
\item ចូររកសមីការនៃជ្រុងត្រីកោណ $ABC$។
\item  ចូរសរសេរសមីការនៃខ្សែមេដ្យាទ័ររបស់ជ្រុងនៃត្រីកោណ $ABC$។
\end{enumerate}

%N13 
\item បើចំណុច $(a,6)$ និង $(2,a)$ ឋិតនៅលើបន្ទាត់ $4x+2y=b$។ ចូរកំណត់តម្លៃ $a$ និង $b$ រួចសង់បន្ទាត់នេះ។

\end{enumerate}


%%%%%%%%%%%%%%%%%%%%%%%%%%%%%%%%%%%%%%
\newpage
\proa
	


