\chapter{ចំនួនអសនិទាន}
\section{ឬសការេ}
\begin{generality}
បើ $a>0,\quad x^2=a$ គេបាន $x=\sqrt{a}$ និង  $x=-\sqrt{a}$។
\end{generality}

\section{ឬសគួប}
\begin{generality}
បើ $ x^3=a$ គេបាន $x=\sqrt[3]{a},\quad a$ អាចវិជ្ជមាន ឬ អវិជ្ជមាន ។
\end{generality}

\section{ប្រមាណវិធីលើរ៉ាឌីកាល់}
\subsection{វិធីគុណ}
\begin{generality}
ផលគុណនៃរ៉ាឌីកាល់ដែលមានសន្ទស្សន៍ដូចគ្នាស្មើនឹងរ៉ាឌីកាល់នៃផលគុណ
\begin{enumerate}[label=\alph*.]
	\item $\sqrt{a}\cdot \sqrt{b}=\sqrt{ab},\quad a\ge 0,b\ge 0$
	\item $\sqrt[3]{a}\cdot \sqrt[3]{b}=\sqrt[3]{ab}\quad a,b$ អាចជាចំនួនវិជ្ជមានឬអវិជ្ជមាន។
\end{enumerate}
\end{generality}
\subsection{វិធីចែក}
\begin{generality}
ផលចែកនៃរ៉ាឌីកាល់ដែលមានសន្ទស្សន៍ដូចគ្នាស្មើនឹងរ៉ាឌីកាល់នៃផលចែក
\begin{enumerate}[label=\alph*.]
	\item $\dfrac{\sqrt{a}}{\sqrt{b}}=\sqrt{\dfrac{a}{b}},\quad a\ge 0,b> 0$
	\item $\dfrac{\sqrt[3]{a}}{\sqrt[3]{b}}=\sqrt[3]{\dfrac{a}{b}},\quad ,b\ne 0$
\end{enumerate}
\end{generality}
\subsection{ការបញ្ចេញមួយចំនួនពីក្នុងរ៉ាឌីកាល់}
\begin{generality}
ដើម្បីបញ្ចេញមួយចំនួនពីរ៉ាឌីកាល់ គេត្រូវបំប្លែងរ៉ាឌីកង់ជាស្វ័យគុណដែលមាននិទស្សន្តស្មើនឹងសន្ទស្សន៍នៃរ៉ាឌីកាល់។
\end{generality}

\subsection{កាបញ្ចូលមួយចំនួនទៅក្នុងរ៉ាឌីកាល់}
\begin{generality}
ដើម្បីបញ្ចេញមួយចំនួនទៅក្នុងរ៉ាឌីកាល់ គេត្រូវលើកចំនួននោះជាស្វ័យគុណដោយឲ្យនិទស្សន្តនៃស្វ័យគុណស្មើនឹងសន្ទស្សន៍នៃរ៉ាឌីកាល់។
\end{generality}
\subsection{វិធីបូក និង វិធីដក}

\begin{generality}
ដើម្បីគណនាផលបូក និង ផលដករ៉ាឌីកាល់ដែលមានសន្ទស្សន៍ដូចគ្នាគេត្រូវបំប្លែងរ៉ាឌីកង់ឲ្យដូចគ្នា។ 
ដើម្បីគណនាផលបូក និង ផលដករ៉ាឌីកាល់ដែលមានសន្ទស្សន៍ដូចគ្នា និង មានសន្ទស្សន៍ដូចគ្នា គេបូក-ដកមេគុណនឹងមេគុណ ហើយរ៉ាឌីកាល់ទុកដដែល។
\end{generality}
%%%%%%%%%%%%%%%%%%%%%%%%%%%%%%%%%%%%%%%%%%%%%%
\newpage
\section{ផ្នែកលំហាត់មានដំណោះស្រាយ}
\begin{center}
ចូរជ្រើសរើសចម្លើយដែលមានតែមួយគត់ នៅក្នុងសំណួរនីមួយៗខាងក្រោម៖
\end{center}
\begin{enumerate}
%P1
\item តម្លៃ $x$ ដែលផ្ទៀងផ្ទាត់សមីការ $2x+3=5$ គឺ
\begin{enumerate}[k,4]
	\item $x=1$
	\item $x=-1$
	\item $x=2$
	\item $x=-2$
\end{enumerate}

%P2
\item តម្លៃ $x$ ដែលផ្ទៀងផ្ទាត់សមីការ $-2x+3=5$ គឺ
\begin{enumerate}[k,4]
	\item $x=1$
	\item $x=-1$
	\item $x=2$
	\item $x=-2$
\end{enumerate}

%P3
\item តម្លៃ $x$ ដែលផ្ទៀងផ្ទាត់សមីការ $(x+1)(x+2)=x^2+5$ គឺ
\begin{enumerate}[k,4]
	\item $x=1$
	\item $x=-1$
	\item $x=2$
	\item $x=-2$
\end{enumerate}

%P4
\item កន្សោម$A=x^2-4$ អាចសរសេរជាទម្រង់៖
\begin{enumerate}[k,4]
	\item $A=(x+2)^2$
	\item $A=(x-2)^2$
	\item $A=(x+2)(x+2)$
	\item $A=(x-2)(x+2)$
\end{enumerate}

%P5
\item សមីការ $x^2-4=0$ មានចម្លើយវិជ្ជមានមួយគត់គឺ៖
\begin{enumerate}[k,4]
	\item $x=-4$
	\item $x=4$
	\item $x=2$
	\item $x=-2$
\end{enumerate}

%P6
\item តម្លៃ $x$ ដែលផ្ទៀងផ្ទាត់សមភាព $x^2=25$ គឺ៖
\begin{enumerate}[k,4]
	\item $x=-\sqrt{5}$
	\item $x=\sqrt{5}$
	\item $x=5$
	\item $x=-5$
\end{enumerate}

%P7
\item តម្លៃ $x$ ដែលផ្ទៀងផ្ទាត់សមភាព $x^2=49$ គឺ៖
\begin{enumerate}[k,4]
	\item $x=-\sqrt{7}$
	\item $x=\sqrt{7}$
	\item $x=7$
	\item $x=-7$
\end{enumerate}

%P8
\item តម្លៃនៃកន្សោម $A=\sqrt{64}$  គឺ៖
\begin{enumerate}[k,4]
	\item $x=-8,8$
	\item $x=-8$
	\item $x=8$
	\item $x=4$
\end{enumerate}

%P9
\item តម្លៃនៃកន្សោម $A=-\sqrt{64}$  គឺ៖
\begin{enumerate}[k,4]
	\item $x=-8,8$
	\item $x=-8$
	\item $x=8$
	\item $x=4$
\end{enumerate}

%P10
\item តម្លៃនៃកន្សោម $A=\sqrt{\left(1-\sqrt{3}\right)^2}$  គឺ៖
\begin{enumerate}[k,4]
	\item $x=1-\sqrt{3}$
	\item $x=-1-\sqrt{3}$
	\item $x=1+\sqrt{3}$
	\item $x=-1+\sqrt{3}$
\end{enumerate}

%P11
\item សមីការ $x^2=0$ មានឬសគឺ៖
\begin{enumerate}[k,4]
	\item $x=2$
	\item $x=-2$
	\item $x=0$
	\item $x=-1$
\end{enumerate}

%P12
\item សមីការ $x^2=-4$ មានឬសគឺ៖
\begin{enumerate}[k,4]
	\item $x=2$
	\item $x=-2$
	\item សមីការមានចម្លើយមួយ
	\item សមីការគ្មានចម្លើយ
\end{enumerate}

%P13
\item សមីការ $x^2=\dfrac{25}{4}$ មានឬសអវិជ្ជមានមួយគឺ៖
\begin{enumerate}[k,4]
	\item $x=-2$
	\item $x=-5$
	\item $x=-\dfrac{5}{2}$
	\item $x=\pm \dfrac{5}{2}$
\end{enumerate}

%P14
\item សមីការ $x^2-1=80$ មានឬសគឺ៖
\begin{multicols}{4}
\begin{enumerate}[label=\alph*.]
	\item $x=-9$
	\item $x=-8$
	\item $x=\pm 9$
	\item $x=\pm 8$
\end{enumerate}
\end{multicols}

%P15
\item សមីការ $x^2=0.01$ មានឬសគឺ៖
\begin{multicols}{4}
\begin{enumerate}[label=\alph*.]
	\item $x=0.001$
	\item $x=-0.001$
	\item $x=\pm 0.1$
	\item $x=\pm 0.2$
\end{enumerate}
\end{multicols}

%P16
\item សមីការ $x^2=\dfrac{1}{36}$ មានឬសគឺ៖
\begin{multicols}{4}
\begin{enumerate}[label=\alph*.]
	\item $x=-6$
	\item $x=6$
	\item $x=\dfrac{1}{6}$
	\item $x=\pm \dfrac{1}{6}$
\end{enumerate}
\end{multicols}

%P17
\item សមីការ $x^2=3$ មានឬសគឺ៖
\begin{multicols}{4}
\begin{enumerate}[label=\alph*.]
	\item $x=-3$
	\item $x=3$
	\item $x=\sqrt{3}$
	\item $x=\pm \sqrt{3}$
\end{enumerate}
\end{multicols}

%P18
\item សមីការ $x^2=625$ មានឬសគឺ៖
\begin{multicols}{4}
\begin{enumerate}[label=\alph*.]
	\item $x=-15$
	\item $x=\pm 15$
	\item $x=-25$
	\item $x=\pm 25$
\end{enumerate}
\end{multicols}

%P19
\item សមីការ $x^2=-1225$ មានឬសគឺ៖
\begin{multicols}{4}
\begin{enumerate}[label=\alph*.]
	\item $x=-35$
	\item $x=35$
	\item $x=\pm \sqrt{-1225}$
	\item គ្មានចម្លើយ
\end{enumerate}
\end{multicols}

%P20
\item សមីការ $x^2=-225$ មានឬសគឺ៖
\begin{multicols}{4}
\begin{enumerate}[label=\alph*.]
	\item $x=-15$
	\item $x=15$
	\item $x=\pm \sqrt{-1225}$
	\item គ្មានចម្លើយ
\end{enumerate}
\end{multicols}

%P21
\item ចូរសរសេរតម្លៃនៃ$8$ ជាទម្រង់ផលគុណកត្តាបឋម៖
\begin{multicols}{4}
\begin{enumerate}[label=\alph*.]
	\item $2\times 4$
	\item $1\times 8$
	\item $2\times 2\times 3$
	\item $2\times 2\times 2$
\end{enumerate}
\end{multicols}

%P22
\item ចូរសរសេរតម្លៃនៃ$210$ ជាទម្រង់ផលគុណកត្តាបឋម៖
\begin{multicols}{4}
\begin{enumerate}[label=\alph*.]
	\item $21\times 10$
	\item $42 \times 5$
	\item $21\times 2\times 5$
	\item $2\times 3\times 5\times 7$
\end{enumerate}
\end{multicols}

%P23
\item ចូរសរសេរតម្លៃនៃ$32$ ជាទម្រង់ផលគុណកត្តាបឋម៖
\begin{multicols}{4}
\begin{enumerate}[label=\alph*.]
	\item $16\times 2$
	\item $8 \times 4$
	\item $2^5+1$
	\item $2^5$
\end{enumerate}
\end{multicols}
%P24
\item ចូរសរសេរតម្លៃនៃ$250$ ជាទម្រង់ផលគុណកត្តាបឋម៖
\begin{multicols}{4}
\begin{enumerate}[label=\alph*.]
	\item $25\times 10$
	\item $125\times 2$
	\item $50\times 5$
	\item $2\times 5^3$
\end{enumerate}
\end{multicols}
%P25
\item ចូរសរសេរតម្លៃនៃ$54$ ជាទម្រង់ផលគុណកត្តាបឋម៖
\begin{multicols}{4}
\begin{enumerate}[label=\alph*.]
	\item $27\times 2$
	\item $9\times 6$
	\item $18\times 3$
	\item $2\times 3^3$
\end{enumerate}
\end{multicols}

%P26
\item តម្លៃ $x$ ដែលផ្ទៀងផ្ទាត់សមីការ $x^3=8$គឺ៖
\begin{multicols}{4}
\begin{enumerate}[label=\alph*.]
	\item $x=-2$
	\item $x=2$
	\item $x=\pm 2$
	\item គ្មានតម្លៃ$x$
\end{enumerate}
\end{multicols}

%P27
\item តម្លៃ $x$ ដែលផ្ទៀងផ្ទាត់សមីការ $x^3=27$គឺ៖
\begin{multicols}{4}
\begin{enumerate}[label=\alph*.]
	\item $x=-3$
	\item $x=3$
	\item $x=\pm 3$
	\item $x=0$
\end{enumerate}
\end{multicols}

%P28
\item តម្លៃ $x$ ដែលផ្ទៀងផ្ទាត់សមីការ $x^3=-27$គឺ៖
\begin{multicols}{4}
\begin{enumerate}[label=\alph*.]
	\item $x=-3$
	\item $x=3$
	\item $x=\pm 3$
	\item $x=0$
\end{enumerate}
\end{multicols}

%P29
\item តម្លៃ $x$ ដែលផ្ទៀងផ្ទាត់សមីការ $x^3=64$គឺ៖
\begin{multicols}{4}
\begin{enumerate}[label=\alph*.]
	\item $x=4$
	\item $x=-4$
	\item $x=\pm 4$
	\item $x=\pm 8$
\end{enumerate}
\end{multicols}

%P30
\item តម្លៃ $x$ ដែលផ្ទៀងផ្ទាត់សមីការ $x^3=125$គឺ៖
\begin{multicols}{4}
\begin{enumerate}[label=\alph*.]
	\item $x=15$
	\item $x=-15$
	\item $x=\pm 5$
	\item $x=5$
\end{enumerate}
\end{multicols}

%P31
\item តម្លៃ $x$ ដែលផ្ទៀងផ្ទាត់សមីការ $x^3+125=0$គឺ៖
\begin{multicols}{4}
\begin{enumerate}[label=\alph*.]
	\item $x=15$
	\item $x=-15$
	\item $x=-5$
	\item $x=5$
\end{enumerate}
\end{multicols}

%P32
\item តម្លៃ $x$ ដែលផ្ទៀងផ្ទាត់សមីការ $x^3-1=215$គឺ៖
\begin{multicols}{4}
\begin{enumerate}[label=\alph*.]
	\item $x=16$
	\item $x=-16$
	\item $x=-6$
	\item $x=6$
\end{enumerate}
\end{multicols}

%P33
\item តម្លៃ $x$ ដែលផ្ទៀងផ្ទាត់សមីការ $x^3=-216$គឺ៖
\begin{multicols}{4}
\begin{enumerate}[label=\alph*.]
	\item $x=16$
	\item $x=-16$
	\item $x=-6$
	\item $x=6$
\end{enumerate}
\end{multicols}
%P34
\item តម្លៃ $x$ ដែលផ្ទៀងផ្ទាត់សមីការ $x^3=\dfrac{1}{8}$គឺ៖
\begin{multicols}{4}
\begin{enumerate}[label=\alph*.]
	\item $x=\dfrac{1}{2}$
	\item $x=-\dfrac{1}{2}$
	\item $x=-2$
	\item $x=2$
\end{enumerate}
\end{multicols}

%P35
\item តម្លៃ $x$ ដែលផ្ទៀងផ្ទាត់សមីការ $x^3=0.008$គឺ៖
\begin{multicols}{4}
\begin{enumerate}[label=\alph*.]
	\item $x=\dfrac{1}{2}$
	\item $x=-\dfrac{1}{2}$
	\item $x=-0.2$
	\item $x=0.2$
\end{enumerate}
\end{multicols}
%P36
\item តម្លៃ $x$ ដែលផ្ទៀងផ្ទាត់សមីការ $x^3=0.125$គឺ៖
\begin{multicols}{4}
\begin{enumerate}[label=\alph*.]
	\item $x=\dfrac{1}{5}$
	\item $x=-\dfrac{1}{5}$
	\item $x=-0.5$
	\item $x=0.5$
\end{enumerate}
\end{multicols}

%P37
\item តម្លៃ $x$ ដែលផ្ទៀងផ្ទាត់សមីការ $x^3=-0.216$គឺ៖
\begin{multicols}{4}
\begin{enumerate}[label=\alph*.]
	\item $x=\dfrac{1}{6}$
	\item $x=-\dfrac{1}{6}$
	\item $x=-0.6$
	\item $x=0.6$
\end{enumerate}
\end{multicols}

%P38
\item តម្លៃ $x$ ដែលផ្ទៀងផ្ទាត់សមីការ $x^3=9$គឺ៖
\begin{multicols}{4}
\begin{enumerate}[label=\alph*.]
	\item $x=3$
	\item $x=-3$
	\item $x=\sqrt{3}$
	\item $x=\sqrt[3]{9}$
\end{enumerate}
\end{multicols}

%P39
\item តម្លៃ $x$ ដែលផ្ទៀងផ្ទាត់សមីការ $x^3=1331$គឺ៖
\begin{multicols}{4}
\begin{enumerate}[label=\alph*.]
	\item $x=121$
	\item $x=-11$
	\item $x=11$
	\item $x=-121$
\end{enumerate}
\end{multicols}

%P40
\item អាងទឹកមួយមានរាងជាគូបដែលមានជ្រុងប្រវែង$1m$ គណនាចំណុះទឹកដែលអាងនោះអាចស្តុកបាន៖
\begin{multicols}{4}
\begin{enumerate}[label=\alph*.]
	\item $100l$
	\item $100m^3$
	\item $10l$
	\item $1000l$
\end{enumerate}
\end{multicols}

%P41
\item កំណត់ចំនួនគត់វិជ្ជមាន $n$ តូចបំផុតដែលធ្វើអោយ $\sqrt[3]{32n}$ ជាចំនួនគត់៖
\begin{multicols}{4}
\begin{enumerate}[label=\alph*.]
	\item $n=1$
	\item $n=2$
	\item $n=3$
	\item $n=4$
\end{enumerate}
\end{multicols}

%P42
\item ផលគុណនៃកន្សោម$A=\sqrt{2}\times \sqrt{8}$  មានតម្លៃស្មើនឹង៖
\begin{multicols}{4}
\begin{enumerate}[label=\alph*.]
	\item $A=2$
	\item $A=3$
	\item $A=4$
	\item $A=5$
\end{enumerate}
\end{multicols}
%P43
\item ផលគុណនៃកន្សោម$A=\sqrt{3}\times \sqrt{27}$  មានតម្លៃស្មើនឹង៖
\begin{multicols}{4}
\begin{enumerate}[label=\alph*.]
	\item $A=-9$
	\item $A=3$
	\item $A=9$
	\item $A=5$
\end{enumerate}
\end{multicols}

%P44
\item ផលគុណនៃកន្សោម$A=\sqrt[3]{2}\times \sqrt[3]{8}$  មានតម្លៃស្មើនឹង៖
\begin{multicols}{4}
\begin{enumerate}[label=\alph*.]
	\item $A=2$
	\item $A=3$
	\item $A=4$
	\item $A=5$
\end{enumerate}
\end{multicols}
%P45
\item ផលគុណនៃកន្សោម$A=\sqrt[3]{2}\times \sqrt[3]{\frac{1}{16}}$  មានតម្លៃស្មើនឹង៖
\begin{multicols}{4}
\begin{enumerate}[label=\alph*.]
	\item $A=2$
	\item $A=\frac{1}{2}$
	\item $A=4$
	\item $A=-\frac{1}{2}$
\end{enumerate}
\end{multicols}

%P46
\item ផលគុណនៃកន្សោម$A=\sqrt{2}\times \sqrt{32}$  មានតម្លៃស្មើនឹង៖
\begin{multicols}{4}
\begin{enumerate}[label=\alph*.]
	\item $A=-9$
	\item $A=3$
	\item $A=9$
	\item $A=5$
\end{enumerate}
\end{multicols}

%P47
\item ផលគុណនៃកន្សោម$A=\sqrt{50}\times \sqrt{2}$  មានតម្លៃស្មើនឹង៖
\begin{multicols}{4}
\begin{enumerate}[label=\alph*.]
	\item $A=-9$
	\item $A=3$
	\item $A=9$
	\item $A=10$
\end{enumerate}
\end{multicols}

%P48
\item ផលគុណនៃកន្សោម$A=\sqrt{0.1}\times \sqrt{10}$  មានតម្លៃស្មើនឹង៖
\begin{multicols}{4}
\begin{enumerate}[label=\alph*.]
	\item $A=-1$
	\item $A=1$
	\item $A=-2$
	\item $A=2$
\end{enumerate}
\end{multicols}

%P49
\item ផលគុណនៃកន្សោម$A=\sqrt[3]{0.1}\times \sqrt[3]{10}$  មានតម្លៃស្មើនឹង៖
\begin{multicols}{4}
\begin{enumerate}[label=\alph*.]
	\item $A=-1$
	\item $A=1$
	\item $A=-2$
	\item $A=2$
\end{enumerate}
\end{multicols}

%P50
\item ផលចែកនៃកន្សោម$A=\dfrac{\sqrt{8}}{\sqrt{2}}$  មានតម្លៃស្មើនឹង៖
\begin{multicols}{4}
\begin{enumerate}[label=\alph*.]
	\item $A=-1$
	\item $A=1$
	\item $A=-2$
	\item $A=2$
\end{enumerate}
\end{multicols}

%P51
\item ផលចែកនៃកន្សោម$A=\dfrac{\sqrt{288}}{\sqrt{2}}$  មានតម្លៃស្មើនឹង៖
\begin{multicols}{4}
\begin{enumerate}[label=\alph*.]
	\item $A=10$
	\item $A=12$
	\item $A=-12$
	\item $A=22$
\end{enumerate}
\end{multicols}

%P52
\item ផលចែកនៃកន្សោម$A=\dfrac{\sqrt{27}}{\sqrt{3}}$  មានតម្លៃស្មើនឹង៖
\begin{multicols}{4}
\begin{enumerate}[label=\alph*.]
	\item $A=-3$
	\item $A=3$
	\item $A=-4$
	\item $A=4$
\end{enumerate}
\end{multicols}

%P53
\item ផលចែកនៃកន្សោម$A=\dfrac{\sqrt{10}}{\sqrt{0.1}}$  មានតម្លៃស្មើនឹង៖
\begin{multicols}{4}
\begin{enumerate}[label=\alph*.]
	\item $A=-10$
	\item $A=10$
	\item $A=-2$
	\item $A=2$
\end{enumerate}
\end{multicols}

%P54
\item ផលចែកនៃកន្សោម$A=\dfrac{\sqrt[3]{48}}{\sqrt[3]{6}}$  មានតម្លៃស្មើនឹង៖
\begin{multicols}{4}
\begin{enumerate}[label=\alph*.]
	\item $A=-10$
	\item $A=10$
	\item $A=-2$
	\item $A=2$
\end{enumerate}
\end{multicols}
%P55
\item ផលចែកនៃកន្សោម$A=\dfrac{\sqrt[3]{16}}{\sqrt[3]{2}}$  មានតម្លៃស្មើនឹង៖
\begin{multicols}{4}
\begin{enumerate}[label=\alph*.]
	\item $A=-10$
	\item $A=10$
	\item $A=-2$
	\item $A=2$
\end{enumerate}
\end{multicols}

%P56
\item ផលចែកនៃកន្សោម$A=\dfrac{\sqrt[3]{25}}{\sqrt[3]{0.2}}$  មានតម្លៃស្មើនឹង៖
\begin{multicols}{4}
\begin{enumerate}[label=\alph*.]
	\item $A=-5$
	\item $A=5$
	\item $A=-2$
	\item $A=2$
\end{enumerate}
\end{multicols}

%P57
\item ផលចែកនៃកន្សោម$A=\dfrac{\sqrt[3]{\frac{1}{2}}}{\sqrt[3]{4}}$  មានតម្លៃស្មើនឹង៖
\begin{multicols}{4}
\begin{enumerate}[label=\alph*.]
	\item $A=-\dfrac{1}{2}$
	\item $A=\dfrac{1}{2}$
	\item $A=-2$
	\item $A=2$
\end{enumerate}
\end{multicols}
%P58
\item ផលចែកនៃកន្សោម$A=\dfrac{\sqrt[3]{100}}{\sqrt[3]{0.1}}$  មានតម្លៃស្មើនឹង៖
\begin{multicols}{4}
\begin{enumerate}[label=\alph*.]
	\item $A=-\dfrac{1}{10}$
	\item $A=\dfrac{1}{10}$
	\item $A=-10$
	\item $A=10$
\end{enumerate}
\end{multicols}

%P59
\item ផលចែកនៃកន្សោម$A=\dfrac{\sqrt{7}}{\sqrt{28}}$  មានតម្លៃស្មើនឹង៖
\begin{multicols}{4}
\begin{enumerate}[label=\alph*.]
	\item $A=-\dfrac{1}{2}$
	\item $A=\dfrac{1}{2}$
	\item $A=-2$
	\item $A=2$
\end{enumerate}
\end{multicols}
%P60
\item ផលចែកនៃកន្សោម$A=\dfrac{\sqrt[3]{128}}{\sqrt[3]{2}}$  មានតម្លៃស្មើនឹង៖
\begin{multicols}{4}
\begin{enumerate}[label=\alph*.]
	\item $A=-\dfrac{1}{4}$
	\item $A=\dfrac{1}{4}$
	\item $A=-4$
	\item $A=4$
\end{enumerate}
\end{multicols}

%P61
\item តម្លៃនៃកន្សោម $A=\sqrt{\frac{2}{8}}$ មានតម្លៃស្មើនឹង៖
\begin{multicols}{4}
\begin{enumerate}[label=\alph*.]
	\item $A=-\dfrac{1}{2}$
	\item $A=\dfrac{1}{2}$
	\item $A=-2$
	\item $A=2$
\end{enumerate}
\end{multicols}

%P62
\item តម្លៃនៃកន្សោម $A=\sqrt[3]{\frac{81}{3}}$ មានតម្លៃស្មើនឹង៖
\begin{multicols}{4}
\begin{enumerate}[label=\alph*.]
	\item $A=-\dfrac{1}{3}$
	\item $A=\dfrac{1}{3}$
	\item $A=-3$
	\item $A=3$
\end{enumerate}
\end{multicols}

%P63
\item តម្លៃនៃកន្សោម $A=\sqrt{12}$ មានតម្លៃស្មើនឹង៖
\begin{multicols}{4}
\begin{enumerate}[label=\alph*.]
	\item $A=-3\sqrt{3}$
	\item $A=3\sqrt{2}$
	\item $A=-2\sqrt{3}$
	\item $A=2\sqrt{3}$
\end{enumerate}
\end{multicols}

%P64
\item តម្លៃនៃកន្សោម $A=\sqrt[3]{81}$ មានតម្លៃស្មើនឹង៖
\begin{multicols}{4}
\begin{enumerate}[label=\alph*.]
	\item $A=3\sqrt{3}$
	\item $A=3\sqrt[3]{3}$
	\item $A=-2\sqrt[3]{3}$
	\item $A=2\sqrt[3]{3}$
\end{enumerate}
\end{multicols}

%P65
\item តម្លៃនៃកន្សោម $A=\sqrt{50}$ មានតម្លៃស្មើនឹង៖
\begin{multicols}{4}
\begin{enumerate}[label=\alph*.]
	\item $A=4\sqrt{2}$
	\item $A=-5\sqrt{2}$
	\item $A=5\sqrt{2}$
	\item $A=-5\sqrt{3}$
\end{enumerate}
\end{multicols}

%P66
\item តម្លៃនៃកន្សោម $A=\sqrt{\dfrac{2}{25}}$ មានតម្លៃស្មើនឹង៖
\begin{multicols}{4}
\begin{enumerate}[label=\alph*.]
	\item $A=\dfrac{2}{5}$
	\item $A=\dfrac{\sqrt{2}}{5}$
	\item $A=-\dfrac{\sqrt{2}}{5}$
	\item $A=-\dfrac{2}{5}$
\end{enumerate}
\end{multicols}

%P67
\item តម្លៃនៃកន្សោម $A=\dfrac{2}{3}\sqrt{\dfrac{12}{5}}$ មានតម្លៃស្មើនឹង៖
\begin{multicols}{4}
\begin{enumerate}[label=\alph*.]
	\item $A=\sqrt{\dfrac{16}{15}}$
	\item $A=\sqrt{\dfrac{25}{16}}$
	\item $A=-\sqrt{\dfrac{16}{15}}$
	\item $A=\sqrt{\dfrac{8}{15}}$
\end{enumerate}
\end{multicols}

%P68
\item តម្លៃនៃកន្សោម $A=5\sqrt{\dfrac{3}{50}}$ មានតម្លៃស្មើនឹង៖
\begin{multicols}{4}
\begin{enumerate}[label=\alph*.]
	\item $A=\sqrt{\dfrac{3}{4}}$
	\item $A=\sqrt{\dfrac{3}{25}}$
	\item $A=\sqrt{\dfrac{3}{2}}$
	\item $A=\sqrt{\dfrac{6}{25}}$
\end{enumerate}
\end{multicols}

%P69
\item តម្លៃនៃកន្សោម $A=-7\sqrt{5}$ មានតម្លៃស្មើនឹង៖
\begin{multicols}{4}
\begin{enumerate}[label=\alph*.]
	\item $A=\sqrt{245}$
	\item $A=\sqrt{235}$
	\item $A=\sqrt{225}$
	\item $A=\sqrt{215}$
\end{enumerate}
\end{multicols}

%P70
\item តម្លៃនៃកន្សោម $A=-5\sqrt[3]{\dfrac{4}{25}}$ មានតម្លៃស្មើនឹង៖
\begin{multicols}{4}
\begin{enumerate}[label=\alph*.]
	\item $A=-\sqrt[3]{10}$
	\item $A=-\sqrt[3]{20}$
	\item $A=\sqrt[3]{10}$
	\item $A=\sqrt[3]{20}$
\end{enumerate}
\end{multicols}

%P71
\item តម្លៃនៃកន្សោម $A=-2\sqrt[3]{\dfrac{5}{12}}$ មានតម្លៃស្មើនឹង៖
\begin{multicols}{4}
\begin{enumerate}[label=\alph*.]
	\item $A=-\sqrt[3]{\dfrac{10}{3}}$
	\item $A=\sqrt[3]{\dfrac{10}{3}}$
	\item $A=-\sqrt[3]{\dfrac{10}{7}}$
	\item $A=\sqrt[3]{\dfrac{10}{13}}$
\end{enumerate}
\end{multicols}

%P72
\item តម្លៃនៃកន្សោម $A=\dfrac{1}{2}\sqrt[3]{4}$ មានតម្លៃស្មើនឹង៖
\begin{multicols}{4}
\begin{enumerate}[label=\alph*.]
	\item $A=-\sqrt[3]{\dfrac{1}{2}}$
	\item $A=\sqrt[3]{\dfrac{1}{2}}$
	\item $A=-\sqrt[3]{\dfrac{2}{3}}$
	\item $A=\sqrt[3]{\dfrac{3}{2}}$
\end{enumerate}
\end{multicols}

%P73
\item តម្លៃនៃកន្សោម $A=3\sqrt{11}+5\sqrt{44}-3\sqrt{99}$ គឺ៖
\begin{multicols}{4}
\begin{enumerate}[label=\alph*.]
	\item $A=2\sqrt{11}$
	\item $A=-2\sqrt{11}$
	\item $A=3\sqrt{11}$
	\item $A=4\sqrt{11}$	
\end{enumerate}
\end{multicols}

%P74
\item តម្លៃនៃកន្សោម $A=3\sqrt{18}-\sqrt{12}+\sqrt{75}+\sqrt{2}$ គឺ៖
\begin{multicols}{2}
\begin{enumerate}[label=\alph*.]
	\item $A=10\sqrt{2}+3\sqrt{3}$
	\item $A=10\sqrt{2}-3\sqrt{3}$
	\item $A=7\sqrt{2}+3\sqrt{3}$
	\item $A=10\sqrt{2}+4\sqrt{3}$
\end{enumerate}
\end{multicols}

%P75
\item តម្លៃនៃកន្សោម $A=3\sqrt[3]{24}+6\sqrt[3]{81}$ គឺ៖
\begin{multicols}{4}
\begin{enumerate}[label=\alph*.]
	\item $A=23\sqrt[3]{3}$
	\item $A=24\sqrt[3]{3}$
	\item $A=25\sqrt[3]{3}$
	\item $A=26\sqrt[3]{3}$	
\end{enumerate}
\end{multicols}

%P76
\item តម្លៃនៃកន្សោម $A=6\sqrt[3]{8x^2}-2\sqrt[3]{27x^2}$ គឺ៖
\begin{multicols}{4}
\begin{enumerate}[label=\alph*.]
	\item $A=-6\sqrt[3]{x^2}$
	\item $A=6\sqrt[3]{x^2}$
	\item $A=12\sqrt[3]{x^2}$
	\item $A=8\sqrt[3]{x^2}$
\end{enumerate}
\end{multicols}

%P77
\item តម្លៃនៃកន្សោម $A=\sqrt{20}+\sqrt{80}-\sqrt{45}$ គឺ៖
\begin{multicols}{4}
\begin{enumerate}[label=\alph*.]
	\item $A=3\sqrt{5}$
	\item $A=-3\sqrt{5}$
	\item $A=4\sqrt{5}$
	\item $A=-4\sqrt{5}$
\end{enumerate}
\end{multicols}


%P78
\item តម្លៃនៃកន្សោម $A=14\sqrt{3}+6\sqrt{2}-11\sqrt{3}$ គឺ៖
\begin{multicols}{2}
\begin{enumerate}[label=\alph*.]
	\item $A=3\sqrt{3}+6\sqrt{2}$
	\item $A=3\sqrt{3}-6\sqrt{2}$
	\item $A=-3\sqrt{3}+6\sqrt{2}$
	\item $A=-3\sqrt{3}-6\sqrt{2}$
\end{enumerate}
\end{multicols}

%P79
\item តម្លៃនៃកន្សោម $A=5\sqrt{50}-8\sqrt{32}$ គឺ៖
\begin{multicols}{4}
\begin{enumerate}[label=\alph*.]
	\item $A=7\sqrt{2}$
	\item $A=-7\sqrt{2}$
	\item $A=8\sqrt{2}$
	\item $A=-8\sqrt{2}$
\end{enumerate}
\end{multicols}

%P80
\item តម្លៃនៃកន្សោម $A=\sqrt{12x+12}+\sqrt{27x+27}$ គឺ៖
\begin{multicols}{2}
\begin{enumerate}[label=\alph*.]
	\item $A=5\sqrt{3(x+1)}$
	\item $A=-5\sqrt{3(x+1)}$
	\item $A=6\sqrt{3(x+1)}$
	\item $A=-6\sqrt{3(x+1)}$
\end{enumerate}
\end{multicols}

%P81
\item តម្លៃនៃកន្សោម $A=\sqrt{128y}-\sqrt{2y},\quad y>0$ គឺ៖
\begin{multicols}{2}
\begin{enumerate}[label=\alph*.]
	\item $A=5\sqrt{2y}$
	\item $A=-5\sqrt{2y}$
	\item $A=6\sqrt{2y}$
	\item $A=7\sqrt{2y}$
\end{enumerate}
\end{multicols}

%P82
\item ក្រោយពីការបំបាត់រ៉ាឌីកាល់ពីភាគបែងកន្សោម $A=\dfrac{\sqrt{3}+\sqrt{2}}{\sqrt{3}-\sqrt{2}}$ អាចសរសេរជាទម្រង់៖
\begin{multicols}{2}
\begin{enumerate}[label=\alph*.]
	\item $A=5+2\sqrt{6}$
	\item  $A=-5+2\sqrt{6}$
	\item $A=5-2\sqrt{6}$
	\item  $A=-5-2\sqrt{6}$
\end{enumerate}
\end{multicols}

%P83
\item ក្រោយពីការបំបាត់រ៉ាឌីកាល់ពីភាគបែងកន្សោម $A=\dfrac{1+\sqrt{2}}{3-\sqrt{3}}$ អាចសរសេរជាទម្រង់៖
\begin{multicols}{2}
\begin{enumerate}[label=\alph*.]
	\item $A=\dfrac{3+\sqrt{3}+3\sqrt{2}+\sqrt{6}}{6}$
	\item  $A=\dfrac{3-\sqrt{3}+3\sqrt{2}+\sqrt{6}}{6}$
	\item $A=\dfrac{3+\sqrt{3}-3\sqrt{2}+\sqrt{6}}{6}$
	\item  $A=\dfrac{3+\sqrt{3}+3\sqrt{2}+\sqrt{6}}{6}$
\end{enumerate}
\end{multicols}

%P84
\item ក្រោយពីការបំបាត់រ៉ាឌីកាល់ពីភាគបែងកន្សោម $A=\dfrac{1+\sqrt{2}}{2+\sqrt{5}}$ អាចសរសេរជាទម្រង់៖
\begin{multicols}{2}
\begin{enumerate}[label=\alph*.]
	\item $A={-2+\sqrt{5}-2\sqrt{3}+\sqrt{15}}$
	\item $A={-2+\sqrt{5}+2\sqrt{3}+\sqrt{15}}$
	\item $A={-2+\sqrt{5}-2\sqrt{3}-\sqrt{15}}$
	\item $A={-2-\sqrt{5}-2\sqrt{3}+\sqrt{15}}$
\end{enumerate}
\end{multicols}

%P85
\item ក្រោយពីការបំបាត់រ៉ាឌីកាល់ពីភាគបែងកន្សោម $A=\dfrac{8\sqrt{2}}{\sqrt{20}-\sqrt{18}}$ អាចសរសេរជាទម្រង់៖
\begin{multicols}{2}
\begin{enumerate}[label=\alph*.]
	\item $A=8\sqrt{10}+24$
	\item $A=-8\sqrt{10}+24$
	\item $A=8\sqrt{10}-24$
	\item $A=6\sqrt{10}+24$
\end{enumerate}
\end{multicols}
\end{enumerate}
%%%%%%%%%%%%%%%%%%%%%%%%%%%%%%%%%%%%%
\newpage
\section{ផ្នែកលំហាត់ត្រិះរិះ}
%N1
\begin{enumerate}
\item ចូរគណនាតម្លៃនៃកន្សោម៖
\begin{enumerate}[k,4]
\item $\sqrt{9}$
\item $\sqrt{16}$
\item $\sqrt{36}$
\item $-\sqrt{64}$
\item $-\sqrt{100}$
\item $\sqrt{121}$
\item $-\sqrt{144}$
\item $\sqrt{625}$
\item $\sqrt[3]{8}$
\item $\sqrt[3]{-8}$
\item $\sqrt[3]{27}$
\item $\sqrt[3]{64}$
\item $\sqrt[3]{125}$
\item $\sqrt[3]{216}$
\item $\sqrt[3]{1000}$
\end{enumerate}
%N2
\item ចូរគណនាតម្លៃកន្សោមខាងក្រោម៖
\begin{enumerate}[k,4]
\item $\sqrt{\dfrac{9}{16}}$
\item $\sqrt{\dfrac{49}{9}}$
\item $\sqrt{\dfrac{81}{4}}$
\item $\sqrt{\dfrac{169}{49}}$
\item $\sqrt{\dfrac{196}{25}}$
\item $\sqrt{\dfrac{400}{225}}$
\item $\sqrt[3]{\dfrac{1}{8}}$
\item $\sqrt[3]{\dfrac{8}{27}}$
\item $\sqrt[3]{\dfrac{64}{125}}$
\item $\sqrt[3]{\dfrac{512}{343}}$
\item $\sqrt[3]{\dfrac{216}{1000}}$
\end{enumerate}
%N3
\item ចូរបញ្ចេញចំនួនខាងក្រោមពីរ៉ាឌីកាល់៖
\begin{enumerate}[k,4]
\item $\sqrt{16^3}$
\item $-\sqrt{36^2}$
\item $\sqrt{64^3}$
\item $\sqrt[3]{-8^3}$
\item $\sqrt[3]{-27^3}$
\item $\sqrt[3]{1^5}$
\item $\sqrt[3]{8^2}$
\item $\sqrt[3]{64^2}$
\item $\sqrt[3]{(-27)^2}$
\end{enumerate}
%N4
\item ចូរគណនាតម្លៃកន្សោមខាងក្រោម៖
\begin{enumerate}[k,4]
\item $\sqrt{y^2}$
\item $\sqrt{x^4}$
\item $\sqrt{x^2y^4}$
\item $\sqrt{y^6}$
\item $\sqrt{\dfrac{16}{x^2}}$
\item $\sqrt{\dfrac{100}{n^4}}$
\item $\sqrt[3]{8x^3}$
\item $\sqrt{64m^3}$
\end{enumerate}
%N5
\item ចូរគណនាតម្លៃកន្សោមខាងក្រោម៖
\begin{enumerate}[k,4]
\item $\sqrt{(2x)^2}$
\item $\sqrt[3]{(-5y)^3}$
\item $\sqrt{(4-a)^2}$
\item $\sqrt[3]{(x+3)^3}$
\item $\sqrt{16b^2+24b+9}$
\item $\sqrt{9x^2-30x+25}$
\item $\sqrt{4m^2-20mn+25n^2}$
\item $\sqrt{49x^2-112xy+64y^2}$
\end{enumerate}
%N6
\item ចូរបញ្ចេញមួយចំនួនពីរ៉ាឌីកាល់៖
\begin{enumerate}[k,4]
\item $\sqrt{18}$
\item $\sqrt{48}$
\item $\sqrt{75}$
\item $\sqrt{\dfrac{30}{49}}$
\item $\sqrt{\dfrac{10}{121}}$
\item $\sqrt[3]{40}$
\item $\sqrt[3]{54}$
\item $\sqrt[3]{128}$
\item $\sqrt[3]{192}$
\item $\sqrt[3]{\dfrac{3m}{8n^3}}$
\item $\sqrt[3]{16a^5}$
\end{enumerate}

%N7
\item ចូរបញ្ចេញមួយចំនួនពីរ៉ាឌីកាល់៖
\begin{enumerate}[k,4]
\item $\sqrt{36a^3b^3}$
\item $\sqrt{27a^4b^3}$
\item $\sqrt{72x^5y^2}$
\item $\sqrt{112a^3b^4}$
\item $\sqrt{80m^4n^3}$
\item $\sqrt{64x^2y^3}$
\item $\sqrt[3]{16m^3n^3}$
\item $\sqrt[3]{54x^4b^3}$
\item $\sqrt[3]{128a^5y^3}$
\item $\sqrt[3]{24p^3q^5}$
\end{enumerate}

%N8
\item ចូរបញ្ចូលមួយចំនួនទៅក្នុងរ៉ាឌីកាល់$x,y,m$ និង $n$ ជាចំនួនពិតវិជ្ជមាន។
\begin{enumerate}[k,4]
\item $5\sqrt{6}$
\item $2m\sqrt{m}$
\item $\dfrac{\sqrt{23}}{y^3}$
\item $2\sqrt[3]{5}$
\item $2x\sqrt[3]{4}$
\item $\dfrac{\sqrt[3]{3m}}{2n}$
\end{enumerate}
%N9
\item ចូរគណនាផលបូក និង ផលដកកន្សោមខាងក្រោម៖
\begin{enumerate}[k,4]
\item $3\sqrt{2}-4\sqrt{2}+5\sqrt{2}-3\sqrt{2}$
\item $5\sqrt{2}-3\sqrt{3}-6\sqrt{2}+5\sqrt{3}$
\item $3\sqrt{15}-4\sqrt{3}-3\sqrt{15}+6\sqrt{3}$
\item $4\sqrt{3}-2\sqrt{17}+3\sqrt{17}-3\sqrt{3}-2\sqrt{2}$
\item $2\sqrt[3]{2}-8\sqrt[3]{3}+\sqrt[3]{2}+3\sqrt[3]{3}$
\item $8\sqrt[3]{2}-3\sqrt[3]{3}-5\sqrt[3]{2}+2\sqrt[3]{3}$
\end{enumerate}

%N10
\item ចូរគណនាតម្លៃនៃកន្សោមខាងក្រោម៖
\begin{enumerate}[label=\alph*.]
\begin{multicols}{3}
\item $\dfrac{2}{3}\sqrt{27}-\dfrac{3}{4}\sqrt{48}$
\item $\dfrac{1}{4}\sqrt{288}-\dfrac{1}{6}\sqrt{72}$
\item $\dfrac{3}{5}\sqrt{75}-\dfrac{2}{3}\sqrt{27}$
\item $5\sqrt[3]{128}-3\sqrt[3]{250}$
\item $3\sqrt[3]{81}-\dfrac{1}{2}\sqrt[3]{192}$
\item $4\sqrt[3]{54}-3\sqrt[3]{128}$
\end{multicols}
\end{enumerate}

%N11
\item ចូរគណនាតម្លៃកន្សោមខាងក្រោម៖
\begin{enumerate}[label=\alph*.]
\begin{multicols}{2}
\item $2\sqrt{8}-3\sqrt{98}-2\sqrt{200}$
\item $-3\sqrt{50}-\sqrt{32}+5\sqrt{200}$
\item $3\sqrt{175}-2\sqrt{28}+3\sqrt{63}-\sqrt{112}$
\item $\sqrt{108}-2\sqrt{27}-\sqrt{40}-5\sqrt{160}$
\item $2\sqrt[3]{16}+3\sqrt[3]{54}-2\sqrt[3]{128}$
\item $3\sqrt[3]{81}+\dfrac{1}{2}\sqrt[3]{128}-3\sqrt[3]{192}+4\sqrt[3]{54}$
\item $4\sqrt[3]{54}-6\sqrt[3]{81}-4\sqrt[3]{16}+3\sqrt[3]{24}$
\item $-2\sqrt[3]{40}-3\sqrt[3]{135}+5\sqrt[3]{320}+8\sqrt[3]{5}$
\end{multicols}
\end{enumerate}

%N12
\item ចូរគណនាតម្លៃកន្សោមខាងក្រោម៖
\begin{enumerate}[label=\alph*.]
\begin{multicols}{2}
\item $-2(2\sqrt{12}-\sqrt{18})-5(3\sqrt{32}-\sqrt{27})$
\item $\dfrac{2\sqrt{27}}{3}-3\sqrt{48}+\dfrac{4\sqrt{50}}{5}-\dfrac{4\sqrt{18}}{3}$
\item $3(3\sqrt[3]{40}-\sqrt[3]{135})+4(\sqrt[3]{320}-\sqrt[3]{40})$
\item $\dfrac{2}{3}\sqrt[3]{81}-\dfrac{1}{2}\sqrt[3]{24}+\dfrac{2\sqrt[3]{135}}{3}-\dfrac{3\sqrt[3]{40}}{2}$
\end{multicols}
\end{enumerate}

%N13
\item ចូរគណនាកន្សោមខាងក្រោម $a,b,x,y$ និង $z$ ជាចំនួនវិជ្ជមាន៖
\begin{enumerate}[label=\alph*.]
\begin{multicols}{2}
\item $-3\sqrt{32x}+6\sqrt{8x}$
\item $2\sqrt{125x^2z}+8x\sqrt{80z}$
\item $7a\sqrt{b^3}+b\sqrt{4a^2b}-\sqrt{4b}$
\item $8b\sqrt{49b}-7\sqrt{9b^3}+a\sqrt{4a}+\sqrt{a^3}$
\item $3xy\sqrt{x^2y}-2\sqrt{x^4y^3}$
\item $-3a\sqrt{a^3b^5}-2b\sqrt{a^5b^3}+5\sqrt{a^3b^3}$
\item $8a\sqrt[3]{54a}+6\sqrt[3]{16a^4}$
\item $3\sqrt[3]{x^4y}-6\sqrt[3]{xy^4}+2\sqrt[3]{x^4y^4}$
\end{multicols}
\end{enumerate}
%N14
\item ចូរគណនាតម្លៃលេខនៃកន្សោម $A$ ចំពោះ $a=5,b=3; A=\sqrt{ab}-\sqrt{ab^3}-\sqrt{9a^3b^3}-\sqrt{a^3b}$។
%N15
\item ចូរគណនាតម្លៃលេខនៃកន្សោម $A=\sqrt{4a}+a\sqrt{a^2b}+\sqrt{b^2a}+b\sqrt{9b}$ ចំពោះ $a=3, b=2$ ។
%N16
\item ចូរគណនា៖
\begin{enumerate}[label=\alph*.]
\begin{multicols}{3}
\item $(2\sqrt{3})(3\sqrt{2})$
\item $(4\sqrt{6})(-2\sqrt{5})$
\item $(3\sqrt{5})(5\sqrt{3})$
\item $(6\sqrt{2})(-12\sqrt{3})$
\item $(3\sqrt{8})(-3\sqrt{48})$
\item $(-3\sqrt{75})(-2\sqrt{48})$
\item $2\sqrt[3]{3}\left(-\dfrac{1}{2}\sqrt[3]{2}\right)$
\item $(3\sqrt[3]{2})(5\sqrt[3]{15})$
\item $(6\sqrt[3]{8})(-3\sqrt[3]{2})$
\end{multicols}
\end{enumerate}

%N17
\item ចូរគណនា៖
\begin{enumerate}[label=\alph*.]
\begin{multicols}{3}
\item $3\sqrt{5}(2\sqrt{18}-3\sqrt{48})$
\item $-3\sqrt{3}(3\sqrt{6}-3\sqrt{2})$
\item $\dfrac{1}{2}\sqrt{3}(2\sqrt{48}-3\sqrt{32})$
\item $\dfrac{3}{2}\sqrt{2}(2\sqrt{18}-3\sqrt{48})$
\item $-4\sqrt[3]{3}(2\sqrt[3]{6}-2\sqrt[3]{5})$
\item $2\sqrt[3]{5}(3\sqrt[3]{3}-5\sqrt[3]{2})$
\item $3\sqrt[3]{3}(3\sqrt[3]{8}-2\sqrt[3]{18})$
\item $-3\sqrt[3]{5}(4\sqrt[3]{20}-2\sqrt[3]{45})$
\item $3\sqrt[3]{4}(4\sqrt[3]{2}-7\sqrt[3]{16})$
\end{multicols}
\end{enumerate}
%N18
\item ចូរគណនា៖
\begin{enumerate}[label=\alph*.]
\begin{multicols}{2}
\item $(2\sqrt{3}-8)(9+2\sqrt{5})$
\item $(3\sqrt{5}-2\sqrt{10})(\sqrt{50}-2\sqrt{80})$
\item $(\sqrt{50}-\sqrt{75})(\sqrt{32}-\sqrt{48})$
\item $(\sqrt{125}-\sqrt{75})(\sqrt{80}-\sqrt{48})$
\item $(3\sqrt[3]{18}+3\sqrt[3]{27})(2\sqrt[3]{8}-2\sqrt[3]{12})$
\item $(\sqrt[3]{80}-2\sqrt[3]{27})(-3\sqrt[3]{20}-3\sqrt[3]{12})$
\end{multicols}
\end{enumerate}

%N19
\item គេឲ្យ $a=3\sqrt{5}-2\sqrt{10},b=5\sqrt{7}+2\sqrt{10},c=\sqrt[3]{18}-\sqrt[3]{27}$ និង $d=3\sqrt[3]{6}+\sqrt[3]{8}$។ ចូរគណនា៖
\begin{enumerate}[label=\alph*.]
\begin{multicols}{4}
\item $-3ab$
\item $a^2+b^2$
\item $a^2-b^2$
\item $a^2-2b^2$
\item $b^2-2ab$
\item $\dfrac{1}{2}cd$
\item $c^2-b^2$
\item $c^2+2cd$
\end{multicols}
\end{enumerate}
%N20
\item ចូរសម្រួលកន្សោមខាងក្រោម៖
\begin{enumerate}[label=\alph*.]
\begin{multicols}{2}
\item $\sqrt{\dfrac{b^2}{b^2-14b+49}}$
\item $\sqrt{\dfrac{49x^2-56x+16}{36x^2}}$
\item $\sqrt{\dfrac{a^2+16ab+64b^2}{a^2+10ab+25b^2}}$
\item $\sqrt{\dfrac{25b^2+10ab+a^2}{16b^2+24ab+9a^2}}$
\end{multicols}
\end{enumerate}


%N21
\item ចូរបំបាត់រ៉ាឌីកាល់ពីភាគបែង៖
\begin{enumerate}[label=\alph*.]
\begin{multicols}{4}
\item $\dfrac{3}{\sqrt{10}}$
\item $\dfrac{3\sqrt{3}}{\sqrt{33}}$
\item $\dfrac{6}{\sqrt{48}}$
\item $\dfrac{8}{\sqrt{27}}$
\item $\dfrac{9\sqrt{4}}{\sqrt{18}}$
\item $\dfrac{5\sqrt{5}}{\sqrt{72}}$
\item $\dfrac{\sqrt[3]{9}}{\sqrt[3]{10}}$
\item $\dfrac{2\sqrt[3]{3}}{\sqrt[3]{30}}$
\item $\dfrac{\sqrt[3]{18}}{\sqrt[3]{20}}$
\item $\sqrt[3]{\dfrac{7m}{36n}}$
\item $\sqrt[3]{\dfrac{11p}{49q}}$
\item $\sqrt[3]{\dfrac{3}{4y^2}}$
\end{multicols}
\end{enumerate}
%N22
\item ចូរបំបាត់រ៉ាឌីកាល់ពីភាគបែង៖
\begin{enumerate}[label=\alph*.]
\begin{multicols}{4}
\item $\dfrac{36-\sqrt{6}}{\sqrt{8}}$
\item $\dfrac{\sqrt{3}+\sqrt{5}}{3\sqrt{20}}$
\item $\dfrac{8}{2\sqrt{75}-3\sqrt{50}}$
\item $\dfrac{2\sqrt{3}}{2\sqrt{80}-\sqrt{45}}$
\item $\dfrac{9-\sqrt[3]{3}}{2\sqrt[3]{32}}$
\item $\dfrac{5\sqrt[3]{4}+\sqrt[3]{3}}{8\sqrt[3]{13}}$
\item $\dfrac{2\sqrt[3]{6}}{2\sqrt[3]{27}-\sqrt[3]{9}}$
\item $\dfrac{2\sqrt[3]{2}}{\sqrt[3]{16}-\sqrt[3]{12}}$
\end{multicols}
\end{enumerate}

%N23
\item គេឲ្យ $m=3\sqrt{8}+\sqrt{5}$ និង $n=3\sqrt{8}-\sqrt{5}$។ ចូរគណនា៖
\begin{enumerate}[label=\alph*.]
\begin{multicols}{3}
\item $\dfrac{mn+m^2}{m}$
\item $\dfrac{m^2-n^2}{m+n}$
\item $\dfrac{n^2-2mn}{n}$
\end{multicols}
\end{enumerate}

%N24
\item 
\begin{enumerate}[label=\alph*.]
\begin{multicols}{2}
\item ចូរប្រៀបធៀបចំនួន $\sqrt{2}$ និង $\sqrt[3]{3}$។
\item ចូរសម្រួលកន្សោម $A=\sqrt{22-\sqrt{288}}$។
\end{multicols}
\end{enumerate}
%N25
\item 
\begin{enumerate}[label=\alph*.]
\begin{multicols}{2}
\item ចូរបង្ហាញថា $2^{2019}+2^{2019}=2^{2020}$។
\item ចូរកំណត់តម្លៃ $x$ ដែល $2^x\cdot 2^{x+3}=8^4$។
\end{multicols}
\end{enumerate}



\end{enumerate}






