\chapter{សមាមាត្រ}
	
\section{សមាមាត្រ}
\begin{definition}
បើ $a,b,c,d$ ជាចំនួនពិតខុសពីសូន្យព្រមគ្នាគេបាន $\dfrac{a}{b}=\dfrac{c}{d}\Leftrightarrow ad=bc$។
\end{definition}
\subsection{សមាមាត្រស្រប}
\begin{definition}
ក្នុងសមាមាត្រ ផលគុណតួចុងស្មើនឹងផលគុណតួមធ្យម $\dfrac{x_1}{x_2}=\dfrac{y_2}{y_1}\Leftrightarrow x_1x_2=y_1y_2$។
\end{definition}
\subsection{សមាមាត្រច្រាស}
\begin{definition}
ក្នុងសមាមាត្រ ផលគុណតួចុងស្មើនឹងផលគុណតួមធ្យម $\dfrac{y_1}{x_2}=\dfrac{y_2}{x_2}\Leftrightarrow x_2y_1=x_1y_2$។
\end{definition}
\section{លក្ខណៈសមាមាត្រ}
\begin{definition}
គេមាន $x_1,x_2,x_3,y_1,y_2$ និង $y_3$ ជាចំនួនពិត៖
\begin{itemize}
\item បើ $x_1$ និង $x_2$ ជាពីរចំនួនខុសពីសូន្យដែល $x_1+x_2\ne 0$ គេបាន $\dfrac{y_1}{x_1}=\dfrac{y_2}{x_2}\Leftrightarrow \dfrac{y_1}{x_1}=\dfrac{y_2}{x_2}=\dfrac{y_1+y_2}{x_1+x_2}$។
\item បើ $x_1,x_2$ និង $x_3$ ជាពីរចំនួនខុសពីសូន្យដែល $x_1+x_2+x_3\ne 0$ \\គេបាន $\dfrac{y_1}{x_1}=\dfrac{y_2}{x_2}=\dfrac{y_3}{x_3}\Leftrightarrow \dfrac{y_1}{x_1}=\dfrac{y_2}{x_2}=\dfrac{y_1+y_2+y_3}{x_1+x_2+x_3}$។
\end{itemize}
\end{definition}
\section{ចំណោទដែលទាក់ទង និង សមាមាត្រ}
\section{ចំណោទដែលទាក់ទង និង ភាគរយ}
\section{ចំណោទនៃភាគរយកើន ឬ ថយ}
\section{ការប្រាក់}


%%%%%%%%%%%%%%%%%%%%%%%%%%%%%%%%%%%%
\newpage
\pros
\begin{enumerate}
%N1
\item តម្លៃនៃ $0.5$ ស្មើនឹង៖
\begin{multicols}{4}
\begin{enumerate}[label=\alph*.]
\item $5\%$
\item $0.05\%$
\item $\dfrac{1}{2}$
\item $\frac{1}{5}$
\end{enumerate}
\end{multicols}

%N2
\item កន្សោម$\dfrac{6}{36}$ មានតម្លៃស្មើនឹង៖
\begin{multicols}{4}
\begin{enumerate}[label=\alph*.]
\item $\dfrac{1}{3}$
\item $\dfrac{1}{5}$
\item $\dfrac{1}{6}$
\item $\dfrac{1}{7}$
\end{enumerate}
\end{multicols}

%N3
\item កន្សោម$\dfrac{4}{28}$ មានតម្លៃស្មើនឹង៖
\begin{multicols}{4}
\begin{enumerate}[label=\alph*.]
\item $\dfrac{1}{3}$
\item $\dfrac{1}{5}$
\item $\dfrac{1}{6}$
\item $\dfrac{1}{7}$
\end{enumerate}
\end{multicols}

%N4
\item កន្សោម$\dfrac{9}{81}$ មានតម្លៃស្មើនឹង៖
\begin{multicols}{4}
\begin{enumerate}[label=\alph*.]
\item $\dfrac{1}{8}$
\item $\dfrac{1}{9}$
\item $\dfrac{1}{6}$
\item $\dfrac{1}{7}$
\end{enumerate}
\end{multicols}

%N5
\item កន្សោម$\dfrac{81}{27}$ មានតម្លៃស្មើនឹង៖
\begin{multicols}{4}
\begin{enumerate}[label=\alph*.]
\item $\dfrac{3}{8}$
\item $\dfrac{4}{3}$
\item $3$
\item $\dfrac{1}{7}$
\end{enumerate}
\end{multicols}

%N6
\item សៀវភៅ១០ក្បាលថ្លៃ$12000$រៀល នោះសៀវភៅ១០០ក្បាលមានតម្លៃស្មើនឹង៖
\begin{multicols}{4}
\begin{enumerate}[label=\alph*.]
\item $120$រៀល
\item $2400$រៀល
\item $12000$រៀល
\item $120000$រៀល
\end{enumerate}
\end{multicols}

%N7
\item សៀវភៅគណិត $3$ក្បាលថ្លៃ$27000$រៀល នោះសៀវភៅគណិតវិទ្យា ១៥ ក្បាលមានតម្លៃស្មើនឹង៖
\begin{multicols}{4}
\begin{enumerate}[label=\alph*.]
\item $250000$រៀល
\item $122500$រៀល
\item $235000$រៀល
\item $135000$រៀល
\end{enumerate}
\end{multicols}


\end{enumerate}


%%%%%%%%%%%%%%%%%%%%%%%%%%%%%%%%%%%%%%%%%%%%
\newpage 
\problem

