\chapter*{អារម្ភកថា}
\addcontentsline{toc}{chapter}{អារម្ភកថា}
កថាខណ្ឌនេះពិពណ៌នាអំពីដំណើរដងទងនៃការចាប់កំណើតឡើងនៃសៀវភៅនេះ។ ដំបូងឡើយវាគ្រាន់តែជាកម្រងលំហាត់សម្រាប់ឲ្យសិស្សអនុវត្តន៏បន្ថែមលើការសិក្សាម៉ោងរដ្ធតែប៉ុណ្ណោះ។ ដោយពេលវេលាមានរយៈពេលខ្លី ការដាក់ឧទាហរណ៏ និងលំហាត់គំរូពុំសូវបានច្រើនជាហេតុបណ្ដាលអោយខ្ញុំកើតគំនិតសរសេរចម្លើយដើម្បីអោយសិស្សអាន និងអនុវត្តន៏ដោយខ្លួនឯង។
\\[1em]
សៀវភៅនេះបែងចែកជាបួនផ្នែករួមមាន មេរៀនសង្ខេបអមដោយឧទាហរណ៏គំរូ កម្រងលំហាត់បញ្ចប់មេរៀន ដំណោះស្រាយលំហាត់ និង សេចក្ដីបន្ថែម។ នៅផ្នែកមេរៀនសង្ខេបយើងមាន ការរំលឹកខ្លី និយមន័យ លក្ខណៈ និងទ្រឹស្ដីបទ។ ឧទាហរណ៏គំរូសម្រាប់និយមន័យនីមួយៗ ក៏ត្រូវបានរួមបញ្ចូលនៅផ្នែកនេះដែរ។ សម្រាប់សម្រាយបញ្ជាក់ លក្ខណៈ និងទ្រីស្ដីបទសំខាន់អ្នកអានរកមើលនៅផ្នែកបន្ថែមដែលបានដាក់នៅជំពូកចុងក្រោយគេបង្អស់នៃសៀវភៅ។ នៅផ្នែកកម្រងលំហាត់បញ្ចប់មេរៀន យើងមានតែលំហាត់សុទ្ធដែលត្រូវបានរៀបចំតាមខ្លឹមសារមេរៀន និងតាមលំដាប់កើននៃភាពលំបាក។ បន្ទាប់ពីផ្នែកនេះគឺជាចម្លើយលើកម្រងលំហាត់។ រីឯផ្នែកចុងក្រោយ ជាសេចក្ដីបន្ថែម ដែលភាគច្រើនដកស្រង់ចេញពីមេរៀនថ្នាក់ក្រោម។ អ្នកអានគួរផ្ដោតការយកចិត្តទុកលើផ្នែកនេះជាចំបង។ ផ្នែកនេះគួរតែអានមុនគេដើម្បីបង្កភាពងាយស្រួលក្នុងការអានផ្នែកផ្សេងៗទៀត។
\\[1em]
បញ្ជាក់ជួនដល់អ្នកអានសៀវភៅនេះឲ្យបានជ្រាបថា វាគឺជាស្នារដៃដំបូងរបស់អ្នកនិពន្ធ។ សៀវភៅនេះត្រូវបានបង្កើតឡើងដោយមនុស្សតែម្នាក់ប៉ុណ្ណោះ។ ជាងនេះទៅទៀតវាពុំទាន់បានឆ្លងកាត់ការត្រួតពិនិត្យទាំងផ្នែកបច្ចេកទេស និងអក្ខរាវិរុទ នៅឡើយទេ។ បើប្រិយមិត្តរកឃើញកំហុសឆ្គងណាមួយ សូមជួនដំណឹងដល់អ្នកសរសេរសៀវភៅដោយការផ្ញើរសារជាអក្សរ ឬ រូបភាពមកកាន់ប្រអប់សារអេឡិចត្រូនិច ដែលមានអាស័យដ្ឋាន \textcolor{magenta}{\itshape bunnybookauthor@gmail.com} បើមិនអញ្ចឹងទេអ្នកអាចជួបពិភាក្សាផ្ទាល់បើអាចធ្វើទៅបាន។
\\[1em]
ទាក់ទិននឹងការធ្វើអាជីវកម្មលើសៀវភៅនេះ អ្នកនិពន្ធរក្សាសិទ្ធិកម្មសិទ្ធិបញ្ញាដោយមិនអនុញ្ញាតអោយធ្វើការបោះពុម្ភ ថតចំលង ឬចែកចាយដោយគ្មានការអនុញ្ញាតឡើយ។ ចំពោះកំណាត់សៀវភៅនេះជាឯកសារអេឡិចត្រូនិច អ្នកអាចទាញយកមកអាន និងប្រើប្រាស់ផ្ទាល់ខ្លួនបានដោយមិនគិតថ្លៃតាមរយៈដំណរ\\ \textcolor{magenta}{\itshape bunnybookshelf.blogspot.com/p/conic.html}~។