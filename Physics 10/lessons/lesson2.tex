\chapter{ចលនាអង្គធាតុតាមមួយវិមាត្រ}
\section{ចលនាមេកានិច}
\begin{definition}
	\begin{itemize}
		\item បម្លាស់ប្តូរទីតាំងអង្គធាតុមួយធៀបនឹងអង្គធាតុមួយទៀត ហៅថាចលនាមេកានិច។
		\item ចំពោះអង្គធាតុណាមួយដែលកំណត់ចលនានៃអង្គធាតុផ្សេងទៀតធៀបនឹងវា គេហៅថាអង្គធាតុនោះថា តម្រុយ។
	\end{itemize}
\end{definition}
\section{បម្លាស់ទី ល្បឿន វុិចទ័រល្បឿន}
\quad យើងនឹងបែងចែងអំពីភាពខុសគ្នារវាង ចម្ងាយចរ និងបម្លាស់ទី ហើយល្បឿន និងវ៉ិចទ័រល្បឿន។
\subsection{ចម្ងាយចរ និងបម្លាស់ទី}
\begin{definition}
	\emph{\kml ចម្ងាយចរៈ} ជាប្រវែងសរុបនៃចលនារបស់អង្គធាតុដោយមិនគិតពីទិសដៅនៃចលនា។\\
	\emph{\kml បម្លាស់ទីៈ} ជាចម្ងាយចរដែលវាស់តាមខ្សែត្រង់ និងតាមទិសដៅជាក់លាក់។
\end{definition}
\begin{remark}
	លក្ខណៈសម្គាល់ទាំងពីរនៃបម្លាស់ទីគឺៈ
	\begin{itemize}
		\item [$-$] \emph{\kml បម្លាស់ទី} គឺជាចម្ងាយចររវាងទីតាំងដើម និងទីតាំងស្រេចរបស់អង្គធាតុ។
		\item [$-$] \emph{\kml បម្លាស់ទី} មានទិសដៅពីទីតាំងដើម​ទៅទីតាំងស្រេចរបស់អង្គធាតុ។
	\end{itemize}
\end{remark}
\subsection{ល្បឿន វុិចទ័រល្បឿន}
\begin{enumerate}[m]
	\item \emph{\kml ល្បឿន}\\
	ល្បឿននៃអង្គធាតុមួយសម្គាល់ភាពលឿន ឬភាពយឺតនៃចលនារបស់អង្គធាតុនោះ ហើយកំណត់ដោយផលធៀបរវាងចម្ងាយចរ និងរយៈពេល។ យើងបានៈ \fbox{$\text{ល្បឿន}=\frac{\text{ចម្ងាយចរ}}{\text{រយៈពេល}}$ ឬ $v=\frac{d}{t}$}\\
	\begin{multicols}{2}
		\begin{itemize}
			\item [$-$] ចម្ងាយចរគិតជាម៉ែត្រ $\left(m\right)$
			\item [$-$] រយៈពេលគិតជាវិនាទី $\left(s\right)$
			\item [$-$] ល្បឿនគិតជាម៉ែត្រក្នុងមួយវិនាទី $\left(m/s\right)$
		\end{itemize}
	\end{multicols}
	ភាកច្រើននៃអង្គធាតុមិនមានចលនាដោយល្បឿនថេរទេ ល្បឿនបស់វាពេលខ្លះយឺត និងពេលខ្លះលឿន។ ហេតុនេះហើយគេត្រូវកំណត់ល្បឿនរបស់អង្គធាតុនោះជាល្បឿនមធ្យមដែលល្បឿននេះដោយផលធៀបរវាងចម្ងាយចរសរុប និងរយៈពេលសរុប។\\
	យើងបានៈ \fbox{$\text{ល្បឿនមធ្យម}=\frac{\text{ចម្ងាយចរសរុប}}{\text{រយៈពេលសរុប}}$ ឬ $\overline{v}=\frac{d}{t}$}
	\item \emph{\kml វុិចទ័រល្បឿន}\\
	វុិចទ័រល្បឿនគឺជាបម្លាស់ទីរបស់វត្ថុក្នុងមួយខ្នាតពេល។\\
	យើងបានៈ \fbox{$\text{វុិចទ័រល្បឿន}=\frac{\text{បម្លាស់ទី(ចម្ងាយត្រង់)}}{\text{រយៈពេលចរ}}$}។
	\begin{multicols}{2}
		ឧបមាថានៅខណៈ $t_{1}$ ចល័តស្ថិតនៅត្រង់ចំណុចមួយដែលមានទីតាំង $x_{1}$ ហើយនៅខណៈ $t_{2}$ ចល័តស្ថិតនៅត្រង់ចំណុចមួយដែលមានទីតាំង $x_{2}$។
		\begin{figure}[H]
			\centering
			\begin{tikzpicture}
				\begin{scope}
					\draw [->, -Stealth] (-1,0) --(5,0);
					\coordinate[label=below:$O$] (0,0) at (0,0);
					\coordinate[label=below:$x_{1}$] (1,0) at (1,0);
					\coordinate[label=below:$x_{2}$] (3,0) at (3,0);
					\coordinate[label=above:$t_{1}$] (1,0) at (1,0);
					\coordinate[label=above:$t_{2}$] (3,0) at (3,0);
					\draw (0,0) node {$\bullet$};
					\draw (1,0) node {$\bullet$};
					\draw (3,0) node {$\bullet$};
					\draw [|-|] (1,1) --(3,1);
					\draw (2,1.5) node {$\Delta x$};
				\end{scope}
			\end{tikzpicture}
		\end{figure}
	\end{multicols}
	យើងបានៈ \fbox{$\text{វុិចទ័រល្បឿនមធ្យម}=\frac{\text{បម្លាស់ទីសរុប}}{\text{រយៈពេលចរ}}$ ឬ $v=\frac{\Delta x}{\Delta t}=\frac{x_{2}-x_{1}}{t_{2}-t_{1}}$}\\
	\quad នៅក្នុងជីវភាពរស់នៅយើងតែងតែប្រើពាក្យល្បឿនតែមួយគត់។ ប៉ូន្តែនៅក្នុងរូបវិទ្យា គេបានញែកពាក្យនេះជាពីរដាច់ចេញពីគ្នាគឺ ល្បឿន និង វុិចទ៏័រល្បឿន។ ល្បឿន ជាចម្ងាយចរក្នុងមួខ្នាតពេល។ ចំណែកឯវុិចទ័រល្បឿន ជាបម្លាស់ទីក្នុងមួយខ្នាតពេល។
	\begin{remark}
		កាលណាគេនិយាយពីវុិចទ័រល្បឿននៃអង្គធាតុមួយគេត្រូវគិតដល់ល្បឿននិងទិសដៅដែលវាបានឆ្លងកាត់។\\
		ក្នុងចលនាត្រង់ស្មើ វុិចទ័រល្បឿននិងវុិចទ័របម្លាស់ទីមានទិស និងទិសដៅដូចគ្នា ដូចនេះគេអាចសរសេរៈ $v=\frac{\Delta x}{\Delta t}$
	\end{remark}
\end{enumerate}
\subsection{វុិចទ័រល្បឿនខណៈ}
\quad \quad វ៉ិចទ័រល្បឿនខណៈ $v_{x}$ ជាល្បឿនត្រង់ ទីតាំងណាមួយ លើគន្លងដែលវាបានផ្លាស់ទី។ យើងបាន $v_{x}=\lim\limits_{\Delta t\to 0}$$\frac{\Delta x}{\Delta t}$(មានន័យថាចន្លោះពេល $\Delta t$ តូទៅៗ ពោលគឺ $\Delta t \to 0$ ដែលនាំឲ្យចន្លោះពេលនេះក្លាយជាមួយទីតាំងតែម្តង) ហើយក្នុងគណិតវិទ្យាលីមីតនេះជាដេរីវេនៃ $x$ ធៀបនឹងពេល $t$ ដែលគេតាងដោយ $\frac{dx}{dt}$។ ដូចនេះ $v_{x}=\lim\limits_{\Delta t\to 0}\frac{\Delta x}{\Delta t}=\frac{dx}{dt}$ វាជាមេគុណប្រាប់ទិសនៃបន្ទាត់ប៉ះលើខ្សែកោងត្រង់ចំណុចមួយនោះ ដែលខ្សែកោងនេះជាសមីការទីតាំង $x\left(t\right)$ នៃចលនា។
\subsection{សំទុះ សំទុះមធ្យម សំទុះខណៈ}
\begin{definition}
	នៅពេលដែលចល័តមួយផ្លាស់ប្តូរល្បឿនធៀបនឹងពេលគេថាអង្គធាតុនោះកំពុងមានសំទុះ។
\end{definition}
\begin{enumerate}[m]
	\item \emph{\kml សំទុះមធ្យម}\\
	\quad ឧបមានៅខណៈ $t_{1}$ ចល័តមួយមានវ៉ិចទ័រល្បឿន $v_{1}$ ហើយនៅខណៈ $t_{2}$ វាមានល្បឿន $v_{2}$។ សំទុះមធ្យម $\overline{a}$ ឬ $a_{av}$ នៅចន្លោះពីរទីតាំងរបស់ចល័តផ្លាស់ទីមួយគឺជាផលធៀបរវាងវ៉ិចទ័រល្បឿន $\Delta v$ និងចន្លោះពេល $\Delta t$។
	\begin{flalign*}
		\text{យើងបានសំទុះមធ្យម}\quad :&\quad \overline{a}=a_{av}=\frac{\Delta v}{\Delta t}=\frac{v_{2}-v_{1}}{t_{2}-t_{1}}
	\end{flalign*}
	\item \emph{\kml សំទុះខណៈ}\\
	\begin{definition}
		សំទុះខណៈ $a$ នៃវត្ថុផ្លាស់ទីគឺជាសំទុះត្រង់ទីតាំងណាមួយក្នុងគន្លងរបស់វា។ មានន័យថាជាលីមីតនៃសំទុះមធ្យមនៅពេលណាដែល $\Delta t$ ខិតទៅរកសូន្យ។
	\end{definition}
	\begin{flalign*}
		\text{យើងបាន}\quad :&\quad a=\lim\limits_{\Delta t\to 0}\frac{\Delta v}{\Delta t}=\frac{dv}{dt}=\frac{d}{dt}\left(\frac{dx}{dt}\right)=\frac{d^{2}x}{dt^{2}}
	\end{flalign*}
	\begin{remark}
		បើបកស្រាយតាមក្រាហ្វិច គេអាចនិយាយថាៈ
		\begin{itemize}
			\item សំទុះមធ្យមគឺជាមេគុណប្រាប់ទិសនៃទីតាំងពីរដែលនៅលើខ្សែកោង $v-t$។
			\item សំទុះខណៈនៃទីតាំងមួយលើគន្លង គឺជាមេគុណប្រាប់ទិសនៃបន្ទាត់ប៉ះទៅនឹងខ្សែកោង $v-t$ ត្រង់ទីតាំងនោះ។
		\end{itemize}
	\end{remark}
\end{enumerate}
\section{ចលនាអង្គធាតុតាមមួយវិមាត្រ}
\quad \quad ចលនារបស់អង្គធាតុមានច្រើនបែបច្រើនយ៉ាងដូចជា ចលនាត្រង់ ចលនាវង់ ចលនាកោង ចលនារង្វិល ចលនារំកិល...។ ប៉ុន្តែក្នុងមេរៀននេះយើងលើកយកតែចលនាត្រង់ មកសិក្សាប៉ុណ្ណោះ។
\subsection{ចលនាត្រង់ស្មើ}
\begin{definition}
	ចលនាត្រង់ស្មើជាចលនារបស់អង្គធាតុមួយដែលមានគន្លងជាបន្ទាត់ត្រង់ និងមានល្បឿនថេរ។
\end{definition}
\begin{example}
	បើរថយន្តមួយចរបានចម្ងាយ $10m$ រៀងរាល់មួយវិនាទី គេអាចនិយាយថា រថយន្តនោះចរដោយល្បឿនថេរគឺ $10m/s$។
\end{example}
\begin{multicols}{3}
	\begin{itemize}
		\item \emph{\kml សមីការចលនា} $x=vt+x_{0}$
		\item \emph{\kml ល្បឿន} $v=\frac{x-x_{0}}{t}$ ថេរ
		\item \emph{\kml សំទុះ} $a=0$(ព្រោះល្បឿនថេរ)
	\end{itemize}
\end{multicols}
\begin{remark}
	នៅលើកុងទ័រល្បឿន ល្បឿនគិតជាគឺឡូម៉ែត្រក្នុងមួយម៉ោង $\left(km/h\right)$ ឬម៉ាយក្នុងមួយម៉ោង $\left(MPH\right)$  ឬ​$\left(mi/h\right)$។
	\begin{multicols}{2}
		\begin{itemize}
			\item $1$ ម៉ាយ $=1.61$ គីម៉ែត្រ ឬ $1mi=1.61km$
			\item $1$ នាទី $=60$ វិនាទី ឬ $1mn=60s$
			\item $1$ ម៉ោង $=60$ នាទី ឬ $1h=60nm$
			\item $1$ ម៉ោង $=3600$ វិនាទី ឬ $1h=3600s$
		\end{itemize}
	\end{multicols}
\end{remark}
\begin{example}
	ចល័តមួយផ្លាស់ទីដោយចលនាត្រង់ស្មើនៅខណៈ $t=0$ វាចរបានចម្ងាយ $5m$ រយៈពេល $5s$ ក្រោយមកវាចរបានចម្ងាយ $50m$។ គណនាល្បឿនរបស់ចល័ត។
\end{example}
\subsection{ចលនាត្រង់ប្រែប្រួលស្មើ}
\begin{definition}
	ចលនាត្រង់ប្រែប្រួលស្មើ ជាចលនាដែលល្បឿនកើនឡើងឬថយចុះដោយតម្លៃស្មើគ្នាក្នុងរយៈពេលស្មើគ្នា។\\ គេបានបែងចែកប្រភេទនៃចលនាត្រង់ប្រែប្រួលស្មើជាពីរគឺ ចលនាស្ទុះស្មើ និងចលនាយឺតស្មើ។
\end{definition}
\newpage
\begin{enumerate}
	\item \emph{\kml ចលនាស្ទុះស្មើ}
	\begin{definition}
		ចលនាមួយ ជាចលនាស្ទុះស្មើ កាលណាល្បឿនរបស់វាកើនឡើងដោយតម្លៃស្មើៗគ្នា ក្នុងរយៈពេលដូចគ្នាៗ។\\ ក្នុងចលនាស្ទុះស្មើសំទុះ $a>0$ និងល្បឿនខណៈធំជាងល្បឿនដើម $v>v_{0}$។
	\end{definition}
	\item \emph{\kml ចលនាយឺតស្មើ}
	\begin{definition}
		ចលនាមួយ ជាចលនាយឺតស្មើ កាលណាល្បឿនរបស់វាថយចុះដោយតម្លៃស្មើៗគ្នា ក្នុងរយៈពេលដូចគ្នាៗ។\\ ក្នុងចលនាយឺតស្មើសំទុះ $a<0$ និងល្បឿនខណៈតូចជាងល្បឿនដើម $v<v_{0}$។
	\end{definition}
\end{enumerate}
	\begin{multicols}{2}
		\begin{itemize}
			\item \emph{\kml សមីការចលនា ឬសមីការអាប់ស៊ីស} $x=\frac{1}{2}at^{2}+v_{0}t+x_{0}$
			\item \emph{\kml ល្បឿន} $v=v_{0}+at$ និង $v_{av}=\frac{v+v_{0}}{2}$
			\item \emph{\kml សំទុះ} $a=\frac{v-v_{0}}{t}$(ថេរ)
			\item \emph{\kml ទំនាក់ទំនងគ្មានពេល} $v^{2}-v^{2}_{0}=2a\left(x-x_{0}\right)$
		\end{itemize}
	\end{multicols}
	\begin{exercise}
		\begin{enumerate}
			\item រថយន្តមួយកំពុងបើកលើផ្លូវត្រង់ដោយចលនាត្រង់ស្មើ ដោយល្បឿន $v_{0}=20m/s$។ អ្នកបើកបានឃើញស្ពាននៅខាងមុខ $53m$ បាក់ភ្លាមៗនោះគាត់ក៏ជាន់ប្រាំង រថយន្តក៏បន្តដំណើរដោយចលនាយឺតស្មើដែលមានសំទុះ\\ $a=-4m/s^{2}$។ តើរថយន្តនោះធ្លាក់ស្ពាន ឬទេ?
			\item ទូកមួយឆ្លងកាត់ពីមាត់ច្រាំងម្ខាងទៅមាត់ច្រាំងម្ខាងទៀតដោយល្បឿន $5m/s$ ហើយទូកនោះត្រូវចរន្តទឹកហូរដោយល្បឿន $2m/s$។ ចូរតាងវ៉ិចទ័រល្បឿននៃបម្លាស់ទីរបស់ទូកតាមមាត្រដ្ឋាន $1cm$ ត្រូវនឹង $1m/s$។
			\item ចល័តមួយធ្វើចលនាដោយល្បឿនប្រែប្រួលពី $20m/s$ ទៅ $40m/s$ ដោយប្រើរយៈពេល $2mn$។ \\គណនាល្បឿនមធ្យម និងបម្លាស់ទីរបស់ចល័តក្នុងរយៈពេល $2mn$។
			\item រថយន្តមួយបើកឡើងភ្នំដោយល្បឿនមធ្យម $10m/s$។ រថយន្តមានចលនាយឺតស្មើ។\\ គណនាល្បឿនដើមដោយដឹងថាល្បឿនស្រេចគឺ $v=5m/s$។
		\end{enumerate}
	\end{exercise}
\newpage
