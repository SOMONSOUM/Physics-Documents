\chapter{ចលនាអង្គធាតុតាមមួយវិមាត្រ}
\section{ចលនាមេកានិច}
\begin{definition}
	\begin{itemize}
		\item បម្លាស់ប្តូរទីតាំងអង្គធាតុមួយធៀបនឹងអង្គធាតុមួយទៀត ហៅថាចលនាមេកានិច។
		\item ចំពោះអង្គធាតុណាមួយដែលកំណត់ចលនានៃអង្គធាតុផ្សេងទៀតធៀបនឹងវា គេហៅថាអង្គធាតុនោះថា តម្រុយ។
	\end{itemize}
\end{definition}
\section{បម្លាស់ទី ល្បឿន វុិចទ័រល្បឿន}
\subsection{ចម្ងាយចរ និងបម្លាស់ទី}
\begin{definition}
	\emph{\kml ចម្ងាយចរៈ} ជាប្រវែងសរុបនៃចលនារបស់អង្គធាតុដោយមិនគិតពីទិសដៅនៃចលនា។\\
	\emph{\kml បម្លាស់ទីៈ} ជាចម្ងាយចរដែលវាស់តាមខ្សែត្រង់ និងតាមទិសដៅជាក់លាក់។
\end{definition}
\begin{remark}
	លក្ខណៈសម្គាល់ទាំងពីរនៃបម្លាស់ទីគឺៈ
	\begin{itemize}
		\item [$-$] \emph{បម្លាស់ទី} គឺជាចម្ងាយចររវាងទីតាំងដើម និងទីតាំងស្រេចរបស់អង្គធាតុ។
		\item [$-$] \emph{បម្លាស់ទី} មានទិសដៅពីទីតាំងដើម​ទៅទីតាំងស្រេចរបស់អង្គធាតុ។
	\end{itemize}
\end{remark}
\subsection{ល្បឿន វុិចទ័រល្បឿន}
\begin{enumerate}[m]
	\item \emph{\kml ល្បឿន}\\
	ល្បឿននៃអង្គធាតុមួយសម្គាល់ភាពលឿន ឬភាពយឺតនៃចលនារបស់អង្គធាតុនោះ ហើយកំណត់ដោយផលធៀបរវាងចម្ងាយចរ និងរយៈពេល។ យើងបានៈ \fbox{$\text{ល្បឿន}=\frac{\text{ចម្ងាយចរ}}{\text{រយៈពេល}}$ ឬ $v=\frac{d}{t}$}\\
	\begin{multicols}{2}
		\begin{itemize}
			\item [$-$] ចម្ងាយចរគិតជាម៉ែត្រ $\left(m\right)$
			\item [$-$] រយៈពេលគិតជាវិនាទី $\left(s\right)$
			\item [$-$] ល្បឿនគិតជាម៉ែត្រក្នុងមួយវិនាទី $\left(m/s\right)$
		\end{itemize}
	\end{multicols}
	ភាកច្រើននៃអង្គធាតុមិនមានចលនាដោយល្បឿនថេរទេ ល្បឿនបស់វាពេលខ្លះយឺត និងពេលខ្លះលឿន។ ហេតុនេះហើយគេត្រូវកំណត់ល្បឿនរបស់អង្គធាតុនោះជាល្បឿនមធ្យមដែលល្បឿននេះដោយផលធៀបរវាងចម្ងាយចរសរុប និងរយៈពេលសរុប។\\
	យើងបានៈ \fbox{$\text{ល្បឿនមធ្យម}=\frac{\text{ចម្ងាយចរសរុប}}{\text{រយៈពេលសរុប}}$ ឬ $\overline{v}=\frac{d}{t}$}
	\item \emph{\kml វុិចទ័រល្បឿន}\\
	វុិចទ័រល្បឿនគឺជាបម្លាស់ទីរបស់វត្ថុក្នុងមួយខ្នាតពេល។\\
	យើងបានៈ \fbox{$\text{វុិចទ័រល្បឿន}=\frac{\text{បម្លាស់ទី(ចម្ងាយត្រង់)}}{\text{រយៈពេលចរ}}$}។
	\begin{multicols}{2}
		ឧបមាថានៅខណៈ $t_{1}$ ចល័តស្ថិតនៅត្រង់ចំណុចមួយដែលមានទីតាំង $x_{1}$ ហើយនៅខណៈ $t_{2}$ ចល័តស្ថិតនៅត្រង់ចំណុចមួយដែលមានទីតាំង $x_{2}$។
		\begin{figure}[H]
			\centering
			\begin{tikzpicture}
				\begin{scope}
					\draw [->, -Stealth] (-1,0) --(5,0);
					\coordinate[label=below:$O$] (0,0) at (0,0);
					\coordinate[label=below:$x_{1}$] (1,0) at (1,0);
					\coordinate[label=below:$x_{2}$] (3,0) at (3,0);
					\coordinate[label=above:$t_{1}$] (1,0) at (1,0);
					\coordinate[label=above:$t_{2}$] (3,0) at (3,0);
					\draw (0,0) node {$\bullet$};
					\draw (1,0) node {$\bullet$};
					\draw (3,0) node {$\bullet$};
					\draw [|-|] (1,1) --(3,1);
					\draw (2,1.5) node {$\Delta x$};
				\end{scope}
			\end{tikzpicture}
		\end{figure}
	\end{multicols}
	យើងបានៈ \fbox{$\text{វុិចទ័រល្បឿនមធ្យម}=\frac{\text{បម្លាស់ទីសរុប}}{\text{រយៈពេលចរ}}$ ឬ $v=\frac{\Delta x}{\Delta t}=\frac{x_{2}-x_{1}}{t_{2}-t_{1}}$}\\
	\quad នៅក្នុងជីវភាពរស់នៅយើងតែងតែប្រើពាក្យល្បឿនតែមួយគត់។ ប៉ូន្តែនៅក្នុងរូបវិទ្យា គេបានញែកពាក្យនេះជាពីរដាច់ចេញពីគ្នាគឺ ល្បឿន និង វុិចទ៏័រល្បឿន។ ល្បឿន ជាចម្ងាយចរក្នុងមួខ្នាតពេល។ ចំណែកឯវុិចទ័រល្បឿន ជាបម្លាស់ទីក្នុងមួយខ្នាតពេល។
	\begin{remark}
		កាលណាគេនិយាយពីវុិចទ័រល្បឿននៃអង្គធាតុមួយគេត្រូវគិតដល់ល្បឿននិងទិសដៅដែលវាបានឆ្លងកាត់។\\
		ក្នុងចលនាត្រង់ស្មើ វុិចទ័រល្បឿននិងវុិចទ័របម្លាស់ទីមានទិស និងទិសដៅដូចគ្នា ដូចនេះគេអាចសរសេរៈ $v=\frac{\Delta x}{\Delta t}$
	\end{remark}
\end{enumerate}
\subsection{វុិចទ័រល្បឿនខណៈ}

