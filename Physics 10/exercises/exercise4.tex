\section{សំណួរ លំហាត់អនុវត្តន៍ និងកិច្ចការផ្ទះ}
\begin{enumerate}
	\item ស្រ្តីម្នាក់ទូលល្អីមួយមានចេកពេញ។ គាត់ធ្វើដំណើរសំដៅទៅផ្សារដើម្បីលក់ចេក។ តើកម្លាំងដេលគាត់ទូលល្អីនោះបានបំពេញកម្មន្តដែរ ឬទេ? ព្រោះអ្វី?
	\item ស្រ្តីម្នាក់រុញកូនរទេះមួយនៅក្នុងផ្សារទំនើប។ បើគាត់បញ្ចេញកម្លាំង $30.0\si{\newton}$ ទៅលើកូនរទេះក្នុងទិសដៅនៃបម្លាស់ទីនោះកូនរទេះផ្លាស់ទីបានចម្ងាយ $5.00\si{\metre}$។ គណនាកម្មន្តដែលបានបំពេញដោយកម្លាំងលើកូនរទេះ។
	\item ដើម្បីឲ្យវត្ថុមួយមានម៉ាស $m$ ផ្លាស់ទីពី $A$ ទៅ $B$ ដែល $AB=5\si{\metre}$ គេត្រូវប្រើកម្លាំង $F=20\si{\newton}$។
	\begin{enumerate}
		\item ចូរគូសក្រាបតាង(កម្លាំង-បម្លាស់ទី)។
		\item តាមក្រាបនេះ ចូរគណនាក្រឡាផ្ទៃរបស់ចតុកោណដែលមានជ្រុងស្មើនឹង $F$ និង $AB$។
		\item គណនាកម្មន្នដែលបានធ្វើនោះ។
	\end{enumerate} 
	\begin{figure}[H]
		\centering
		\begin{tikzpicture}
			\coordinate[label=below:$A$] (A) at (1.5,-.5);
			\coordinate[label=below:$B$] (B) at (5.5,-.5);
			\coordinate[label=above:{$\vec{F}$}] (F) at (2.5,.5);
			\fill[pattern=bricks, pattern color=black, preaction={fill=lightpink}] (0, 0) rectangle (7, -.3);
			\filldraw[gray!40!black,fill=cadetgrey] (.5,0) rectangle (1.5,1);
			\filldraw[dashed, gray!40!black,fill=cadetgrey] (5.5,0) rectangle (6.5,1);
			\draw[->, line width=2pt] (1.5,.5)--(F);
			\draw[->, line width=1.5pt] (A)--(B);
			\node at (1,.5) {$m$};
		\end{tikzpicture}
	\end{figure}
\end{enumerate}