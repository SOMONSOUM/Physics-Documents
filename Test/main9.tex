\documentclass[11pt,a4paper]{article}
\usepackage{tabu,tikz}
\usepackage[margin=2cm]{geometry}
\usepackage{fontspec,lmodern}
\setmainfont{Ang DaunTep}
\usepackage{amssymb,amsmath,amsfonts,enumerate}
\usepackage{xlop,polynom,mathrsfs,cancel,multicol}
\everymath{\displaystyle}
\newcommand{\KM}{\fontspec[Script=Khmer]{Khmer M1}}
\newcommand{\ML}{\fontspec[Script=Khmer]
{Khmer M2}}
\newcommand{\en}{\fontspec[Script=Khmer]{Times New Roman}}
\usepackage{polyglossia}%ត្រូវដាក់លើ setmainlanguage
\setmainlanguage[numerals=khmer]{khmer}
\usepackage{graphicx}
\usepackage{fancyhdr}
\pagestyle{fancy}
\cfoot{​\thepage}
\lfoot{ក្រុមពិភាក្សាទី១}
\rfoot{​គណិតវិទ្យាក្រុម ២}
\chead{របៀបសរសេរកិច្ចតែងការ}
\renewcommand{\footrulewidth}{1pt}
\begin{document}
{\KM\large\noindent \begin{flushleft}
វិទ្យាស្ថានជាតិអប់រំ
\end{flushleft}\begin{flushright}
ព្រះរាជាណាចក្រកម្ពុជា\\[5pt]
ជាតិ សាសនា ព្រះមហាក្សត្រ
\end{flushright}}
\begin{center}
{\KM\LARGE កិច្ចតែងការបង្រៀន}
\end{center}
\begin{tabbing}
\hspace{3.3cm}\=\hspace{1cm}\=\kill
\textbullet\quad កាលបរិច្ឆេទ \> ៖ \> ​ថ្ងៃ អង្គារ ទី ២០ ខែ វិឆ្ឆិកា ឆ្នាំ ២០១៨\\[4pt]
\textbullet\quad មុខវិជ្ជា \> ៖ \> គណិតវិទ្យា\\[4pt]
\textbullet\quad ថ្នាក់ទី \> ៖ \> ថ្នាក់ទី១០ ~«~ក~»\\[4pt]
\textbullet\quad ជំពូកទី ៣ \> ៖ \> សមីការ និង វិសមីការ\\[4pt]
\textbullet\quad មេរៀនទី១ \> ៖ \> ​សមីការដឺក្រេទី២ មានមួយអញ្ញាត\\[4pt]
\> ​ \> ​$ \en{1.1} $~~ ចំនួនកុំផ្លិច\\[4pt]
\> ​ \> ​~~ក.~ ឯកតានិម្មិត \\[4pt]
\> ​ \> ​​~~ខ.~ ឬសការេនៃចំនួនអវិជ្ជមាន\\[4pt]
\> ​ \> ​​~~គ.~ និយមន័យនៃចំនួនកុំផ្លិច\\[4pt]
\textbullet\quad រយៈពេល \> ៖ \> ៥០​ នាទី
\end{tabbing}
\begin{enumerate}
\item [$ \mathrm{I.} $]{\KM វត្ថុបំណងមេរៀន}\quad ក្រោយពីសិក្សាមេរៀននេះចប់សិស្សនឹងអាច
\begin{tabbing}
\hspace{3.4cm}\=\hspace{0.68cm}\=\kill
\textbullet\quad វិជ្ជាសម្បទា\>៖ \> ពន្យល់បានពីឯកតានិម្មិត និយមន័យចំនួនកុំផ្លិច និង ឬសការេនៃចំនួន\\[4pt]
​\> \>អវិជ្ជមានបានត្រឹមត្រូវតាមរយៈឧទាហរណ៍ \\[4pt]
\textbullet\quad បំណិនសម្បទា\>៖ \> បញ្ចេញ និងគណនាការេនៃចំនួនអវិជ្ជមានពីរ៉ាឌីកាល់បានត្រឹមត្រូវ\\[4pt]
\> ​ \> ត្រូវតាមរយៈលំហាត់គំរូ\\[4pt]
\textbullet\quad ចរិយាសម្បទា\> ៖ \> បណ្តុះស្មារតីអោយចូលចិត្តសិក្សាគណិតវិទ្យាតាម​លក្ខណៈជាក្រុម ឬជាបុគ្គល
\end{tabbing}
\item [$ \mathrm{II.} $]{\KM សម្ភារៈឧបទេស}\quad
\begin{enumerate}
\item [$ - $] សៀវភៅសិក្សាថ្នាក់ទី~១០\quad (~ទំព័រ ~$ 91-94 $~)
\item [$ - $] សៀវភៅគ្រូទំព័រ ~~(~$ 92-94 $~)~~
\item [$ - $]​ប័ណ្ណសំណួរ
\end{enumerate}
\item[$ \mathrm{III.} $] {\KM វិធីសាស្រ្តបង្រៀន}\\[5pt]
$ -\quad $ តាមវិធីសាស្រ្តចម្រុះ
\item[$ \mathrm{IV.} $] {\KM វិធីសាស្រ្តបង្រៀន}
\end{enumerate}
\noindent
\begin{tabu} to\linewidth{|X[1,l]|X[1,l]|X[1,l]|} \hline
\centering {\KM សកម្មភាពគ្រូ }& \centering {\KM ខ្លឹមសារមេរៀន} &\centering {\KM សកម្មភាពសិស្ស } \\ \hline
&\vspace{-1mm}\centering {\ML ជំហានទី១ }&\\
&\vspace{-2mm}\centering (~រដ្ឋបាលថ្នាក់~: ~៣ នាទី)&\\
bត្រូវដាក់លំហាត់អោយសិស្សធ្វើកិច្ចការផ្ទះ& សរសេរកិច្ចការផ្ទះ
\begin{center}
\begin{tikzpicture}
\draw (0,0)--(2,0)--(0,3)--cycle;
\end{tikzpicture}
\end{center}
&b ត្រូវធ្វើកិច្ចការផ្ទះ\\
&\centering {\ML ជំហានទី២ }​&\\
&\vspace{-2mm}\centering (~រំលឹកមេរៀនចាស់~: ~៧ នាទី)&\\
bត្រូវដាក់លំហាត់អោយសិស្សធ្វើកិច្ចការផ្ទះ& សរសេរកិច្ចការផ្ទះ
\begin{center}
\begin{tikzpicture}
\draw (0,0)--(2,0)--(0,3)--cycle;
\end{tikzpicture}
\end{center}
&ត្រូវធ្វើកិច្ចការផ្ទះ\\
&\centering {\ML ជំហានទី៣ }​&\\
&\vspace{-2mm}\centering (~មេរៀនប្រចាំថ្ងៃ~: ~៣០នាទី)&\\
ត្រូវដាក់លំហាត់អោយសិស្សធ្វើកិច្ចការផ្ទះ & សរសេរកិច្ចការផ្ទះ
\begin{center}
\begin{tikzpicture}
\draw (0,0)--(2,0)--(0,3)--cycle;
\end{tikzpicture}
\end{center}
& ត្រូវធ្វើកិច្ចការផ្ទះ\\
&\centering {\ML ជំហានទី៤ }​&\\
&\vspace{-2mm}\centering (~ពង្រឹងពុទ្ធិ~:~៥ នាទី)&\\
ត្រូវដាក់លំហាត់អោយសិស្សធ្វើកិច្ចការផ្ទះ& សរសេរកិច្ចការផ្ទះ
\begin{center}
\begin{tikzpicture}
\draw (0,0)--(2,0)--(0,3)--cycle;
\end{tikzpicture}
\end{center}
& ត្រូវធ្វើកិច្ចការផ្ទះ\\\hline
\end{tabu}
\newpage
\noindent
\begin{tabu} to\linewidth{|X[1,l]|X[1,l]|X[1,l]|} \hline
&\centering {\ML ជំហានទី៥ }​&\\
&\vspace{-2mm}\centering (~បណ្តាំផ្ញើរ~:~៥ នាទី)&\\
ត្រូវដាក់លំហាត់អោយសិស្សធ្វើកិច្ចការផ្ទះ& សរសេរកិច្ចការផ្ទះ
\begin{center}
\begin{tikzpicture}
\draw (0,0)--(2,0)--(0,3)--cycle;
\end{tikzpicture}
\end{center}
\begin{enumerate}
\item
\item
\item
\end{enumerate}
&ត្រូវធ្វើកិច្ចការផ្ទះ\\\hline
\end{tabu}
\end{document}