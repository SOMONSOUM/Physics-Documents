\documentclass[12pt, a4paper]{classes/exam}
\usepackage{array}
\usepackage{tikz}
\usepackage{circuitikz}
\usepackage{siunitx}
\usepackage{array}
\usepackage{multicol}
\usepackage{multirow}
\usepackage{graphicx}
\usepackage{physics}
\graphicspath{ {./images/} }
\usepackage{wrapfig}
\usepackage[table]{xcolor}
\usepackage{multicol}
\usepackage{pdflscape}
\usepackage{adjustbox}
\usepackage{pdfpages}
\usepackage{tkz-euclide}
\definecolor{myyellow}{RGB}{254,241,24}
\definecolor{myorange}{RGB}{234,125,1}
\definecolor{fancyorange1}{RGB}{253,138,9}
\definecolor{fancyorange2}{RGB}{246,156,123}
\usepackage{tkz-euclide}
\usetikzlibrary{quotes,arrows.meta}
\usepackage{makecell}
\begin{document}
	\begin{flushright}
		{\kml ព្រះរាជាណាចក្រកម្ពុជា~~~~~\\
			ជាតិ សាសនា ព្រះមហាក្សត្រ}
	\end{flushright}
	\begin{flushleft}
		{\kbk មន្ទីអប់រំ យុវជននិងកីឡារាជធានភ្នំពេញ\\
			~~~~~~~~~~សាលាមេតូឌីស្ទកម្ពុជា}
	\end{flushleft}
	\begin{changemargin}{1cm}{1cm}
		\begin{center}
		\fontsize{16.8}{20.16}\selectfont\kml កិច្ច​តែង​ការ​បង្រៀន{(\en Lesson Plan)}
		\end{center}
		\begin{tabbing}
		\hspace{3cm}	\= $\blacktriangleright$ {\km កាលបរិច្ឆេទ}		\= $:$ ថ្ងៃទី ០២ ខែ មករា ឆ្នាំ ២០២០\\
						\> $\blacktriangleright$ {\km មុខ​វិជ្ជា} 		   \> $:$ រូបវិទ្យាថ្នាក់ទី១២\\
						\> $\blacktriangleright$ {\km ថ្នាក់ទី} 		  \> $:$ {\kml ​១២ក}\\				
						\> $\blacktriangleright$ {\km ជំ​ពូក​១}		   \> $:$ ទែម៉ូឌីណាមិច{\en(Thermodynamics)}\\
						\> $\blacktriangleright$ {\km មេ​រៀន​ទី៣}		  \> $:$ ម៉ាស៊ីន{\en(Heat Engine)}\\
						\> $\blacktriangleright$ {\km រយៈ​ពេល}		   \> $:$ ១ម៉ោង ៤០នាទី\\
						\> $\blacktriangleright$ {\km លោកគ្រូ}		   \> $:$ {\kml ស៊ុំ សំអុន}
		\end{tabbing}
		\begin{enumerate}[m]
		\item {\kml វត្ថុ​បណង៖ }
			\begin{description}[leftmargin=2cm, style=sameline]
			\item [{\km វិជ្ជាសម្បទា}] 	$:$ សិស្សតម្រូវឲ្យឡើងធ្វើបទបង្ហាញជាក្រុម ដើម្បីពន្យល់មិត្តរួមថ្នាក់៖
			\begin{itemize}
				\item បកស្រាយអំពីដ្យាក្រាមនៃថាមពលរបស់ម៉ាស៊ីនកម្តៅ(មានរូបភាពបញ្ជាក់)។
				\item បកស្រាយអំពីប្រសិទ្ធភាពនៃម៉ាស៊ីនកម្តៅ(សរសេររូបមន្ត និងអនុវត្តន៍លំហាត់គំរូ)។
				\item និយាយអំពីរបបគំហើញរបស់លោក សាឌី កាណូ និងបកស្រាយអំពីសិុចកាណូ។
				\item បកស្រាយអំពីទ្រឹស្តីបទកាណូ និងរូបមន្តទិន្នផល ឬប្រសិទ្ធភាពអតិបរមានៃម៉ាស៊ីន កាណូ និងម៉ាស៊ីនអ៊ីដេអាល់ រួចអនុវត្តន៍លំហាត់គំរូ។
				\item បកស្រាយអំពីនិយមន័យនៃម៉ាស៊ីនកម្តៅជាមួយ ដោយលើកឧទាហរណ៍ អំពីម៉ាស៊ីនម៉ាស៊ូត និងម៉ាស៊ីនសាំង។
				\item និយាយអំពីភាពខុសគ្នារវាងម៉ាស៊ីនចំហេះក្នុង និងម៉ាស៊ីនចំហេះក្រៅ(យកម៉ាស៊ីនចំហេះក្រៅមកបង្ហាញ និងបង្ហាញអំពីដំណើរការនៃម៉ាស៊ីនចំហេះក្នុងតាមរយៈការបញ្ចាំងស្លាយ)។
				\item បកស្រាយអំពីប្រភេទម៉ូទ័របន្ទុះ៤ វគ្គ និងម៉ូទ័របន្ទុះ២ វគ្គ(បង្ហាញដោយប្រើ {\en Projector} និងរូបភាពនៃដំណើរការរបស់ម៉ាស៊ីន)។
			\end{itemize} 
			\item [{\km បំណិនសម្បទា}] $:$ 
			\begin{itemize}
				\item សិស្សនឹងពន្យល់អំពីទ្រឹស្តីមួយចំនួនក្នុងមេរៀនបានត្រឹមត្រូវ រួចយកទ្រឹស្តី និងរូបមន្តទាំងនោះមកសិក្សា និងពិភាក្សាជាក្រុមដើម្បីដោះស្រាយលំហាត់ឲ្យបានត្រឹមត្រូវ។
				\item សិស្សនឹងចេះវិភាគ អំពីបាតុភូតក្នុងការដោះស្រាយលំហាត់។
			\end{itemize}
			\newpage
			\item [{\km ចរិយាសម្បទា}] $:$ ក្រោយបញ្ចប់មេរៀននេះសិស្សនឹងទទួលបាន៖ 
			\begin{itemize}
				\item សិស្សនឹងយល់ដឹងបន្ថែមអំពីសារសំខាន់នៃថាមពលកម្តៅ និងបម្រើបម្រាស់របស់វាក្នុងការរស់នៅរបស់យើងសព្វថ្ងៃ។
				\item ចេះចង់ដឹងចង់យល់បន្ថែមអំពីវិទ្សាសាស្រ្ត។
				\item ចេះសហការ និងធ្វើការជាក្រុម ដោយមានស្មារតីទទួលខុសត្រូវ។
				\item ដឹងអំពីរបៀបស្វែងរកដំណោះស្រាយនៃបញ្ហាជាក្រុម។
			\end{itemize}
			\end{description}
		\item​ {\kml ឯកសារយោង}
		\begin{itemize}
			\item [$-$] សៀវ​ភៅ​រូបវិទ្យាថ្នាក់ទី១២  បោះពុម្ភ ដោយក្រសួងអប់រំយុវជន និងកីឡា ទំព័រទី 26-39។
			\item [$-$] រូបវិទ្យាថ្នាក់ទី១២ មេរៀនសង្ខេប និងលំហាត់ {\kml រៀបរៀងដោយ ស៊ុំ សំអុន}(មេរៀនម៉ាស៊ីន)។
			\item [$-$] ឯកសារមួយចំនួនទៀតដែលត្រូវស្រាវជ្រាវតាមអ៊ីធើណែត
		\end{itemize}
		\item {\kml សម្ភារៈឧបទេស}
		\begin{itemize}
			\item ម៉ាស៊ីនចំហេះក្រៅគំរូចំនួនមួយ
			\item កុំព្យួទ័រ និង {\en Projector} ចំនួន១
		\end{itemize}
		\end{enumerate}
	\end{changemargin}
\fontsize{10}{12}\selectfont
\begin{longtable}{|p{0.31\textwidth}|p{0.31\textwidth}|p{0.31\textwidth}|}
	\hline
	\thead{\fontsize{12}{14.4}\selectfont\kml សកម្ម​ភាព​គ្រូ}	
	&\thead{\fontsize{12}{14.4}\selectfont\kml ខ្លឹម​សារ​មេ​រៀន}		
	&\thead{\fontsize{12}{14.4}\selectfont\kml សកម្ម​ភាព​សិស្ស}\\
\hline
\thead{\kml ត្រួតពិនិត្យ}& \thead{{\kml ជំហាន់ទី១​\kbk (៥នាទី)}} & ប្រធានថ្នាក់ ឬអនុប្រធានថ្នាក់ជួយសម្របសម្រួល។\\
សណ្ដាប់ធ្នាប់ក្នុងថ្នាក់ & \centering{\kml រដ្ឋបាលក្នុងថ្នាក់}​ &​ឡើងរាយការណ៍អំពីអវត្តមានសិស្ស។\\
អវត្តមាន	&  &\\
អនាម័យក្នុងថ្នាក់	&  &\\
\hline
 & \thead{{\kml ជំហាន់ទី២​\kbk (១០នាទី)}} & សិស្សឆ្លើយសំណួរ\\
 - ដូចម្តេចដែលហៅថាម៉ូទ័រកម្តៅ? &\centering{\kml រំលឹកមេររៀនចាស់} & - ម៉ូទ័រកម្តៅជាម៉ូទ័រ ឬឧបករណ៍ដែលបម្លែងថាមពលកម្តៅជាកម្មន្ត។\\
 - ដូចម្តេចដែលហៅថាលំនាំអាដ្យាបាទិច? និយាយពីលក្ខណៈនៃបម្លែងនេះ។ & & 
 - លំនាំអាដ្យាបាទិចជាលំនាំដែលប្រព័ន្ធមិនមានបណ្តូរថាមពលកម្តៅជាមួយមជ្ឃដ្ឋានក្រៅ។
 $Q=0$ នោះ $W=-\Delta U$ ជាលក្ខណៈនៃប្រព័ន្ធត្រមោច​ ដែលត្រូវបានហ៊ុំព័ទដោយអ៊ីសូឡង់កម្តៅ។\\
 \hline
 & \thead{{\kml ជំហាន់ទី៣​\kbk (១០-១៥នាទី)}} & សិស្សមកអង្គុយតាមក្រុម\\
 ចែកបញ្ជីឈ្មោះ និងចំណុចដែលសិស្សត្រូវឡើងធ្វើបទបង្ហាញ។ & \centering{\kml ពិភាក្សាក្រុម } & ប្រធានក្រុមត្រូវដឹកនាំក្រុម និងចែកកិច្ចការឲ្យកូនក្រុម។
 						សមាជិកក្រុមត្រូវជួយពន្យល់គ្នា និងរកដំណោះស្រាយទាំងអស់គ្នា។
\\
\hline
\newpage
\hline
& \thead{{\kml ជំហាន់ទី៤​\kbk (៥០នាទី)}} & សិស្សតាមក្រុមឡើងធ្វើបទបង្ហាញ\\
រៀបចំសម្របសម្រួល និងបន្ថែមចំណុចខ្វះខាត់របស់សិស្ស & \centering{\kml ធ្វើបទបង្ហាញ} & ក្រុមនីមួយៗមានរយៈពេល ១៥ នាទី ដើម្បីពន្យល់+ឆ្លើយសំណួរ+ធ្វើលំហាត់គំរូ។ គ្រប់ក្រុមនីមួយៗអាចរៀបចំបទបង្ហាញដោយប្រើ {\en Projector} និងសម្ភារៈពិសោធន៍ ដើម្បីពន្យល់មិត្តរួមថ្នាក់។ ក្រោយពេលចប់បទបង្ហាញនីមួយៗ ត្រូវមានសំណួរដើម្បីសួរទៅមិត្តរួមថ្នាក់ដែលបានស្តាប់ចំណុចដែលក្រុមខ្លួនបានធ្វើបទបង្ហាញ ក៏ដូចគ្នាផងដែរ អ្នកស្តាប់ក៏អាចសួរសំនួរបានដូចគ្នា។ \\
\hline
& \thead{{\kml ជំហាន់ទី៥​\kbk (១៥នាទី)}} & \\
ពន្យល់បន្ថែម និងប្រាប់គន្លឹះសំខាន់ៗដើម្បីដោះស្រាយលំហាត់ & \centering{\kml ពង្រឹងពុទ្ធិ} & 
- សិស្សស្តាប់ការពន្យល់ និងបន្ថែមចំណុចសំខាន់ៗក្នុងការដោះស្រាយលំហាត់។\\
\hline		
& \thead{{\kml ជំហាន់ទី៦​\kbk (៥នាទី)}} & \\
ប្រាប់សិស្សអំពីលេខរៀងលំហាត់ និងសំណួរដែលត្រូវកិច្ចការផ្ទះ & \centering{\kml កិច្ចការផ្ទះ} & 
កត់ត្រាលេខរៀងលំហាត់ និងសំណួរកិច្ចការផ្ទះ\\
\hline					
\end{longtable}
\begin{flushright}
	{ភ្នំពេញ,ថ្ងៃទី០២ ខែ មករា ឆ្នាំ២០២០\\
		រៀបរៀងដោយ~~~~~~~~~~~~}
\end{flushright}
\begin{flushleft}
	{~~~~~~បានឃើញ និងឯកភាព\\
		~~~~~~~~~~នាយកវិទ្យាល័យ}
\end{flushleft}
\end{document}
