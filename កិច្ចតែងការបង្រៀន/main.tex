\documentclass{classes/exam} 
\usepackage{chemfig}
\usepackage{tabu,tikz}
\usepackage{physics}
\usepackage{circuitikz}
\usepackage{graphicx}
\graphicspath{ {./images/} }
\usepackage[version=4]{mhchem}
\usepackage{tkz-euclide}
\usepackage{tkz-euclide}
\usetkzobj{all}
\tikzstyle arrowstyle=[scale=1]
\tikzstyle directed=[postaction={decorate,decoration={markings,
		mark=at position .65 with {\arrow[arrowstyle]{stealth}}}}]
\tikzstyle direct=[postaction={decorate,decoration={markings,
		mark=at position .65 with {\arrow[arrowstyle]{stealth reversed}}}}]
\usetikzlibrary{shadings,shapes.geometric,calc, patterns, angles, quotes, arrows.meta, shapes, decorations.pathmorphing, decorations.shapes, decorations.text,calc,angles,quotes,decorations.markings}
\tikzset{>=latex}
\usepackage{chemfig}
\usepackage{multirow}
\usetikzlibrary{quotes,arrows.meta}
\begin{document}
{\kml\large\noindent 
\begin{flushleft}
	សាលាមេតូឌីស្ទកម្ពុជា
\end{flushleft}
\begin{flushright}
ព្រះរាជាណាចក្រកម្ពុជា\\
ជាតិ សាសនា ព្រះមហាក្សត្រ
\end{flushright}}
\begin{center}
{\kml\LARGE កិច្ចតែងការបង្រៀន}
\end{center}
\begin{itemize}
	\item កាលបរិច្ឆេទ ​ថ្ងៃ អង្គារ ទី ២៨ ខែ តុលា ឆ្នាំ ២០១៩ 
	\item មុខវិជ្ជា រូបវិទ្យា{\en(Physics)}
	\item ថ្នាក់ទី ១២ $A$
	\item ជំពូកទី ១ ទែម៉ូឌីណាមិច{\en(Thermodynamics)}
	\item មេរៀនទី១ ទ្រឹស្តីសុីនេទិចនៃឧស្ម័ន{\en(The Kinetic Theory of Gases)}
\end{itemize}
\begin{enumerate}[I]
	\item {\kml វត្ថុបំណងមេរៀន}\quad ក្រោយពីសិក្សាមេរៀននេះចប់សិស្សនឹងអាច
\begin{itemize}
	\item វិជ្ជាសម្បទា ពន្យល់បានពីឯកតានិម្មិត និយមន័យចំនួនកុំផ្លិច និង ឬសការេនៃចំនួន\\
	អវិជ្ជមានបានត្រឹមត្រូវតាមរយៈឧទាហរណ៍
	\item បំណិនសម្បទា បញ្ចេញ និងគណនាការេនៃចំនួនអវិជ្ជមានពីរ៉ាឌីកាល់បានត្រឹមត្រូវ
	ត្រូវតាមរយៈលំហាត់គំរូ
	\item ចរិយាសម្បទា បណ្តុះស្មារតីអោយចូលចិត្តសិក្សាគណិតវិទ្យាតាម​លក្ខណៈជាក្រុម ឬជាបុគ្គល
\end{itemize}
	\item {\kml សម្ភារៈឧបទេស}\quad
\begin{itemize}
	\item [$ - $] សៀវភៅសិក្សាថ្នាក់ទី~១០\quad (~ទំព័រ ~$ 91-94 $~)
	\item [$ - $] សៀវភៅគ្រូទំព័រ ~~(~$ 92-94 $~)~~
	\item [$ - $]​ប័ណ្ណសំណួរ
\end{itemize}
	\item {\kml វិធីសាស្រ្តបង្រៀន}\\[5pt]
	$ -\quad $ តាមវិធីសាស្រ្តចម្រុះ
	\item {\kml វិធីសាស្រ្តបង្រៀន}
\end{enumerate}
\noindent
\begin{tabu} to\linewidth{|X[1,l]|X[1,l]|X[1,l]|} \hline
	\centering {\kml សកម្មភាពគ្រូ }& \centering {\kml ខ្លឹមសារមេរៀន} &\centering {\kml សកម្មភាពសិស្ស } \\ \hline
	&\vspace{-1mm}\centering {\ml ជំហានទី១ }&\\
	&\vspace{-2mm}\centering (~រដ្ឋបាលថ្នាក់~: ~៣ នាទី)&\\
	bត្រូវដាក់លំហាត់អោយសិស្សធ្វើកិច្ចការផ្ទះ& សរសេរកិច្ចការផ្ទះ
\begin{center}
	\begin{tikzpicture}
		\draw (0,0)--(2,0)--(0,3)--cycle;
	\end{tikzpicture}
\end{center}
&b ត្រូវធ្វើកិច្ចការផ្ទះ\\
&\centering {\ml ជំហានទី២ }​&\\
&\vspace{-2mm}\centering (~រំលឹកមេរៀនចាស់~: ~៧ នាទី)&\\
bត្រូវដាក់លំហាត់អោយសិស្សធ្វើកិច្ចការផ្ទះ& សរសេរកិច្ចការផ្ទះ
\begin{center}
	\begin{tikzpicture}
		\draw (0,0)--(2,0)--(0,3)--cycle;
	\end{tikzpicture}
\end{center}
	& ត្រូវធ្វើកិច្ចការផ្ទះ\\
	&\centering {\ml ជំហានទី៣ }​&\\
	&\vspace{-2mm}\centering (~ពង្រឹងពុទ្ធិ~:~៥ នាទី)&\\
	ត្រូវដាក់លំហាត់អោយសិស្សធ្វើកិច្ចការផ្ទះ& សរសេរកិច្ចការផ្ទះ
\begin{center}
	\begin{tikzpicture}
		\draw (0,0)--(2,0)--(0,3)--cycle;
	\end{tikzpicture}
\end{center}
	& ត្រូវធ្វើកិច្ចការផ្ទះ\\\hline
\end{tabu}
\newpage
\noindent
\begin{tabu} to\linewidth{|X[1,l]|X[1,l]|X[1,l]|} \hline
	&\centering {\ml ជំហានទី៤ }​&\\
	&\vspace{-2mm}\centering (~បណ្តាំផ្ញើរ~:~៥ នាទី)&\\
	ត្រូវដាក់លំហាត់អោយសិស្សធ្វើកិច្ចការផ្ទះ& សរសេរកិច្ចការផ្ទះ
\begin{center}
	\begin{tikzpicture}
		\draw (0,0)--(2,0)--(0,3)--cycle;
	\end{tikzpicture}
\end{center}
\begin{enumerate}
	\item
	\item
	\item
\end{enumerate}
&ត្រូវធ្វើកិច្ចការផ្ទះ\\\hline
\end{tabu}
\end{document}